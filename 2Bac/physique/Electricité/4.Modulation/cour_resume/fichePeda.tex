\documentclass[12pt]{article}
\usepackage[a4paper, margin=.20in]{geometry}
%\usepackage{array}
\usepackage{graphicx, subfig, wrapfig, makecell,fancyhdr,xcolor }
\newcommand\headerMe[2]{\noindent{}#1\hfill#2}
\renewcommand \thesection{\Roman{section}}

\newcolumntype{M}[1]{>{\raggedright}m{#1}}

\chead{\includegraphics[width = 0.1\textwidth]{./img/logoMin.png}}
\cfoot{helllo}

\begin{document}

\begin{center}
\includegraphics[width = 0.18\textwidth]{./img/logoMin.png}
\vspace{-3cm}
\end{center}
\headerMe{Matière : Physique-Chimie}{Établissement : \emph{Lycée SKHOR qualifiant}}\\
\headerMe{ Unité : Electricité }{  Professeur :\emph{Zakaria Haouzan}}\\
\headerMe{Niveau : 2BAC-SM-X}{Heure : 12H}\\

\begin{center}
	%\vspace{1cm}
\underline{Leçon N°9: Applications : Production d’ondes électromagnétiques et
communication}\\
Durée 5h00
\\
    \vspace{.2cm}
\hrulefill
\Large{Fiche Pédagogique}
\hrulefill\\
\end{center}
%end Headerss------------------------


%__________________Chimie ______________________-
%%%%%%%+_+_+_+_+_+_+_+_+_Partie1
 \begin{center}
	 \begin{tabular}{|p{0.2\textwidth}||p{0.3\textwidth}||p{0.3\textwidth}||p{0.1\textwidth}|}
\hline
\textbf{Prérequis} & \textbf{Compétences visées } & \textbf{Savoir et savoir-faire}&\textbf{Outils didactiques }\\
    \hline
	
	-les ondes électromagnétiques

	-les ondes lumineuses.

	-La relation entre la longueur d'onde, la fréquence et la vitesse de propagation

	-Réalisation du circuit électrique à partir le schéma.

	-Effectuer des mesures avec un oscilloscope.
				   &
- la compréhension des principes de base de la transmission d'informations à travers les ondes électromagnétiques

- a capacité à utiliser différents types de modulateurs pour transmettre des informations numériques ou analogiques, la connaissance des différents types d'ondes électromagnétiques et de leurs caractéristiques

- la capacité à utiliser des outils mathématiques pour analyser et concevoir des systèmes de transmission d'ondes électromagnétiques.
				   & 

- Connaissance de la manière dont l'information est transmise par une onde électromagnétique.

-connaître les opérations les plus importantes requises pour envoyer un  messages sans fils.

- Connaître l'expression mathématique de la tension sinusoïdale

-Connaître les différents types de modulation de tension sinusoïdale


 & Ordinateur  simulation data-show 

 - GBF oscilloscope AD633 Diode Transistor2N2222

 \\
    \hline
\end{tabular} 
\end{center}
\section*{Situation-problème :}

Depuis toujours, l’Homme a cherché à transmettre des messages, le
plus vite et le plus loin possible... Nous étudierons dans ce chapitre comment peuvent être transmises les
informations d’un bout à l’autre du monde ou même dans l’espace, ainsi que la nécessité de coder puis
décoder l’information pour pouvoir la transmettre. La transmission rapide d’information sans fil est devenue
une banalité 

\begin{enumerate}
	\item sur quel principe repose-t-elle ?
\end{enumerate}

\begin{center}
	 \begin{tabular}{|p{0.2\textwidth}||p{0.3\textwidth}||p{0.3\textwidth}||p{0.1\textwidth}|}
\hline
\multicolumn{4}{|c|}{Déroulement}\\\hline
Eléments du & \multicolumn{2}{c||}{Activités didactiques} &  \\\cline{2-3}
cours & Enseignant & Apprenant & Evaluation\\\hline

\color{red}{I-Introduction}	  &
-Le professeur pose la situation-problème.

-Demande aux apprenants de répondre aux questions de la situation-problème.

-Ecrire les hypothèses proposées par les apprenants.

-Garde les hypothèses convenues pour vérifier pendant
du cours.
				  &
				  -L’apprenant analyse la situation déclenchante
et formule des hypothèses.

\textbf{Exemple des hypothèses attendues :}

---Les ondes électromagnétiques reposent sur le principe de la propagation d'une perturbation électrique et magnétique à travers l'espace. 

---La modulation est le processus par lequel une information est ajoutée à une onde électromagnétique de porteuse pour en transmettre une version modifiée. Il existe différents types de modulation, tels que la modulation en amplitude (AM), la modulation en fréquence (FM) et la modulation en phase (PM)

				  &
				  Evaluation
diagnostique\\\hline


%II- partie 2 
\color{red}{II Les Ondes électromagnétiques- Transmission d’informations.}

\vspace{0.5cm}
\color{blue}1 Définition:
\vspace{0.5cm}

\color{blue}2
La classification des ondes électromagnétiques:
\vspace{0.5cm}

\color{blue}3
Intérêt des ondes électromagnétiques
\vspace{0.5cm}

\color{blue}4
Mise en évidence de la présence d’ondes électromagnétiques :
\vspace{0.5cm}

\color{blue}5
le rôle de la modulation
\vspace{0.5cm}

\color{blue}6
Les différents types de modulations :






				  &

- Le professeur pose la situation problème avec un rappel sur la nature des ondes lumineuse  puis les ondes électromagnétiques

- l'enseignant donne la classification des ondes électromagnétiques ensuite intérêt des ondes électromagnétiques.

-L'enseignant montre l'expérience II.4 mises en évidence de la présence d’ondes électromagnétiques

- L'enseignant introduit des difficultés physiques dans la transmission d'un message, la question de longueur d'onde liée à la longueur d'antenne.

				  &
				  --- L’apprenant répond la question en donnant la
définition des ondes électromagnétiques . 

\vspace{0.5cm}

--- L’apprenant  écrit les partis précédents

\vspace{0.5cm}

---l'apprenant analyse l'expérience II.4 et conclut ensuite qu'elle vérifie bien ce qui a été dit dans la définition des ondes électromagnétiques.

\vspace{0.2cm}

--- L’apprenant  écrit les partis précédents

				  & 
	Évaluation formative\\\hline			  
\end{tabular}
\end{center}

%Page 3 in parte 2

\begin{center}
	 \begin{tabular}{|p{0.2\textwidth}||p{0.3\textwidth}||p{0.3\textwidth}||p{0.1\textwidth}|}
\hline
\multicolumn{4}{|c|}{Déroulement}\\\hline
Eléments du & \multicolumn{2}{c||}{Activités didactiques} &  \\\cline{2-3}
cours & Enseignant & Apprenant & Evaluation\\\hline

%II- partie 2 
\color{red}{III Modulation d’amplitude:}

\vspace{0.5cm}
\color{blue}1
Définition:

\vspace{0.5cm}
\color{blue}2 Le modulateur d’amplitude:

\vspace{0.5cm}
\color{blue}3 Expression de l’amplitude de la tension modulée:


\vspace{0.5cm}
\color{blue}4 	Qualité de modulation 


	  &
---l'enseignant pose la définition de la modulation qui est le processus par lequel une information est incorporée à un signal de porteuse pour la transmettre à travers une distance plus longue sans interférences.


	  \vspace{0.2cm}

---Pour l'expression mathématique de la modulation l'enseignant développe l'expression avec les apprenantes étapes par étapes
	
\vspace{0.5cm}

---l'enseignant donne la qualité du signal modulé puisque la qualité de modulation peut être définie comme la capacité d'un système modulé à transmettre l'information sans erreur ou perte, et peut-être mesurée par des indicateurs tels que la distorsion, le bruit et la perte de synchronisation.

\vspace{0.2cm}


\vspace{0.2cm}







				  &

---\textbf{Interprétation : }

-Les élèves écrivent une conclusion dans le cahier.
\vspace{0.5cm}

--- l'apprenant déduit l'expression mathématique de la modulation puis il arrive à la conclusion que cette fonction contient 3 fréquences avec une variation entre les extrêmes.

\vspace{0.5cm}
---l'apprenant compare les deux cas de modulation et conclut par la suite les Conditions d’obtention d’une bonne modulation
Pour obtenir une modulation d’amplitude de bonne qualité il faut que :

	---  La tension de décalage U0 doit être plus grande à l’amplitude Smde la
tension modulante : $U0 > Sm$ i.e que $m < 1$

	---La fréquence Fp de la tension porteuse doit être supérieure à la
fréquence fS de la tension modulante. $(Fp >> fs)$. Au minimum
$Fp > 10.fs$ 
				  & 
	Évaluation formative\\\hline 
\end{tabular}
\end{center}






\end{document}
