\documentclass[13pt]{article}
\usepackage[a4paper, margin=.20in]{geometry}
%\usepackage{array}
\usepackage{graphicx, subfig, wrapfig, makecell,fancyhdr, xcolor }
\newcommand\headerMe[2]{\noindent{}#1\hfill#2}
\renewcommand \thesection{\Roman{section}}

\newcolumntype{M}[1]{>{\raggedright}m{#1}}

\chead{\includegraphics[width = 0.1\textwidth]{./img/logoMin.png}}
\cfoot{helllo}

\begin{document}

\begin{center}
\includegraphics[width = 0.18\textwidth]{./img/logoMin.png}
\vspace{-3cm}
\end{center}
\headerMe{Matière : Physique-Chimie}{Établissement : \emph{Lycée SKHOR qualifiant}}\\
\headerMe{ Unité :  Transformations nucléaires}{  Professeur :\emph{Zakaria Haouzan}}\\
\headerMe{Niveau : 2BAC-SM-X}{Heure : 10H}\\

\begin{center}
	\vspace{1cm}
\underline{Leçon N°2: Noyau , énergie et masse}\\
Durée 10h00
\\
    \vspace{.2cm}
\hrulefill
\Large{Fiche Pédagogique}
\hrulefill\\
\end{center}
%end Headerss------------------------


%__________________Chimie ______________________-
%%%%%%%+_+_+_+_+_+_+_+_+_Partie1
 \begin{center}
	 \begin{tabular}{|p{0.2\textwidth}||p{0.3\textwidth}||p{0.3\textwidth}||p{0.1\textwidth}|}
\hline
\textbf{Prérequis} & \textbf{Compétences visées } & \textbf{Savoir et savoir-faire}&\textbf{Outils didactiques }\\
    \hline
    -Notions de base de physique nucléaire.

    -Compréhension des concepts de masse et d'énergie

-Connaissance des éléments chimiques et de leur structure atomique
				   
           &
         - Comprendre la relation entre masse et énergie
         
-Analyser les transformations nucléaires

-Exploiter la relation d'Einstein E = mc²

-Comprendre les mécanismes de fission et fusion nucléaire

-Évaluer les effets biologiques de la radioactivité  
				 
 & 
- Définir l'équivalence masse-énergie

- Calculer l'énergie de liaison nucléaire

- Comprendre le défaut de masse

- Expliquer les processus de fission et fusion nucléaire

- Analyser la stabilité des noyaux radioactifs

 & 


     \\
    \hline
\end{tabular} 
\end{center}
\section*{Situation-problème :}

La masse des noyaux atomiques semble différer de la somme des masses de leurs constituants. Comment expliquer cette différence ? Quelle relation existe-t-il entre la masse et l'énergie dans les transformations nucléaires ?

\begin{enumerate}
  \item Qu'est-ce que l'équivalence masse-énergie ?
  \item Comment calculer l'énergie de liaison d'un noyau ?
  \item Quels sont les mécanismes de fission et fusion nucléaire ?
\end{enumerate}

\begin{center}
	 \begin{tabular}{|p{0.2\textwidth}||p{0.3\textwidth}||p{0.3\textwidth}||p{0.1\textwidth}|}
\hline
\multicolumn{4}{|c|}{Déroulement}\\\hline
Eléments du & \multicolumn{2}{c||}{Activités didactiques} &  \\\cline{2-3}
cours & Enseignant & Apprenant & Evaluation\\\hline

\color{red}{I-Introduction :  Equivalence : Masse - Energie

\vspace{0.5cm}
\color{blue} Unités de masse et d’énergie:

\color{magenta}I.2.1 Unité de masse atomique (uma)- (u) 
\color{magenta}I.2.2Unité de l’énergie : Electronvolt
\color{magenta}I.2.3 Energie équivalente à l’unité de masse atomique :
\vspace{0.5cm}




     }	  &

- Présenter la situation-problème

- Inviter les apprenants à formuler des hypothèses

- Guider la réflexion sur la relation masse-énergie

- Présentation de la relation d'Einstein

- Calcul de l'énergie massique

- Unités de masse et d'énergie (uma, électronvolt)

				  &
           - Analyser la situation

           Proposer des hypothèses

Réfléchir aux mécanismes nucléaires

Hypothèses Attendues :

La masse et l'énergie sont équivalentes

Les noyaux peuvent se transformer en libérant ou absorbant de l'énergie

La relation E = mc² permet d'expliquer ces transformations

				  &
				  Evaluation
diagnostique\\\hline


%II- partie 2 
\color{red}{II  Energie de liaison d’un noyau :}

\vspace{0.5cm}
\color{blue}1  Défaut de masse :

\color{blue}2 Energie de liaison:
\vspace{0.5cm}

\color{blue}3 Courbe d’Aston:

				  &
-Le professeur 
Définir le défaut de masse et son lien avec l’énergie de liaison . Calculs pratiques.

-	Étudier les zones de stabilité et d’instabilité des noyaux en fonction du nombre de nucléons .
				  &
				  ---Résoudre des exercices simples pour appliquer les concepts théoriques. 

          ---Analyser la courbe et identifier les noyaux stables et instables.	
				  & 
	Évaluation formative			  
  \\\hline

\color{red}{III  Fission et fusion nucléaire }

&
Décrire et simuler les processus de fission (Uranium-235) et de fusion (Hydrogène → Hélium).
& 
Comparer les deux processus en termes d’énergie libérée et de défis.
&Formative et sommative
\\\hline





\end{tabular}
\end{center}

%Page 3 in parte 2

\begin{center}
	 \begin{tabular}{|p{0.2\textwidth}||p{0.3\textwidth}||p{0.3\textwidth}||p{0.1\textwidth}|}
\hline
\multicolumn{4}{|c|}{Déroulement}\\\hline
Eléments du & \multicolumn{2}{c||}{Activités didactiques} &  \\\cline{2-3}
cours & Enseignant & Apprenant & Evaluation\\\hline

%II- partie 2 
\color{red}{IV Le bilan massique et énergétique d’une réaction nucléaire :}

				  &

Guider le calcul du bilan énergétique

Expliquer la méthode de calcul

Interpréter les résultats

Vérifier la rigueur scientifique

				  &


          —Résoudre des exercices simples pour
appliquer les concepts théoriques.
				  & 
	Évaluation formative\\\hline 
\end{tabular}
\end{center}






\end{document}
