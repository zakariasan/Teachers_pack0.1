\documentclass[12pt]{article}
\usepackage[a4paper, margin=.20in]{geometry}
%\usepackage{array}
\usepackage{graphicx, subfig, wrapfig, makecell,fancyhdr,xcolor }
\newcommand\headerMe[2]{\noindent{}#1\hfill#2}
\renewcommand \thesection{\Roman{section}}

\newcolumntype{M}[1]{>{\raggedright}m{#1}}

\chead{\includegraphics[width = 0.1\textwidth]{./img/logoMin.png}}
\cfoot{helllo}

\begin{document}

\begin{center}
\includegraphics[width = 0.18\textwidth]{./img/logoMin.png}
\vspace{-3cm}
\end{center}
\headerMe{Matière : Physique-Chimie}{Établissement : \emph{Lycée SKHOR qualifiant}}\\
\headerMe{ Unité : Ondes }{  Professeur :\emph{Zakaria Haouzan}}\\
\headerMe{Niveau : 2BAC-SM-X}{Heure : 5H}\\

\begin{center}
	%\vspace{1cm}
\underline{Leçon N°1: Ondes mécaniques progressives.}\\
Durée 5h00
\\
    \vspace{.2cm}
\hrulefill
\Large{Fiche Pédagogique}
\hrulefill\\
\end{center}
%end Headerss------------------------


%__________________Chimie ______________________-
%%%%%%%+_+_+_+_+_+_+_+_+_Partie1
 \begin{center}
	 \begin{tabular}{|p{0.2\textwidth}||p{0.3\textwidth}||p{0.3\textwidth}||p{0.1\textwidth}|}
\hline
\textbf{Prérequis} & \textbf{Compétences visées } & \textbf{Savoir et savoir-faire}&\textbf{Outils didactiques }\\
    \hline
	 -La relation entre la distance
	et la vitesse

	-Energie mécanique d’un
objet solide

-Effectuer des mesures avec un oscilloscope.
				   &
				 -Relier les phénomènes de la
vie quotidienne aux concepts
et principes des Ondes mécanique progressive.

-Résoudre un problème en rapport avec les ondes mécaniques progressives.

-Utiliser la méthode scientifique à différents stades afin d'analyser les différents problèmes liés aux ondes mécaniques progressives.

-Acquisition d'une
méthodologie de recherche
Méthodologie d'action Autoapprentissage

 & 

-Définir une onde mécanique et
sa célérité.

-Définir une onde transversale
et une onde longitudinale.

-Connaître et exploiter les
propriétés générales des
ondes.

-Définir une onde progressive à
une dimension et savoir la
relation entre l'élongation d'un
point du milieu de
propagation et l'élongation de
la source.

-Exploiter la relation entre le
retard temporel, la distance
et la célérité.


 & Ordinateur  simulation data-show 
Corde   ressort 

Cuve à onde  

diapason

la cloche de l’air pompe à vide
\\
    \hline
\end{tabular} 
\end{center}
\section*{Situation-problème :}
La chute d’une goutte d’eau crée à la surface de l’eau une perturbation qui se déplace à une vitesse donnée. Une onde mécanique progressive prend alors naissance.

\begin{enumerate}
\item Qu’est-ce qu’une onde mécanique ?
\item Qu’est-ce qu’une onde mécanique progressive ? quelles sont ses caractéristiques ?
\item comment peut-on mesurer la vitesse de propagation
d’une onde mécanique ?
\end{enumerate}

\begin{center}
	 \begin{tabular}{|p{0.2\textwidth}||p{0.3\textwidth}||p{0.3\textwidth}||p{0.1\textwidth}|}
\hline
\multicolumn{4}{|c|}{Déroulement}\\\hline
Eléments du & \multicolumn{2}{c||}{Activités didactiques} &  \\\cline{2-3}
cours & Enseignant & Apprenant & Evaluation\\\hline

\color{red}{I-Introduction}	  &
-Le professeur pose la situation-problème.

-Demande aux apprenants de répondre aux questions de la situation-problème.

-Ecrire les hypothèses proposées par les apprenants.

-Garde les hypothèses convenues pour vérifier pendant
du cours.
				  &
				  -L’apprenant analyse la situation déclenchante
et formule des hypothèses.

\textbf{Exemple des hypothèses attendues :}

---On appelle onde mécanique le phénomène
de propagation d’une perturbation dans un
milieu matériel, sans transport de matière
(propagation d’énergie).

---Une onde est appelée progressives car la
propagation de la perturbation s'effectue de
proche en proche plus ou moins rapidement.

---La célirité d'une onde progressive est égale
au quotient de la distance séparant deux
points du milieu par la durée qui sépare les
dates de passage de l'onde en ces points. 
				  &
				  Evaluation
diagnostique\\\hline


%II- partie 2 
\color{red}{II Ondes longitudinales, transversales, et leurs caractéristiques.}

\vspace{0.5cm}
\color{blue}1 Propagation d’une onde mécanique le long d’une corde:
\vspace{0.5cm}

\color{blue}2
Propagation d’une onde mécanique à la surface de l’eau:

\vspace{0.5cm}
\color{blue}3
Propagation d’une onde mécanique le long d’un ressort


				  &
-Le professeur pose la question suivante : 

\emph{Qu’est-ce qu’une perturbation ?}

-Le professeur pose la question suivante :

\emph{Qu’est-ce qu’une onde mécanique ?}


-Le professeur pose la question suivante :

\emph{Qu’est-ce qu’une onde
mécanique progressives ?}


-Le professeur pose la question suivante :

\emph{Qu’est-ce qu’une onde transversale?}


-Le professeur pose la question suivante :

\emph{Qu’est-ce qu’une onde longitudinale ?}


				  &
				  --- L’apprenant répond la question en donnant la
définition d’une perturbation:

-\textbf{La perturbation }est une modification locale et
temporaire d’une ou plusieurs propriétés physique
d’un milieu donné.
\vspace{0.5cm}

--- L’apprenant répond la question en donnant la
définition d’une onde mécanique :

-\textbf{Une onde mécanique} est le phénomène de
propagation de la perturbation dans le milieu matériel
élastique, sans transport de matière mais avec
transport d’énergie.
\vspace{0.5cm}

---L’apprenant répond la question en donnant la
définition d’une onde mécanique progressive :

-\textbf{Une onde} est appelée \textbf{progressives} car la
propagation de la perturbation s'effectue de proche en
proche plus ou moins rapidement.
\vspace{0.2cm}

-\textbf{Une onde} est \textbf{longitudinale} quand la direction du
mouvement des éléments du milieu de propagation
est parallèle à la direction de propagation.


				  & 
	Évaluation formative			  
\end{tabular}
\end{center}

%Page 3 in parte 2

\begin{center}
	 \begin{tabular}{|p{0.2\textwidth}||p{0.3\textwidth}||p{0.3\textwidth}||p{0.1\textwidth}|}
\hline
\multicolumn{4}{|c|}{Déroulement}\\\hline
Eléments du & \multicolumn{2}{c||}{Activités didactiques} &  \\\cline{2-3}
cours & Enseignant & Apprenant & Evaluation\\\hline

%II- partie 2 
\color{red}{II Ondes longitudinales, transversales, et leurs caractéristiques.}

\vspace{0.5cm}
\color{blue}4
Les Ondes sonores :

\vspace{0.5cm}
\color{blue}5 Vitesse de propagation d’une onde :
				  &
Activité : 

on allume le téléphone, puis on vide la cloche de l’air par une pompe.

Exp2 : on frappe le diapason 
\vspace{0.2cm}

Exploitation: 

a-Dire ce qui arrive au son émis par le
téléphone lorsqu’on vide de l’air ? Que
concluez-vous ?

b-Dire ce qui arrive à la balle après avoir frappé
le diapason ? Conclure la nature de l’onde
sonore ?
\vspace{0.5cm}

-Le professeur pose la question suivante : 

\emph{Qu’est-ce qu’une vitesse de propagation d’une onde}
\vspace{0.2cm}

-Le professeur pose la simulation et la  question suivante : 

\emph{Qu’est-ce qu’une vitesse de propagation d’une onde}
\vspace{0.2cm}

-Le professeur pose la simulation et la  question suivante : 

\emph{Qu'est-ce qu'un retard temporaire}





				  &

---\textbf{Interprétation : }

a-On observe l’absence de son après le vidage de
l’air, on conclut que le son ne se
propage pas dans le vide mais il nécessite un milieu
matériel pour se propager.
\vspace{0.2cm}

b- Lorsqu’on frappe le diapason, la balle se déplace
horizontalement, ce qui indique
que la direction de perturbation et celle de
propagation sont alignées, donc le son
est une onde longitudinale.
\vspace{0.5cm}

-Les élèves écrivent une conclusion dans le cahier.
\vspace{0.5cm}


--- L’apprenant répond la question en donnant la
définition d’une onde mécanique :

-\textbf{La vitesse de propagation} d’une onde (nommée célérité) est égale à la distance parcourue au temps mis
à la parcourir.

--- L'apprenant répond à la question en donnant la définition du retard temporaire à partir de la définition de la vitesse :
				  & 
	Évaluation formative\\\hline 
\end{tabular}
\end{center}






\end{document}
