\documentclass[12pt]{article}
\usepackage[a4paper, margin=.20in]{geometry}
%\usepackage{array}
\usepackage{graphicx, subfig, wrapfig, makecell,fancyhdr,xcolor,amsmath }
\newcommand\headerMe[2]{\noindent{}#1\hfill#2}
\renewcommand \thesection{\Roman{section}}
\usepackage{colortbl}
\usepackage{booktabs}
\usepackage{multicol}
\usepackage{enumitem}

\newcolumntype{M}[1]{>{\raggedright}m{#1}}

\chead{\includegraphics[width = 0.1\textwidth]{./img/logoMin.png}}
\cfoot{helllo}

\begin{document}

\begin{center}
	\includegraphics[width = 0.18\textwidth]{./img/logoMin.png}
	\vspace{-3cm}
\end{center}
\headerMe{Matière : Physique-Chimie}{Établissement : \emph{Lycée SKHOR qualifiant}}\\
\headerMe{ Unité : Mécanique }{  Professeur :\emph{Zakaria Haouzan}}\\
\headerMe{Niveau : 2BAC-SM-PC}{Heure : 6H}\\

\begin{center}
	%\vspace{1cm}
	\underline{Leçon N°3: Les mouvements plans}\\
	Durée 7h00
	\\
	\vspace{.2cm}
	\hrulefill
	\Large{Fiche Pédagogique}
	\hrulefill\\
\end{center}
%end Headerss------------------------


\section*{Prérequis}
\begin{itemize}
    \item Cinématique du point matériel (position, vitesse, accélération)
    \item Dynamique du point matériel (lois de Newton, forces)
    \item Champ de pesanteur
    \item Notions de base sur les champs électriques et magnétiques
    \item Trigonométrie et vecteurs
\end{itemize}

\section*{Compétences visées}
\begin{itemize}
    \item Caractériser les différents types de mouvements plans
    \item Établir et exploiter les équations horaires d'un mouvement dans un plan
    \item Déterminer les caractéristiques d'un mouvement de projectile dans le champ de pesanteur
    \item Analyser le mouvement d'une particule chargée dans un champ électrique uniforme
    \item Analyser le mouvement d'une particule chargée dans un champ magnétique uniforme
    \item Appliquer ces connaissances à des situations concrètes (spectromètre de masse, cyclotron)
\end{itemize}

\section*{Savoir et savoir-faire}
\begin{itemize}
    \item Déterminer les équations du mouvement d'un solide sur un plan horizontal et sur un plan incliné
    \item Établir et exploiter les équations du mouvement d'un projectile dans le champ de pesanteur
    \item Calculer la portée et la hauteur maximale d'un projectile
    \item Décrire l'action d'un champ magnétique sur une particule chargée en mouvement
    \item Caractériser la force de Lorentz et ses effets sur la trajectoire d'une particule chargée
    \item Appliquer ces connaissances à l'étude du spectromètre de masse et du cyclotron
\end{itemize}

\section*{Outils didactiques}
\begin{itemize}
    \item Ordinateur et logiciels de simulation
    \item Vidéoprojecteur
    \item Tube de Crookes
    \item Bobines d'Helmholtz
    \item Maquette de plan incliné
    \item Rail avec électroaimant pour l'étude du mouvement des projectiles
    \item Maquettes ou animations de spectromètre de masse et de cyclotron
\end{itemize}

\section*{Situation-problème}
Un joueur de basketball tire un ballon vers le panier. Le ballon suit une trajectoire dans l'air avant d'atteindre (ou non) sa cible.
\begin{enumerate}
    \item Comment peut-on décrire mathématiquement la trajectoire du ballon?
    \item Quels sont les facteurs qui influencent la réussite du tir?
    \item Comment déterminer l'angle optimal pour marquer un panier?
\end{enumerate}

\vspace{1cm}

\begin{table}[htbp]
\centering
\begin{tabular}{|p{4cm}|p{5.5cm}|p{5.5cm}|p{1.5cm}|}
\hline
\rowcolor{gray!30} \textbf{Éléments du cours} & \textbf{Activités didactiques de l'enseignant} & \textbf{Activités didactiques de l'apprenant} & \textbf{Évaluation} \\
\hline
\textbf{I. Introduction aux mouvements plans} & 
\begin{itemize}
    \item Présenter la situation-problème 
    \item Demander aux apprenants de proposer des hypothèses pour répondre aux questions
    \item Collecter et discuter les hypothèses proposées
\end{itemize} & 
\begin{itemize}
    \item Analyser la situation et formuler des hypothèses
    \item Proposer des explications sur la trajectoire du ballon
    \item Identifier les paramètres qui peuvent influencer le tir
\end{itemize} &
Évaluation diagnostique \\
\hline
\textbf{II. Mouvement d'un solide sur un plan horizontal et sur un plan incliné (Rappel)} & 
\begin{itemize}
    \item Rappeler les équations du mouvement sur un plan horizontal
    \item Présenter le cas du plan incliné
    \item Poser des questions sur les forces en jeu
\end{itemize} & 
\begin{itemize}
    \item Identifier les forces agissant sur un solide sur un plan horizontal
    \item Analyser les forces sur un plan incliné
    \item Établir les équations du mouvement
\end{itemize} &
Évaluation formative \\
\hline
\textbf{III. Mouvement d'un projectile dans le champ de pesanteur} &
\begin{itemize}
    \item Réaliser l'expérience avec une bille lancée depuis un rail
    \item Demander d'observer et de décrire la trajectoire
    \item Guider l'établissement des équations du mouvement
\end{itemize} & 
\begin{itemize}
    \item Observer la trajectoire parabolique
    \item Appliquer la 2ème loi de Newton
    \item Établir les équations horaires du mouvement
    \item Déterminer l'équation de la trajectoire
\end{itemize} &
Évaluation formative \\
\hline
\textbf{IV. Mouvement d'une particule chargée dans un champ électrique uniforme} & 
\begin{itemize}
    \item Présenter le dispositif expérimental
    \item Expliquer l'action d'un champ électrique sur une particule chargée
    \item Guider l'établissement des équations du mouvement
\end{itemize} & 
\begin{itemize}
    \item Identifier la force électrique
    \item Appliquer la 2ème loi de Newton
    \item Établir les équations du mouvement
    \item Caractériser la trajectoire
\end{itemize} &
Évaluation formative \\
\hline
\textbf{V. Mouvement d'une particule chargée dans un champ magnétique uniforme} & 
\begin{itemize}
    \item Réaliser l'expérience avec un tube de Crookes
    \item Expliquer la force de Lorentz
    \item Guider l'analyse du mouvement
    \item Démontrer que le mouvement est circulaire uniforme
\end{itemize} & 
\begin{itemize}
    \item Observer la déviation des électrons
    \item Caractériser la force magnétique
    \item Démontrer que le mouvement est circulaire uniforme
    \item Calculer le rayon de la trajectoire
\end{itemize} &
Évaluation formative \\
\hline
\textbf{VI. Applications: Spectromètre de masse et cyclotron} & 
\begin{itemize}
    \item Présenter le principe du spectromètre de masse
    \item Expliquer le fonctionnement du cyclotron
    \item Faire le lien avec les concepts étudiés
\end{itemize} & 
\begin{itemize}
    \item Comprendre le principe de séparation des isotopes
    \item Analyser le mouvement des particules dans ces dispositifs
    \item Calculer les paramètres pertinents
\end{itemize} &
Évaluation sommative \\
\hline
\textbf{VII. Conclusion et évaluation} & 
\begin{itemize}
    \item Synthétiser les concepts clés
    \item Proposer des exercices d'application
    \item Revenir sur la situation-problème initiale
\end{itemize} & 
\begin{itemize}
    \item Résoudre les exercices proposés
    \item Répondre aux questions de la situation-problème
    \item Formuler une conclusion
\end{itemize} &
Évaluation sommative \\
\hline
\end{tabular}
\end{table}

\section*{Déroulement détaillé}

\subsection*{I. Introduction aux mouvements plans}

\begin{itemize}
    \item Définir ce qu'est un mouvement plan (mouvement dont la trajectoire est située dans un plan)
    \item Présenter les différents types de mouvements plans qui seront étudiés:
    \begin{itemize}
        \item Mouvement sur un plan horizontal et sur un plan incliné
        \item Mouvement d'un projectile dans le champ de pesanteur
        \item Mouvement d'une particule chargée dans un champ électrique uniforme
        \item Mouvement d'une particule chargée dans un champ magnétique uniforme
    \end{itemize}
    \item Discuter les applications pratiques de l'étude des mouvements plans (balistique, technologies, sports, etc.)
\end{itemize}

\subsection*{II. Mouvement d'un solide sur un plan horizontal et sur un plan incliné (Rappel)}

\begin{itemize}
    \item Rappeler les équations du mouvement rectiligne uniformément varié
    \item Analyser les forces en jeu sur un plan horizontal (poids, réaction normale, frottements)
    \item Analyser les forces en jeu sur un plan incliné (composantes du poids, réaction normale, frottements)
    \item Établir les équations du mouvement dans chaque cas
\end{itemize}

\subsection*{III. Mouvement d'un projectile dans le champ de pesanteur}

\begin{itemize}
    \item Réaliser l'expérience de la bille lancée depuis un rail
    \item Observer et caractériser la trajectoire parabolique
    \item Établir le bilan des forces (poids uniquement si on néglige les frottements de l'air)
    \item Appliquer la 2ème loi de Newton pour établir les équations du mouvement:
    \begin{align*}
        \begin{cases}
            a_x &= 0 \\
            a_y &= -g
        \end{cases}
    \end{align*}
    \item Intégrer ces équations pour obtenir les composantes de la vitesse:
    \begin{align*}
        \begin{cases}
            v_x &= v_0\cos\alpha \\
            v_y &= -gt + v_0\sin\alpha
        \end{cases}
    \end{align*}
    \item Intégrer à nouveau pour obtenir les équations horaires:
    \begin{align*}
        \begin{cases}
            x &= v_0\cos\alpha \cdot t \\
            y &= -\frac{1}{2}gt^2 + v_0\sin\alpha \cdot t
        \end{cases}
    \end{align*}
    \item Éliminer le paramètre t pour obtenir l'équation de la trajectoire:
    \begin{align*}
        y = -\frac{g}{2v_0^2\cos^2\alpha}x^2 + x\tan\alpha
    \end{align*}
    \item Déterminer les coordonnées du sommet S de la trajectoire:
    \begin{align*}
        x_S &= \frac{v_0^2\sin(2\alpha)}{2g} \\
        y_S &= \frac{v_0^2\sin^2\alpha}{2g}
    \end{align*}
    \item Calculer la portée du projectile:
    \begin{align*}
        x_P = \frac{v_0^2\sin(2\alpha)}{g}
    \end{align*}
    \item Déterminer l'angle optimal pour une portée maximale: $\alpha = \frac{\pi}{4}$
\end{itemize}

\subsection*{IV. Mouvement d'une particule chargée dans un champ électrique uniforme}

\begin{itemize}
    \item Définir le champ électrique uniforme
    \item Caractériser la force électrique: $\vec{F} = q\vec{E}$
    \item Établir les équations du mouvement en appliquant la 2ème loi de Newton
    \item Montrer que le mouvement est uniformément accéléré
    \item Établir l'analogie avec la chute libre dans le champ de pesanteur
\end{itemize}

\subsection*{V. Mouvement d'une particule chargée dans un champ magnétique uniforme}

\begin{itemize}
    \item Définir le champ magnétique uniforme
    \item Présenter l'expérience du tube de Crookes
    \item Observer la déviation des électrons dans différentes configurations
    \item Caractériser la force magnétique (force de Lorentz): $\vec{F} = q\vec{v} \wedge \vec{B}$
    \item Analyser les caractéristiques de cette force:
    \begin{itemize}
        \item Direction: perpendiculaire au plan $(\vec{v}, \vec{B})$
        \item Sens: selon la règle de la main droite
        \item Intensité: $F = |q|vB\sin(\vec{v},\vec{B})$
    \end{itemize}
    \item Démontrer que le mouvement est uniforme:
    \begin{itemize}
        \item La force de Lorentz est perpendiculaire à la vitesse
        \item Son travail est nul: $W(\vec{F}) = 0$
        \item L'énergie cinétique reste constante: $\Delta E_c = 0$
        \item La vitesse est constante: $v = cte$
    \end{itemize}
    \item Démontrer que le mouvement est plan:
    \begin{itemize}
        \item L'accélération tangentielle est nulle: $a_t = \frac{dv}{dt} = 0$
        \item L'accélération est normale
        \item Le mouvement se fait dans un plan perpendiculaire à $\vec{B}$
    \end{itemize}
    \item Démontrer que le mouvement est circulaire:
    \begin{itemize}
        \item En appliquant la 2ème loi de Newton: $q\vec{v} \wedge \vec{B} = m\vec{a}$
        \item En utilisant le repère de Frenet: $|q|vB = m\frac{v^2}{R}$
        \item Calculer le rayon de la trajectoire: $R = \frac{mv}{|q|B}$
    \end{itemize}
    \item Analyser la déviation magnétique:
    \begin{itemize}
        \item Établir la relation: $D_m = \frac{l \cdot D \cdot |q| \cdot B}{m \cdot v_0}$
    \end{itemize}
\end{itemize}

\subsection*{VI. Applications: Spectromètre de masse et cyclotron}

\begin{itemize}
    \item Présenter le principe du spectromètre de masse:
    \begin{itemize}
        \item Description des différentes parties (chambre d'ionisation, d'accélération, de séparation)
        \item Principe de séparation des isotopes
        \item Calcul du diamètre de la trajectoire: $D = 2R = \frac{2mv_0}{|q|B}$
    \end{itemize}
    \item Expliquer le fonctionnement du cyclotron:
    \begin{itemize}
        \item Description du dispositif (dees, champ magnétique, oscillateur)
        \item Principe d'accélération des particules
        \item Synchronisation du champ électrique alternatif
    \end{itemize}
\end{itemize}

\subsection*{VII. Conclusion et évaluation}

\begin{itemize}
    \item Synthétiser les concepts clés des mouvements plans étudiés
    \item Revenir sur la situation-problème initiale:
    \begin{itemize}
        \item Trajectoire parabolique du ballon de basketball
        \item Facteurs influençant la réussite du tir (vitesse initiale, angle, hauteur de lancer)
        \item Détermination de l'angle optimal
    \end{itemize}
    \item Proposer des exercices d'application pour chaque type de mouvement étudié
    \item Évaluer la compréhension des concepts par les apprenants
\end{itemize}



\end{document}
