\documentclass[12pt]{article}
\usepackage[a4paper, margin=.20in]{geometry}
%\usepackage{array}
\usepackage{graphicx, subfig, wrapfig, makecell,fancyhdr,xcolor }
\newcommand\headerMe[2]{\noindent{}#1\hfill#2}
\renewcommand \thesection{\Roman{section}}

\newcolumntype{M}[1]{>{\raggedright}m{#1}}

\chead{\includegraphics[width = 0.1\textwidth]{./img/logoMin.png}}
\cfoot{helllo}

\begin{document}

\begin{center}
\includegraphics[width = 0.18\textwidth]{./img/logoMin.png}
\vspace{-3cm}
\end{center}
\headerMe{Matière : Physique-Chimie}{Établissement : \emph{Lycée SKHOR qualifiant}}\\
\headerMe{ Unité : Ondes }{  Professeur :\emph{Zakaria Haouzan}}\\
\headerMe{Niveau : 2BAC-SM-X}{Heure : 5H}\\

\begin{center}
	%\vspace{1cm}
	\underline{Leçon $N^{\circ}$3: Propagation d'une onde lumineuse.}\\
Durée 5h00
\\
    \vspace{.2cm}
\hrulefill
\Large{Fiche Pédagogique}
\hrulefill\\
\end{center}
%end Headerss------------------------


%__________________Chimie ______________________-
%%%%%%%+_+_+_+_+_+_+_+_+_Partie1
 \begin{center}
	 \begin{tabular}{|p{0.2\textwidth}||p{0.3\textwidth}||p{0.3\textwidth}||p{0.1\textwidth}|}
\hline
\textbf{Prérequis} & \textbf{Compétences visées } & \textbf{Savoir et savoir-faire}&\textbf{Outils didactiques }\\
    \hline

---Connaitre et utiliser la relation : $v$=$\lambda.N$ connaitre la signification et l’unité de chaque terme,
savoir justifier cette relation par une équation aux dimensions

--- Savoir, pour une longueur d’onde donnée, que le phénomène de diffraction est d’autant plus
marqué que la dimension d’une ouverture ou d’un obstacle est plus petite




				   &
				 ---Relier les phénomènes de la
vie quotidienne aux concepts
et principes des Ondes lumineuses.

---Résoudre un problème en rapport avec les ondes lumineuses.

---Utiliser la méthode scientifique à différents stades afin d'analyser les différents problèmes liés aux ondes lumineuses.

---Acquisition d'une
méthodologie de recherche
Méthodologie d'action Autoapprentissage

 & 
--- Connaitre l’importance de la dimension de l’ouverture ou de l’obstacle sur le phénomène de diffraction

---Exploiter une figure de diffractions dans le cas des ondes lumineuses.

---Connaitre et savoir utiliser la relation $c = \lambda.N$ , la signification et l’unité de chaque terme

---Connaitre et savoir utiliser la relation $\theta$=$\frac{\lambda}{a}$ la signification et l’unité de chaque terme

---Définir une lumière monochromatique et une lumière polychromatique

---Connaitre les limites des longueurs d’onde dans le vide du spectre visible les couleurs correspondantes

---Savoir que les milieux transparents sont plus ou moins dispersifs

---Définir l’indice d’un milieu transparent pour une fréquence donnée

 & 

---Ordinateur  

---simulation data-show 

---source laser

---écran

---Obstacle avec un trou

---prisme
\\
    \hline
\end{tabular} 
\end{center}
\section*{Situation-problème :}
L’arc en ciel provient de la lumière du soleil qui rencontre les gouttelettes d’eau.


\begin{enumerate}
	\item La lumière est-elle une onde ?
	\item Comment expliquer le phénomène de l’arc en ciel ?
\end{enumerate}

\begin{center}
	 \begin{tabular}{|p{0.2\textwidth}||p{0.3\textwidth}||p{0.3\textwidth}||p{0.1\textwidth}|}
\hline
\multicolumn{4}{|c|}{Déroulement}\\\hline
Eléments du & \multicolumn{2}{c||}{Activités didactiques} &  \\\cline{2-3}
cours & Enseignant & Apprenant & Evaluation\\\hline

\color{red}{I- Mise en évidence expérimentale de la diffraction de la lumière}	 

\vspace{0.5cm}
\color{blue}I.1- Expérience:
\vspace{0.5cm}

\color{blue}I.2- Conclusion: :
\vspace{0.5cm}

\color{blue}I.3- Etude de la diffraction d’un faisceau laser par une fente::
\vspace{0.5cm}

	  &
---Le professeur pose la situation-problème avec la simulation.

---Le professeur Demande aux apprenants de répondre aux questions de la situation-problème.

---Ecrire les hypothèses proposées par les apprenants.

---Garde les hypothèses convenues pour vérifier pendant
du cours.

---Le professeur pose l'Expérience suivante :
On réalise le montage précédent en plaçant l’écran, maintenu fixe , à une distance D fixe de
l’objet diffractant . On fait une série de mesure de la largeur L de la tache centrale pour des fentes (
ou des fils calibrés ) de largeur  différentes.


--- Le professeur Demande aux apprenants de tracer la courbe de la variation de l’écart angulaire en fonction de $\frac{1}{a}$

--- Déduire la relation liant $\theta$, a et $\lambda$ ,

				  &
				  -L’apprenant analyse la situation déclenchante
et formule des hypothèses.

\textbf{Exemple des hypothèses attendues :}

---La largeur de la tâche centrale augmente lorsque la largeur de la fente diminue.

---Le phénomène de diffraction montre que la lumière a un aspect ondulatoire.

---La lumière peut donc être caractérisée comme toutes les ondes, par sa célérité, sa fréquence et sa
longueur d’onde.

\vspace{0.5cm}

---L’apprenant répond les questions : On constate que $\theta$ est proportionnelle à, $\frac{1}{a}$


donc la relation entre $\lambda$ et $\theta$ : $\theta $=$\frac{L}{a}$


---La largeur a de la fente (épaisseur du fil) : plus que a diminue , plus que L augmente, donc plus que
la diffraction est importante . d’où le phénomène de diffraction de l’onde lumineuse est inversement
proportionnel à la largeur de la fente a

--- La longueur d’onde $\lambda$ de la lumière laser : plus que augmente, plus que la diffraction de l’onde
lumineuse est importante , d’où le phénomène de diffraction est proportionnel à $\lambda$

---La distance D entre l’obstacle et l’écran : plus que D augmente, plus que la diffraction de l’onde
lumineuse est importante, d’où le phénomène de diffraction est proportionnel à D
				  &
				  Evaluation
diagnostique\\\hline


%II- partie 2 
\color{red}{II-  Caractéristiques des ondes lumineuses}
%\vspace{0.5cm}

\color{blue}1-Onde électromagnétique:
%\vspace{0.5cm}

\color{blue}2
Lumière monochromatique et lumière polychromatique:

\color{blue}3
Indice de réfraction d’un milieu transparent :

\color{blue}4
Réfraction de la lumière :

%\vspace{0.5cm}
%\color{red}{III- Détermination expérimentale de la célérité de propagation
%d’une onde sonore :  }

				  &
-Le professeur  donne des rappeles sur la nature de la lumière et sur l'indice de réfraction d’un milieu transparent

%\emph{Décrire le mouvement du point S}

%-Le professeur pose la question suivante :

%\emph{Comparer le mouvement de deux points M et N du milieu de propagation dans chacun des cas }


%-Le professeur la corde avec un stroboscope de fréquence réglable Ne :

%\emph{Quelles sont les fréquences Ne des éclairs qui donnent l’immobilité apparente de la corde ? en
%déduire la fréquence maximale }



				  &
   %               --- L’apprenant répond la question :  Le point S a un mouvement rectiligne sinusoïdal
%\vspace{0.5cm}

%--- L’apprenant répond la question : les deux points M et N vibrent en phase
%si la distance qui les sépare est un multiple de la longueur d’onde

%\vspace{0.5cm}

%---L’apprenant répond la question : L’immobilité apparente est obtenue lorsque la fréquence de l’onde N est un multiple de la
%fréquence des éclairs Ne : N= k Ne
%\vspace{0.2cm}
---Répondre aux questionnaires orientées

---Les élèves écrivent une conclu-
sion dans le cahier.
				  & 
	Évaluation formative\\\hline		  
\end{tabular}
\end{center}

%Page 3 in parte 2

\begin{center}
	 \begin{tabular}{|p{0.2\textwidth}||p{0.3\textwidth}||p{0.3\textwidth}||p{0.1\textwidth}|}
\hline
\multicolumn{4}{|c|}{Déroulement}\\\hline
Eléments du & \multicolumn{2}{c||}{Activités didactiques} &  \\\cline{2-3}
cours & Enseignant & Apprenant & Evaluation\\\hline

%II- partie 2 
\color{red}{III Dispersion de la lumière:}

\vspace{0.5cm}
\color{blue}1
Le prisme:

\vspace{0.5cm}
\color{blue}2 Trajet d’un faisceau lumineux à travers le prisme: 

\vspace{0.5cm}
\color{blue}2 	Les relations du prisme:	
	  &
Activité : 

On remplace le laser par une source de lumière blanche et la fente par un prisme de verre . on observe que si on
fixe la valeur de l’angle d’incidence i , la valeur de l’angle de réfraction r varie lorsque la fréquence de la radiation
incidente varie



\vspace{0.2cm}
-Le professeur pose la question suivante : 

Qu’observe-t-on sur l’écran placé devant le prisme

Montrer que l’indice de réfraction dépend de la fréquence de
la radiation qui traverse le milieu






				  &

---\textbf{Interprétation : }

Le prisme est un milieu transparent et homogène limité par deux faces planes non parallèles, la face opposée
à l’arête est la base du prisme.

\vspace{0.2cm}
En appliquant la loi de réfraction sur la première face du prisme: sini = n.sinr 

n appliquant la loi de réfraction sur la deuxième face du prisme: nsinr' = sini'

D = i'+i -A et A = r+r'


-Les élèves écrivent une conclusion dans le cahier.
\vspace{0.5cm}

				  & 
	Évaluation formative\\\hline 
\end{tabular}
\end{center}






\end{document}
