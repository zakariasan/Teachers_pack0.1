\documentclass[12pt]{article}
\usepackage[a4paper, margin=.20in]{geometry}
%\usepackage{array}
\usepackage{graphicx, subfig, wrapfig, makecell,fancyhdr,xcolor }
\newcommand\headerMe[2]{\noindent{}#1\hfill#2}
\renewcommand \thesection{\Roman{section}}

\newcolumntype{M}[1]{>{\raggedright}m{#1}}

\chead{\includegraphics[width = 0.1\textwidth]{./img/logoMin.png}}
\cfoot{helllo}

\begin{document}

\begin{center}
\includegraphics[width = 0.18\textwidth]{./img/logoMin.png}
\vspace{-3cm}
\end{center}
\headerMe{Matière : Physique-Chimie}{Établissement : \emph{Lycée SKHOR qualifiant}}\\
\headerMe{ Unité : Ondes }{  Professeur :\emph{Zakaria Haouzan}}\\
\headerMe{Niveau : 2BAC-SM-X}{Heure : 5H}\\

\begin{center}
	%\vspace{1cm}
	\underline{Leçon $N^{\circ}$2: Ondes mécaniques progressives périodiques. }\\
Durée 5h00
\\
    \vspace{.2cm}
\hrulefill
\Large{Fiche Pédagogique}
\hrulefill\\
\end{center}
%end Headerss------------------------


%__________________Chimie ______________________-
%%%%%%%+_+_+_+_+_+_+_+_+_Partie1
 \begin{center}
	 \begin{tabular}{|p{0.2\textwidth}||p{0.3\textwidth}||p{0.3\textwidth}||p{0.1\textwidth}|}
\hline
\textbf{Prérequis} & \textbf{Compétences visées } & \textbf{Savoir et savoir-faire}&\textbf{Outils didactiques }\\
    \hline

-Connaître et exploiter les
propriétés générales des
ondes.

-Définir une onde progressive à
une dimension et savoir la
relation entre l'élongation d'un
point du milieu de
propagation et l'élongation de
la source.

-Exploiter la relation entre le
retard temporel, la distance
et la célérité.



				   &
				 -Relier les phénomènes de la
vie quotidienne aux concepts
et principes des Ondes mécanique progressive périodique.

-Résoudre un problème en rapport avec les ondes mécaniques progressives périodique.

-Utiliser la méthode scientifique à différents stades afin d'analyser les différents problèmes liés aux ondes mécaniques progressives périodique.

-Acquisition d'une
méthodologie de recherche
Méthodologie d'action Autoapprentissage

 & 
--- Reconnaitre une onde progressive périodique et sa période

--- Définir pour une onde périodique sinusoïdale, la période, la fréquence, la longueur d’onde

---Connaitre et utiliser la relation : $v$=$\lambda.N$ connaitre la signification et l’unité de chaque terme,
savoir justifier cette relation par une équation aux dimensions

--- Savoir, pour une longueur d’onde donnée, que le phénomène de diffraction est d’autant plus
marqué que la dimension d’une ouverture ou d’un obstacle est plus petite

 & Ordinateur  simulation data-show 
Corde ---ressort 

---Cuve à onde  

---diapason

--- micro et Oscilloscope

--- GBF et Stroboscope

\\
    \hline
\end{tabular} 
\end{center}
\section*{Situation-problème :}
Les cordes vocales de la cantatrice sont des sources d’ondes sonores périodiques se
propageant jusqu’aux oreilles des spectateurs. Ces ondes peuvent rencontrer un obstacle ou
une ouverture de petite dimension

\begin{enumerate}
\item  Qu’est-ce qu’une onde mécanique progressive périodique ?
\item Quelles sont ses caractéristiques ?
\item Que se passe-t-il lorsqu’une onde rencontre un obstacle ou un trou de faible dimension ?
\item La célérité d’une onde dépend elle de sa fréquence ?
\end{enumerate}

\begin{center}
	 \begin{tabular}{|p{0.2\textwidth}||p{0.3\textwidth}||p{0.3\textwidth}||p{0.1\textwidth}|}
\hline
\multicolumn{4}{|c|}{Déroulement}\\\hline
Eléments du & \multicolumn{2}{c||}{Activités didactiques} &  \\\cline{2-3}
cours & Enseignant & Apprenant & Evaluation\\\hline

\color{red}{I- Notion d’onde mécanique progressive périodique}	 

\vspace{0.5cm}
\color{blue}I.1- Définition :
\vspace{0.5cm}

\color{blue}I.2- Périodicité temporelle :
\vspace{0.5cm}

\color{blue}I.3- Périodicité spatiale :
\vspace{0.5cm}

	  &
-Le professeur pose la situation-problème avec la simulation.

-Demande aux apprenants de répondre aux questions de la situation-problème.

-Ecrire les hypothèses proposées par les apprenants.

-Garde les hypothèses convenues pour vérifier pendant
du cours.
				  &
				  -L’apprenant analyse la situation déclenchante
et formule des hypothèses.

\textbf{Exemple des hypothèses attendues :}

--- Une onde progressive périodique est une onde mécanique qui est créée par une source qui a un mouvement périodique.

---Cette onde est caractérisée par une double périodicité, temporelle et spatiale.

--- Le phénomène de diffraction se produit lorsqu'une onde rencontre une ouverture ou un obstacle de faible dimension par rapport à sa longueur d'onde. On observe alors un étalement des directions de propagation de l'onde. Cet étalement produit une figure composée de franges alternativement intenses et peu intenses.

--- La célérité d'une onde dans un milieu dépend de l'état du milieu et de la fréquence de l'onde dans le cas des milieux dispersifs. Elle ne dépend pas de son amplitude.
				  &
				  Evaluation
diagnostique\\\hline


%II- partie 2 
\color{red}{II- Exemples d’ondes mécaniques progressives périodiques :  }

\vspace{0.5cm}
\color{blue}1 Onde progressive le long d’une corde :
\vspace{0.5cm}

\color{blue}2
Onde progressive à la surface de l’eau :

\vspace{0.5cm}
\color{blue}3
Ondes sonores et ultrasonores :

\vspace{0.5cm}
\color{red}{III- Détermination expérimentale de la célérité de propagation
d’une onde sonore :  }

				  &
-Le professeur propose l'experience suivante : Observation stroboscopique d’une onde le long de la corde

Un vibreur provoque une onde périodique sinusoïdale, de fréquence , qui se propage le
long d’une corde élastique à la vitesse V , à partir d’un point S (source d’onde)

\emph{Décrire le mouvement du point S}

-Le professeur pose la question suivante :

\emph{Comparer le mouvement de deux points M et N du milieu de propagation dans chacun des cas }


-Le professeur la corde avec un stroboscope de fréquence réglable Ne :

\emph{Quelles sont les fréquences Ne des éclairs qui donnent l’immobilité apparente de la corde ? en
déduire la fréquence maximale }



				  &
				  --- L’apprenant répond la question :  Le point S a un mouvement rectiligne sinusoïdal
\vspace{0.5cm}

--- L’apprenant répond la question : les deux points M et N vibrent en phase
si la distance qui les sépare est un multiple de la longueur d’onde

\vspace{0.5cm}

---L’apprenant répond la question : L’immobilité apparente est obtenue lorsque la fréquence de l’onde N est un multiple de la
fréquence des éclairs Ne : N= k Ne
\vspace{0.2cm}

				  & 
	Évaluation formative\\\hline		  
\end{tabular}
\end{center}

%Page 3 in parte 2

\begin{center}
	 \begin{tabular}{|p{0.2\textwidth}||p{0.3\textwidth}||p{0.3\textwidth}||p{0.1\textwidth}|}
\hline
\multicolumn{4}{|c|}{Déroulement}\\\hline
Eléments du & \multicolumn{2}{c||}{Activités didactiques} &  \\\cline{2-3}
cours & Enseignant & Apprenant & Evaluation\\\hline

%II- partie 2 
\color{red}{IV Phénomène de diffraction:}

\vspace{0.5cm}
\color{blue}1
Définition

\vspace{0.5cm}
\color{blue}2 Onde diffractée àla surface de l’eau : 
				  &
Activité : 
Une lame vibrante munie d’une réglette, produit des ondes planes dans une cuve à onde, qui
progressent à la surface de l’eau sous formes de rides rectilignes, avec une vitesse v = 1 m/s
.
on éclaire la surface de l’eau avec un stroboscope de telle sorte que sa fréquence soit égale à celle
des ondes ( Ne = N = 100 Hz ) , et voit que tous les points de la surface de l’eau apparaissent
immobiles.
On place deux plaques parallèles dans la cuve de manière à former une fente de largeur a

-Le professeur pose la question suivante : 



\vspace{0.2cm}
Calculer la longueur
d’onde de l’onde
incidente et la comparer
à la largeur a de la fente
dans chaque cas ;

\vspace{0.2cm}



				  &

---\textbf{Interprétation : }

l'onde n'est plus rectiligne au-delà de l’ouverture, la fente se comporte comme
une source ponctuelle donnant naissance à des ondes circulaires

\vspace{0.2cm}

la condition pour que les ondes soient diffractées est : $a < \lambda$

-Les élèves écrivent une conclusion dans le cahier.
\vspace{0.5cm}

				  & 
	Évaluation formative\\\hline 
\end{tabular}
\end{center}






\end{document}
