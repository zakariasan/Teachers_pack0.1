\documentclass[12pt]{article}
\usepackage[a4paper, total={7.5in, 11in}]{geometry}
%\usepackage{array}
\usepackage{graphicx, subfig, wrapfig, fancyhdr, lastpage, multicol ,color,arydshln,makecell,chemfig}
\usepackage[most]{tcolorbox}
\newcommand\headerMe[2]{\noindent{}#1\hfill#2}
\usepackage[mathscr]{euscript}
\usepackage{tabularray}

\setlength{\columnseprule}{1pt}
\def\columnseprulecolor{\color{blue}}


\pagestyle{fancy}
\fancyhf{}

\cfoot{ \vspace{-0.8cm}\em{Page \thepage \hspace{1pt} / \pageref{LastPage}}}


\newtcolorbox{Box2}[2][
enhanced, 
    breakable,
]{
                lower separated=false,
                colback=white,
colframe=white!20!black,fonttitle=\bfseries,
colbacktitle=white!30!gray,
coltitle=black,
enhanced,
attach boxed title to top left={yshift=-0.1in,xshift=0.15in},
title=#2,#1
}
%    \vspace{.2cm}



\begin{document}

\headerMe{Royaume du Maroc}{année scolaire \emph{2023-2024}}\\
\headerMe{Ministère de l'Éducation nationale, }{  }\\
\headerMe{du Préscolaire et des Sports}{Établissement : \emph{Lycée SKHOR qualifiant}}\\
%\vspace{-1cm}
\begin{center}
%Devoir Surveillé  N°1 \\
	Soutien\\
    2ème année baccalauréat Sciences Mathématiques\\
%Durée 2h00
%    \vspace{.2cm}
\hrulefill
  \Large{--Chimie(SN2023)--}
\hrulefill\\

    \emph{Les  parties sont indépendantes}
    %\emph{Les deux parties sont indépendantes}

    \vspace{-.2cm}
\end{center}
%end Headerss------------------------
%__________________Chimie ______________________-
%%%%%%%+_+_+_+_+_+_+_+_+_Partie1
\begin{Box2}{SN2023:Verification de la masse }
 \emph{ \textbf{L’acide propanoïque $C_2H_5COOH$ est un liquide que l’on prépare au laboratoire. Il est utilisé comme
agent conservateur et entre dans la composition de certains médicaments et dans la synthèse de certains
arômes. Cette partie consiste à vérifier, par dosage,}} la masse de l’acide propanoïque dans un médicament.

\textbf{Données :}
\begin{itemize}
  \item Le produit ionique de l’eau: $K_e = 10.^{-14}$ à 25°C.
  \item La masse molaire de l’acide propanoïque $M(C_2H_5COOH) = 74 g/mol$.
\end{itemize}

Le médicament étudié est une solution aqueuse notée $(S)$ . Son étiquette descriptive indique la présence de $46,2mg$
d’acide propanoïque dans un volume $V = 40mL$de cette solution.

Pour vérifier cette indication, on prépare, à 25°C, une solution A $(S)$ en introduisant dans un bécher un
volume $V_A = 10mL$ de la solution (S) auquel on ajoute $V_e =  50mL$ d’eau distillée.

On dose l’acide propanoïque présent dans $(S_A)$ à l’aide d’une solution aqueuse $(S_B)$ d’hydroxyde de sodium $({N_a^+}_{(aq)} + HO^-_{(aq)}  )$ de concentration molaire $C_B = 2,0.10^{-2}mol/L$.

Après l’ajout d’un volume $V_{B_1}= 3,9mL$ de la solution d’hydroxyde de sodium au mélange, la mesure 
du pH du mélange réactionnel donne la valeur $pH_1 = 4,86$

A l’équivalence, le volume de la solution d’hydroxyde de sodium ajouté est $V_{BE} = 7,8mL$

\begin{tabular}{c|l}
0,25	& \makecell[l]{\textbf{1. }Ecrire l’équation modélisant la réaction qui a lieu lors du dosage. }\\ 

	0,5 & \makecell[l]{\textbf{2. }Expliquer pourquoi l’ajout du volume $V_e$ d’eau distillée n’influe pas sur la valeur du volume \\de la solution d’hydroxyde de sodium ajouté à l’équivalence. }\\

	0,75 & \makecell[l]{\textbf{3. }En se basant sur le tableau d’avancement de la réaction du dosage, trouver l’expression du \\ taux d’avancement final de la réaction avant l’équivalence en fonction du pH du milieu\\réactionnel, $K_e$ ,$C_B$,$V_A$ ,$V_e$ et $V_B$.le volume de la solution d’hydroxyde de sodium ajouté.\\Calculer sa valeur après l’ajout de $V_{B1}$ et conclure. }\\

  0,75 & \makecell[l]{\textbf{4. }Calculer, après l’ajout du volume $V_B =V_{B_1}$ , les concentrations  $[C_2H_5COOH]$ et $[C_2H_5COO^-]$ \\Déduire la valeur du $pK_A(C_2H_5COOH/C_2H_5COO^-)$.  } \\

  0,5 & \makecell[l]{\textbf{5. }Justifier la nature basique du mélange réactionnel à l’équivalence. } \\
  
  0,75 & \makecell[l]{\textbf{6. }Calculer le pH de la solution (S).} \\
  0,5 & \makecell[l]{\textbf{7. }Vérifier que la masse de l’acide propanoïque est celle indiquée sur l’étiquette. } \\


\end{tabular}
\end{Box2}

\begin{Box2}{SN2020:Dosage de l’acide lactique dans un lait.}
  \emph{L’acidité d’un lait augmente par fermentation lactique en cas de mauvaise conservation. Le dosage de
l’acide lactique de formule $CH_3-CHOH-COOH$ permet donc d’apprécier l’état de conservation du lait.
Moins le lait est frais, plus il contient de l’acide lactique.}
  
On se propose de doser l’acide lactique présent dans un lait de vache, qui n’a subit aucun traitement, par une
solution aqueuse d’hydroxyde de sodium. On supposera que l’acidité du lait est due uniquement à l’acide
lactique.

L’acide lactique sera simplement noté HA.
  \textbf{Données : }
\begin{itemize}
  \item Toutes les mesures sont effectuées à $25^{\circ}C$
  \item Le produit ionique de l’eau: $K_e = 10.^{-14}$ à 25°C.
  \item La masse molaire de l’acide lactique :$90 g/mol$.
 
\end{itemize}

  \textbf{1- Préparation de la solution aqueuse d’hydroxyde de sodium :}
  On prépare une solution aqueuse $(S_B )$ d’hydroxyde de sodium $(Na^+_{(aq)} + HO^-_{(aq)})$ de volume $V=1,0L$ et de concentration molaire $C_B$, par dissolution d'une masse de soude dans de l'eau distillée.

  la mesure du pH de la solution $S_B$ donne $pH = 12,70$.

\begin{enumerate}
  \item[1-1] Etablir l’expression du pH de la solution $(S_B)$ en fonction de $K_e$ et de $C_B$.\textbf{(0,5pt)}
  \item[1-2] Vérifier que $C_B \simeq 5,0. 10^{-2} mol/L$.\textbf{(0,25pt)}
\end{enumerate}

  \textbf{2-Contrôle de la qualité d’un lait de vache}

  \begin{wrapfigure}[10]{r}{0.44\textwidth}
  \begin{center}
	  \vspace{-2cm}
	\includegraphics[width=0.5\textwidth]{./img/new_vache.png}
  \end{center}
\end{wrapfigure}



  Un technicien de laboratoire dose l’acidité d’un lait de vache. Il réalise le titrage pH-métrique à l’aide de la
  solution aqueuse $(S_B)$ d’hydroxyde de sodium de concentration molaire $C_B$. Pour cela il introduit , dans un
bécher un volume $V_A = 25,0mL$
de lait, puis il verse progressivement un volume $V_B$
de la solution $(S_B)$ et note
pour chaque volume versé le pH du mélange réactionnel.

On note $V_{BE}$
le volume de la solution d’hydroxyde de sodium versé à l’équivalence et
  $K_A$ la constante d’acidité du couple $HA_{(aq)}/ A^-_{(aq)}$.

  \begin{enumerate}
    \item[2-1] Ecrire l’équation chimique modélisant la réaction du dosage. (0,5pt)
    \item[2-2] Etablir la relation permettant de déterminer la concentration $C_A$ en acide lactique du lait en fonction de $V_A, C_B$ et $V_BE$.(0,5pt)
    \item[2-3] Etablir la relation : $V_B.10^{-pH} = K_A.(V_{BE} - V_B)$ avec $0<V_B<V_{BE}$.(0,75pt)
    \item[2-4] La courbe de la figure 1 représente les variations de $10^{-pH}$ en fonction de $V_B$ : $10^{-pH}.V_B$=$ f(V_B)$.En s’aidant de la courbe de la figure 1.
      \begin{enumerate}
        \item[2-4-1] Déterminer le volume $V_{BE}$ et en déduire la concentration $C_A$.(0.5pt)
        \item[2-4-2] déterminer le $pK_A$ du couple $HA_{(aq)}/ A^-_{(aq)}$ (0.5pt)
      \end{enumerate}
  \end{enumerate}


\end{Box2}

 \section*{Partie 2 : Etude de la pile plomb–étain}
 Les piles électrochimiques sont l’une des applications des réactions d’oxydo-réduction. Au cours de leur
fonctionnements, une partie de l’énergie chimique se transforme en énergie électrique.

On réalise, à 25 $^{\circ}C$ , la pile plomb–étain en plongeant une plaque de plomb dans un bécher contenant un 
volume $V_1 $=$30mL$ d’une solution aqueuse de nitrate de plomb $({Pb^{2+}}_{(aq)} + {2NO_3^-}_{(aq)} )$
de concentration molaire initiale $C_1$=$[Pb^{2+}]_0$ 


et en plongeant une plaque d’étain dans un autre bécher contenant un volume $V_2=V_1$ d’une solution aqueuse de chlorure d’étain II $(Sn^{2+}_{aq} + 2Cl^-_{(aq)})$
de concentration molaire initiale $C_2 $=$[Sn^{2+}]_0$=$C_1$ . 

Les deux solutions sont reliées par un pont salin contenant une solution saturée de chlorure d’ammonium $NH^+_{4(aq)} + Cl^-_{(aq)}$

On monte en série entre les pôles de la pile, un conducteur ohmique (D) , un ampèremètre et un interrupteur

On ferme l’interrupteur à l’instant $t = 0$, un courant d’intensité $I = 17,13mA$ circule alors dans le circuit.

La courbe ci-contre représente l’évolution temporelle de la concentration des ions $Sn^{2+}_{(aq)}$.

\textbf{Donnée : }

 \begin{itemize}
	 \item un Volume $V_A = 8,6mL$ d'acide éthanoique (de formule chimique $C_2H_4O_2$ et de densité par rapport à l'eau d=1,05).
	 \item un Volume $V_B =13,8mL$ de l'alcool iso-amylique de formule $C_5H_12O$ (soit 0,15mol).
	 %\item $\chemfig{CH_3-C(=[::+60]O)(-[::-60]O-CH_2-CH_2-C(-[2]CH_3)-CH_3}$
	 \item (\textbf{l'acétate d'iso-amyle})  $\chemfig{CH_3-C(=[::+60]O)(-[::-60]O-CH_2-CH_2-CH(-[2]CH_3)-CH_3)}$

 \end{itemize}

 On donne  : $M(H)=1g/mol$ ; $M(C)=12g/mol$ ; $M(O)=16g/mol$ et $\rho_{eau} = 1g/mol$

\begin{tabular}{c|l}
0,75	& \makecell[l]{\textbf{1. }Montrer que le mélange initial (acide + alcool) est équimolaire.}\\ 
	0,5 & \makecell[l]{\textbf{2.a. }Citer les trois principales propriétés de cette réaction.}\\

	1 & \makecell[l]{\textbf{2.c. }Déterminer la quantité maximale d'acétate d'iso-amyle que peut synthétiser ce chimiste \\sachant que la constante d'équilibre de la réaction de synthèse de l'ester est égale à 4.}\\

	0,75 & \makecell[l]{\textbf{3. }Afin d'améliorer le rendement de cette réaction, le chimiste pense aux opérations suivantes: \\
		\textbf{ - ajouter un catalyseur : l'acide sulfurique concentré par exemple}\\
	\textbf{- réaliser une distillation fractionnée consistant à éliminer progressivement} l'eau formée.
Parmi ces deux propositions , choisir en justifiant celle qui vous semble raisonnable.}\\


		0,5 & \makecell[l]{\textbf{3. } Le chimiste a réaliser une autre expérience en remplaçant l'acide éthanoique par son \\anhydride. Ecrire l'équation de la réaction qui se produit.}\\
\end{tabular}


\begin{center}
%Devoir Surveillé  N°1 \\
%Durée 2h00
    %\vspace{.2cm}
\hrulefill
\Large{--Physique (13 points)--}
\hrulefill\\

    \emph{Les  parties sont indépendantes}
    %\emph{Les deux parties sont indépendantes}
\end{center}


 \section*{Partie 1: Application des ondes ultrasonores\dotfill (2,5 POINTS)  }

 \begin{tcolorbox}
 \emph{Actuellement certaines voitures sont équipées de plusieurs capteurs tels que les capteurs ultrasons
et les capteurs LASER. Ces capteurs servent à faciliter le contrôle de son environnement proche
qui peut atteindre une distance de 2m.}

\emph{Le tableau de bord de ces nouvelles gammes de voitures est très développé, il est constitué d'un
ensemble d'indicateurs et de témoins qui renseignent le conducteur sur le fonctionnement du
moteur et sur les paramètres de conduite (vitesse instantanée, température extérieure ...). Certains
circuits électroniques du tableau de bord comportent des condensateurs, des bobines ... Le confort
dans ces voitures est assuré par plusieurs éléments et accessoires parmi lesquels, les amortisseurs
qui utilisent des ressorts.}
\end{tcolorbox}


\begin{wrapfigure}{r}{0.25\textwidth}
  \begin{center}
	  \vspace{-1cm}
	\includegraphics[width=0.25\textwidth]{./img/ondes01.png}
  \end{center}
\end{wrapfigure}

Une voiture est équipée d’un système comportant un émetteur (E) et un récepteur (R) d’ultrasons
placés côte à côte à l’arrière du véhicule.

Lors d'un stationnement, l'émetteur (E) envoie des ultrasons sous forme de salves. Ces ultrasons sont
captés par le récepteur (R) après réflexion sur un obstacle situé à la distance d de (E).

\textbf{Données : }Vitesse de propagation des ultrasons dans l’air : $v_0$=$340 m.s^{-1}$.

\begin{tabular}{c|l}
	1  & \makecell[l]{\textbf{1. }Répondre par vrai ou faux aux propositions a, b, c et d suivantes: }\\
\end{tabular}

\begin{center}
	\begin{tabular}{|c|l|}
 a & L'onde ultrasonore est une onde longitudinale \\
 b & L'onde ultrasonore se propage dans le vide \\  
 c & La propagation des ultrasons se fait avec transport de matière \\
 d & La fréquence des ultrasons varie en changeant le milieu de propagation \\  
\end{tabular}
\end{center}


\begin{tabular}{c|l}
	& \makecell[l]{\textbf{2. }L'oscillogramme ci-contre donne le signal émis par l'émetteur (E)
et le signal réfléchi par \\le récepteur (R).
 }\\ 

	0,5 & \makecell[l]{\textbf{2.1. }Déterminer graphiquement la durée $\tau$ entre le signal émis et le signal reçu.
 }\\
	0,75 & \makecell[l]{\textbf{2.2. }Calculer la distance d qui sépare l’obstacle de l'émetteur (E). }\\
	0,25 & \makecell[l]{\textbf{2.3. }On considère un point M du milieu de propagation qui se	trouve à la distance $EM = \frac{d}{2}$ \\de l'émetteur (E).Recopier sur votre copie le numéro de la question et écrire la lettre
\\correspondante à la proposition vraie :L'élongation $y_M(t)$ de M en fonction de l'élongation de \\l'émetteur (E) est: }\\
\end{tabular}


\begin{center}
	\begin{tabular}{|c|c|}
		
a & $y_M(t) = y_E(t-\tau)$ \\
b & $y_M(t) = y_E(t-\frac{\tau}{2})$ \\
c & $y_M(t) = y_E(t - \frac{\tau}{4})$ \\
d & $y_M(t) = y_E(t-2.\tau)$ \\ 
\end{tabular}
\end{center}

%\hrulefill
%\Large{Physique 13pts/78min}
%\hrulefill\\
\section*{Partie 2 :Étude de la cible de berkélium 249\dotfill(1,5pts)}

\begin{tcolorbox}
La  première  étape  de  la  synthèse  de  l’élément  117  a  consisté  en  la  fabrication  du berkélium : un mélange de curium et d’américium a été irradié durant 250 jours par un intense  flux  de  neutrons  [...].  Il  a  fallu  ensuite 90  jours  pour  séparer  et  purifier  les  22 milligrammes de berkélium produits. [...] Ce précieux élément, déposé sur un film de titane, [...] a été soumis, 150 jours durant, au flux de calcium. " Il fallait faire vite, selon Hervé  Savajols,  chercheur  au  Grand  Accélérateur  national  d’ions  lourds  (GANIL),  car l’isotope du berkélium utilisé ayant une période de 320 jours, à la fin de l’expérience, il ne restait que 70\% du berkélium initial ".
\end{tcolorbox}


\begin{tabular}{c|l}
	0,25  & \makecell[l]{\textbf{1. }On  donne  l’équation  incomplète  de  la  désintégration  du  noyau  de  berkélium 249 :\\ $^{249}_{97}Bk \rightarrow ^{249}_{98}Cf + ......$ \\En précisant les lois de conservation utilisées, identifier la particule émise.  De quel type de \\radioactivité s’agit-il ici ?  }\\

		0,25 & \makecell[l]{\textbf{2. }Sachant  que  le  bombardement  de  la  cible  de  berkélium  a  duré  150 jours,  vérifier  l’affirmation  :\\ "   À  la  fin  de  l’expérience,  il  ne  restait  que 70\% du berkélium initial "}\\
	& \makecell[l]{\textbf{3. }Activité de la source de berkélium de masse égale à 22 mg :  }\\

		0,5 & \makecell[l]{\textbf{3.1 }Déterminer  le  nombre  initial  $N_0$  de  noyaux  de  berkélium  249  dans l’échantillon  produit  \\sachant  que  la  masse  d’un  atome  de  berkélium 249 est $m_{atome} = 4,136.10^{-25}Kg$.  }\\
		0,5 & \makecell[l]{\textbf{3.2 }Exprimer  l’activité  initiale  $a_0$  de  l’échantillon  de  berkélium  249  en fonction de $N_0$ et $t_{1/2}$. \\La calculer en becquerel.}\\

\end{tabular}

\section*{Partie 3 :Application des ondes ultrasonores \dotfill(5pts)}

\emph{Beaucoup d’appareils électriques contiennent des circuits qui se composent de condensateurs, de
bobines, de conducteurs ohmiques ...La fonction de ces composantes varie selon leurs domaines
d’utilisation et la façon dont elles sont montées dans les circuits.}

Cet partie a pour objectifs :
\begin{itemize}
	\item de vérifier les caractéristiques d’une bobine (b) et de l’utiliser dans un circuit RLC série.
	\item  d’étudier la modulation d’amplitude.
\end{itemize}


\textbf{\underline{I- Détermination des caractéristiques d’une bobine.}}

On réalise le montage expérimental représenté sur la figure 1 comprenant :
\begin{itemize}
	\item  Une bobine (b) d’inductance L et de résistance r.
	\item  Un conducteur ohmique (D) de résistance R.
	\item  Un générateur de tension (G) de force électromotrice E.
	\item  Un ampèremètre (A) de résistance négligeable.
	\item  Un interrupteur K.
\end{itemize}

A l’instant t = 0, on ferme l’interrupteur K , et on visualise à l’aide d’un oscilloscope à mémoire les
variations de la tension $u_{PQ}(t)$ entre les pôles du générateur (G) et de la tension $u_R(t)$ entre les bornes du conducteur ohmique (D). On obtient les courbes 1 et 2 représentées sur la figure 2.


\begin{center}
  \includegraphics[width=0.8\textwidth]{./img/Rl01.png}
\end{center}



La droite (T) représente la tangente à la courbe 2 à l’instant t=0 .
Dans le régime permanent, l’ampèremètre (A) indique la valeur I = 0,2A.

\begin{tabular}{c|l}
	0,5  & \makecell[l]{\textbf{1. }Montrer que l’équation différentielle que vérifie la tension $u_R$ s’écrit sous la forme \\: $L.\frac{du_R}{dt} + (R+r).u_R - E.R = 0$}\\
	0,5 & \makecell[l]{\textbf{2. }Sachant que la solution de l’équation différentielle s’écrit sous la forme : $u_R = U_0.(1-e^{-\frac{t}{\tau}})$,\\trouver l’expression des constantes $U_0$ et $\tau$ en fonction des paramètres du circuit.}\\
		0,75 & \makecell[l]{\textbf{3. }Trouver l’expression de la résistance r de la bobine (b) en fonction de E , I et $U_0$. Calculer \\la valeur
de r.}\\
			0,5 & \makecell[l]{\textbf{4. }Déterminer graphiquement la valeur numérique de t et vérifier que la valeur de l’inductance \\L de la
bobine est $L=0,4H$}\\
		\end{tabular}




\textbf{\underline{II- Etude du dipôle RLC}}

On réalise le montage représenté sur la figure 3 qui comprend une bobine (b), le générateur (G) de force électromotrice E, un condensateur de capacité C, un conducteur ohmique de résistance $R' = 10 \Omega$ et un interrupteur K.

Après avoir chargé totalement le condensateur, on bascule l’interrupteur K à la position 2 à l’instant $t = 0$ et on visualise à l’aide d’un oscilloscope à mémoire les variations de la tension $u_c$ aux bornes du
condensateur en fonction du temps .On obtient l’oscillogramme représenté sur la figure 4.

\begin{center}
  \includegraphics[width=0.8\textwidth]{./img/rlc_01.png}
\end{center}

\begin{tabular}{c|l}
	0,25  & \makecell[l]{\textbf{1. }Donner le nom du régime associé à la courbe de la figure 4. }\\
	0,25 & \makecell[l]{\textbf{2. }Déterminer graphiquement la pseudo-période T.}\\
	0,5 & \makecell[l]{\textbf{3. }On suppose que la pseudo-période est égale à la période propre T0 de l’oscillateur électrique.
\\Déduire la valeur de la capacité C du condensateur }\\
\\
		\end{tabular}


\textbf{\underline{III - Etude de la modulation d’amplitude}}

Afin d’obtenir un signal modulé en amplitude, on utilise
un circuit intégré multiplieur $X$ (fig.6).
On applique à l’entrée :

- $E_1$: la tension $u_1(t) = s(t) + U_0$ avec $s(t) = S_m.cos(2.\pi.f_s.t)$représentant le signal informatif et $U_0$ une composante
continue de la tension. 

- $E_2$: une tension sinusoïdale représentant la porteuse $u_2 = U_m.cos(2.\pi.F_p.t)$

- La tension de sortie $u_s(t)$ obtenue est  $u_s(t) = k.u_1(t).u_2(t)$.

- k est une constante qui dépend du circuit intégré X.
- Rappel: $2cos(a).cos(b) = cos(a+b) + cos(a-b)$

\begin{tabular}{c|l}
	1  & \makecell[l]{\textbf{1. }Montrer que us(t) s’écrit sous la forme : \\$u_s(t) = \frac{A.m}{2}.cos(2.\pi.f_1.t) + A.cos(2.\pi.f_2.t) + \frac{A.m}{2}.cos(2.\pi.f_3.t)$ , où m est le taux de modulation\\ et A une constante.}\\
	0,75 & \makecell[l]{\textbf{2. }La figure 7 représente le spectre de fréquences formé de trois raies de la tension modulée us(t). \\Déterminer m et
la fréquence fs. La modulation est-elle bonne ?}\\
		\end{tabular}


\begin{center}
  \includegraphics[width=0.3\textwidth]{./img/mod01.png}
  \includegraphics[width=0.3\textwidth]{./img/mod02.png}
\end{center}






\section*{Partie 4 :Etude du mouvement d’un pendule pesant \dotfill(2,75pts)}

\begin{wrapfigure}{r}{0.16\textwidth}
	\vspace{-1.2cm}
\begin{center}
  \includegraphics[width=0.16\textwidth]{./img/pendule011.png}
\end{center}
\end{wrapfigure}

Un pendule pesant, de centre d’inertie G et de masse m, constitué d’une tige et d’un corps solide (S).
Ce pendule peut effectuer un mouvement de rotation autour d’un axe horizontal ( D) fixe passant par l’extrémité O de la tige (figure 1 ).
On désigne par $J_{\Delta}$  le moment d’inertie du pendule pesant par rapport à l’axe $(\Delta)$ et par L la distance
séparant G de l’axe ($\Delta$).

\textbf{Données : }

\begin{itemize}
	\item $g \approx 10 m.s^{-2}$ ; $m = 400g$ ; $L=50cm$.
	\item Pour les oscillations de faible amplitude on prendra : $sin(\theta) \approx \theta$ et $1-cos(\theta) \approx \frac{\theta^2}{2}$ avec $\theta $ en radian.
	\item $\pi^2 \approx 10$
\end{itemize}

On écarte le pendule de sa position d’équilibre stable, dans le sens positif,
d’un angle $\theta_m$ très petit, puis on le lâche sans vitesse initiale à l’instant $t=0$.

A chaque instant, la position du pendule est repérée par son abscisse
angulaire $\theta$. On néglige les frottements et on travaille dans l’approximation
de faibles oscillations
\begin{center}
\textbf{Etude dynamique}
\end{center}
\begin{tabular}{c|l}
	0,5  & \makecell[l]{\textbf{1. }Trouver en appliquant la relation fondamentale de la dynamique, l’équation différentielle du
\\mouvement du pendule pesant. }\\
		0,5  & \makecell[l]{\textbf{2. }Trouver l’expression de la période propre $T_0$ de ce pendule en fonction de m , g , L et $J_{\Delta}$ pour \\que
		la solution de l’équation différentielle s’écrit sous la forme $\theta(t) = \theta_m.cos(\frac{2.\pi}{T_0}.t + \phi)$. }\\
	0,25  & \makecell[l]{\textbf{3. }Vérifier par une analyse dimensionnelle que l’expression de $T_0$ a la dimension du temps.}\\

	0,5  & \makecell[l]{\textbf{4. }Sachant que la valeur de la période propre est $T_0 \approx 0,7s$ . Calculer $J_{\Delta}$ }\\
	\end{tabular}
\begin{center}
\textbf{Etude énergétique}
\end{center}
On choisit le plan horizontal passant par le point $G_0$, position de G à l’équilibre stable, comme état de
référence de l’énergie potentielle de pesanteur $E_{pp}(\theta = 0) = 0$.

La figure 2 représente le diagramme d’énergie du pendule étudié.


\begin{center}
  \includegraphics[width=0.35\textwidth]{./img/pendule_pesa.png}
\end{center}





\begin{tabular}{c|l}
	  & \makecell[l]{\textbf{1. }Déterminer la valeur de :}\\
		0,25  & \makecell[l]{\textbf{1.1. }L'abscisse angulaire maximale $\theta_m$}\\
	0,25  & \makecell[l]{\textbf{1.2. }L’énergie mécanique $E_m$ du pendule.}\\

	0,5  & \makecell[l]{\textbf{2. }Calculer les deux abscisses angulaires $\theta_1$ et $\theta_2$ pour lesquelles l’énergie potentielle est égale
\\l’énergie cinétique. } \\
	\end{tabular}




\section*{Partie 5: Test d'amortissement d'une voiture \dotfill(1,25pts)  }

\begin{wrapfigure}{r}{0.24\textwidth}
	\vspace{-1.4cm}
\begin{center}
  \includegraphics[width=0.24\textwidth]{./img/pendule22.png}
\end{center}
\end{wrapfigure}


Pour les voitures, le système d'amortissement permet d'atténuer les oscillations verticales se produisant sur la route. Ce système se compose au niveau
de chaque roue d'un ressort et d'un amortisseur
(généralement à huile).

On modélise la voiture par un solide (S)
de masse m et de centre d'inertie G
qui repose sur un ressort vertical de constante de raideur K (figure 3).

Pour étudier le système oscillant $({solide (S) +  ressort})$, on
 repère les positions de G
par son ordonnée y sur un axe
vertical $Oy$ orienté vers le haut. L'origine O est choisi à la
position d'équilibre de G.


\begin{tabular}{c|l}
	
	 0,5 & \makecell[l]{\textbf{1. }Recopier sur votre copie le numéro de la question et écrire la lettre \\correspondante à \\la proposition vraie:
		 L'expression de la période propre $T_0$ des oscillations libres du système \\oscillant est:\\ a:$T_0 = 2\pi.\frac{K}{m}$ \hspace{1cm} b:$T_0 = 2\pi.\sqrt{\frac{K}{m}}$ \hspace{1cm} c:$T_0 = 2\pi.\sqrt{\frac{m}{K}}$ \hspace{1cm} d:$T_0 = 2\pi.\sqrt{K.m}$ 
 \\Justifier la réponse par analyse dimensionnelle.}\\

	  & \makecell[l]{\textbf{2. }Lors d'un test du système d'amortissement de deux voitures $(V_1)$ et $(V_2 )$, On a relevé les \\variations $y(t)$ des positions du centre d'inertie G de chaque voiture. Ces variations sont \\indiquées sur la figure (4) pour les deux voitures.}\\


	 0,5 & \makecell[l]{\textbf{2.1 }On considère que la pseudo période T est égale à la période propre $T_0$ de l'oscillateur. \\Calculer la valeur de la raideur K , sachant que $ m=1300 kg$ (on prendra $\pi^2 \approx 10 $).}\\

 0,25 & \makecell[l]{\textbf{2.2 }Indiquer, en justifiant la réponse, la voiture qui présente plus de confort.}\\
\end{tabular}

\begin{center}
  \includegraphics[width=0.48\textwidth]{./img/oscillo.png}
\end{center}

\end{document}
