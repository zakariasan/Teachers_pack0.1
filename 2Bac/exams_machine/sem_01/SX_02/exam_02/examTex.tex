\documentclass[12pt]{article}
\usepackage[a4paper, total={7.5in, 11in}]{geometry}
%\usepackage{array}
\usepackage{graphicx, subfig, wrapfig, fancyhdr, lastpage, multicol ,color,arydshln,makecell}
\newcommand\headerMe[2]{\noindent{}#1\hfill#2}
\usepackage[mathscr]{euscript}
\usepackage{tabularray}

\setlength{\columnseprule}{1pt}
\def\columnseprulecolor{\color{blue}}


\pagestyle{fancy}
\fancyhf{}

\cfoot{ \vspace{-0.8cm}\em{Page \thepage \hspace{1pt} / \pageref{LastPage}}}
\begin{document}

\headerMe{Royaume du Maroc}{année scolaire \emph{2023-2024}}\\
\headerMe{Ministère de l'Éducation nationale, }{  Professeur :\emph{Zakaria Haouzan}}\\
\headerMe{du Préscolaire et des Sports}{Établissement : \emph{Lycée SKHOR qualifiant}}\\
%\vspace{-1cm}
\begin{center}
Devoir Surveillé  N°2 \\
    2ème année baccalauréat Sciences physiques\\
Durée 2h00
\\
    \vspace{.2cm}
\hrulefill
\Large{Chimie 7pts - 45min}
\hrulefill\\

\end{center}
%end Headerss------------------------
%__________________Chimie ______________________-
%%%%%%%+_+_+_+_+_+_+_+_+_Partie1

 \section*{Transformations non totales d'un système chimique\dotfill(7pts)-45min }
%\begin{wrapfigure}{r}{0.16\textwidth}
	%\vspace{-1.2cm}
%%\begin{center}
  %%\includegraphics[width=0.16\textwidth]{./img/chimie01.png}
%%\end{center}
%\end{wrapfigure}


%\begin{wrapfigure}[1]{r}{0.5\textwidth}
	%\vspace{0.5cm}
%\begin{center}
  %\includegraphics[width=0.5\textwidth]{./img/chimie02.png}
%\end{center}
%\end{wrapfigure}

\textit{L'acide nitreux a pour formule $HNO_2$. Son nom systématique est : acide dioxonitrique (III) ou acide nitrique
III, Ses sels sont appelés nitrites.
Le but de cet exercice est de savoir si cet acide se dissocie totalement ou partiellement dans l’eau ainsi que de
prévoir l’effet de sa dilution sur le taux d’avancent final et sur la constante d’équilibre.}


	\begin{tabular}{c|l}
		0,5 & \makecell[l]{\textbf{1. }Définir ce qu’un acide au sens de Bronsted.}\\

		0,75 & \makecell[l]{\textbf{2. }Ecrire la réaction de l'acide nitreux avec l’eau et déterminer les couples acide/base mis en jeu.}\\
	\end{tabular}

\textbf{3. }On introduit une masse m de l'acide nitreux dans de l'eau distillée pour obtenir une solution aqueuse de
volume $V_1 = 200 mL$ et de concentration molaire en soluté apporté $C_1$. La mesure du pH de la solution donne
$pH= 2,53$ . La masse molaire de l’acide nitreux vaut : $M = 47g.mol^{-1}$.

Une étude expérimentale a montré que le taux d’avancement final vaut $\tau_1 = 14,62 \%$.

	\begin{tabular}{c|l}
		0,5 & \makecell[l]{\textbf{3.1.}La transformation étudiée est-elle limitée ou totale ? justifier.}\\

		0,5 & \makecell[l]{\textbf{3.2.}Dresser le tableau d’avancement de la réaction en question en fonction de C,V , x et xéq. }\\
		
		0,5 & \makecell[l]{\textbf{3.3.}Montrer que le taux d’avancement final de cette réaction peut \\s’écrire sous la forme: $\tau_1 = \frac{1}{C_1.10^{pH}}$}\\
		
		0,75 & \makecell[l]{\textbf{3.4.}Déduire la valeur de $C_1$ et la valeur de la masse m.}\\
		
		0,5 & \makecell[l]{\textbf{3.5.}Calculer la conductivité $\sigma$ de cette solution à l’équilibre. ( Rappel $1m^3=10^3$ ) }\\
		
		1 & \makecell[l]{\textbf{3.6.}Montrer que le quotient de la réaction à l’équilibre s’écrit : $Q_{r,eq} = \frac{C.\tau^2}{1 - \tau}$ Déduire la valeur \\de $K_1$ la constante d’équilibre liée à sa réaction. }\\
	\end{tabular}

\textbf{4. }On dilue la solution aqueuse d'acide nitreux \textbf{20 fois} , on obtient une solution (S2) de concentration molaire
apportée $C_2$ . La mesure de $pH$ de cette solution à l’équilibre donne la valeur $pH_2= 3,3$.

	\begin{tabular}{c|l}
		0,75 & \makecell[l]{\textbf{4.1.}Calculer $C_2$. En déduire la valeur de $\tau_2$ le taux d’avancement final.}\\

		0,5 & \makecell[l]{\textbf{4.2.}Montrer que la constante d’équilibre associée à cette réaction s’écrit : \\$K_2 = \frac{10^{-2pH}}{C-10^{-pH}}$.Calculer sa valeur.}\\
		
		0,25 & \makecell[l]{\textbf{4.3.}En déduire l’effet de la dilution sur le taux d’avancement final et sur la constante d’équilibre. }\\
		
		0,5 & \makecell[l]{\textbf{5.}On réalise une autre étude en utilisant une solution d’acide éthanoïque de concentration $C_1$.\\Soit $\tau_3$ le taux d’avancement\\
		final de la réaction d’acide éthanoïque avec l’eau . Sachant que la constante d’équilibre\\
		associée à cette réaction est $K_3 = 1,78.10^{-5}$.
		\\. Choisir la bonne réponse : \\ (a)- $\tau_3 > \tau_1$ \\ (b)- $\tau_3 < \tau_1$ \\ (c)- $\tau_3 = \tau_1$}\\
		
	\end{tabular}

	Les conductivités molaires ioniques : $\lambda_{H_3O^+}$=$35 mS.m^2/mol$ ; $\lambda_{NO^-_2}$=$7,18 mS.m^2/mol$
%\hrulefill
%\Large{Physique 13pts/78min}
%\hrulefill\\
\newpage
\begin{center}
    %\vspace{.60cm}
\hrulefill
\Large{Physique 13pts - 75min}
\hrulefill\\
    \emph{Les  parties sont indépendantes}
\end{center}

%\vspace{-1cm}
\section*{Partie 1 : La radioactivité au service de la médecine\dotfill(4pts)}

%\begin{wrapfigure}[2]{r}{0.19\textwidth}
  %\begin{center}
	  %\vspace{-2cm}
	%\includegraphics[width=0.19\textwidth]{./img/ex6.png}
  %\end{center}
%\end{wrapfigure}

La médecine est l’un des domaines qui a connu l’application de la radioactivité en utilisant des noyaux
radioactifs pour diagnostiquer et traité des maladies, l’un des noyaux utilisés est le rhénium 186 dans
le but de soulager les malades atteints de polyarthrite rhumatoïde

Les données : La constante radioactive du rhénium $_{75}^{186}Re$ est $\lambda = 2,2.10^{-6}s^{-1}= 0,19jour^{-1}$

\textbf{1. La désintégration d’un noyau de rhénium $_{75}^{186}Re$.}

\begin{tabular}{c|l}

	1 & \makecell[l]{\textbf{1.1 }Donner la composition du noyau du rhénium $_{75}^{186}Re$.}\\

	1 & \makecell[l]{\textbf{1.2 }La désintégration du noyau de rhénium $_{75}^{186}Re$ donne un noyau d’osmium $_{76}^{186}Os$.\\
Ecrire
l’équation de désintégration du rhénium et déterminer la nature de cette désintégration}\\
	\end{tabular}

	\vspace{0.5cm}
\textbf{2. Injection locale d’une solution contenant du rhénium 186.
Le produit injectable se présente sous la forme d’une solution contenue dans un flacon de volume $V_0= 10 mL$ ayant une activité $a_0 = 4.10^9Bq$ à la date $t=0$, c'est-à-dire à la sortie du laboratoire pharmaceutique.}
	\begin{tabular}{c|l}

		1 & \makecell[l]{\textbf{2.1 }Déterminer en jours la valeur de demi-vie $t_{1/2}$ du rhénium $_{75}^{186}Re$}\\

		0,5 & \makecell[l]{\textbf{2.2 }Trouver, à l’instant $t_1 = 4,8jours$, le nombre $N_1$ de noyau de rhénium contenu dans le flacon.}\\

		0,5 & \makecell[l]{\textbf{3.2 } À l’instant $t_1$ on prélève du flacon de volume $V_0 = 10mL$ une injection de volume V contenant \\$N = 3,65.10^{13}$ noyaux de rhénium 186, on l’injecte à un malade dans l’articulation de l’épaule, \\trouver la valeur de V.}\\
	\end{tabular}


\section*{Partie 2 :  Centrale nucléaire \dotfill(9pts)}
Dans une centrale nucléaire, les noyaux d'uranium $^{235}_{92}U$ subissent la fission sous le choc d'un neutron
lent. Un des nombreux processus possibles conduit à la formation d'un noyau de lanthane $^{144}_{57}La$ ,d'un noyau de brome $^{88}_{35}Br$ et  de plusieurs neutrons.

\begin{tabular}{c|l}

 1& \makecell[l]{\textbf{1. } Définissez l'énergie de liaison d'un noyau.}\\

 1 & \makecell[l]{\textbf{2. } Donnez l'expression littérale qui permettra son calcul.}\\

 1 & \makecell[l]{\textbf{3. } Calculez, en MeV, l'énergie de liaison d’un noyau $^{235}_{92}U$.}\\

 1 & \makecell[l]{\textbf{4. } Calculez l’énergie de liaison par nucléon de ce noyau.}\\

 1 & \makecell[l]{\textbf{5. } Ecrivez l’équation de la réaction de fission étudiée.}\\

 1 & \makecell[l]{\textbf{6. } Exprimez l'énergie libérée par la fission d'un noyau $^{235}_{92}U$ en fonction des énergies de liaison par
\\ nucléon du noyau père et des noyaux fils et calculez la valeur de cette énergie en MeV.}

 \end{tabular}

\textbf{7.  Dans le cœur de la centrale, de nombreuses autres réactions de fission du noyau $^{235}_{92}U$ se produisent. La perte de masse est, en moyenne, de 0,200 u par noyau.}


\begin{tabular}{c|l}

	1,5 & \makecell[l]{\textbf{7.1. } Calculez, en MeV, l'énergie moyenne libérée par la fission d’un noyau. Ce résultat est-il \\en concordance avec celui de la question 6 ?}\\

	1,5 &\makecell[l]{\textbf{7.2. }Calculez, en joule, l'énergie moyenne libérée par une mole de noyaux $^{235}_{92}U$ } 

\end{tabular}

\textbf{Données :}
\begin{itemize}
	\item Célérité de la lumière dans le vide : $c = 2,998 . 10^8 m.s^{-1}$  
	\item Masse du noyau d’uranium 235 : $m( ^{235}_{92}U) = 235,0134u$ 

	\item Energies de liaison par nucléon : $E_l/A(^{144}_{57}La) = 8,28MeV/nucl$éon  ; $E_l/A(^{88}_{35}Br)$=$8,56MeV/nucl$éon
	\item Constante d'Avogadro : $N_A = 6,02.10^{23} mol^{-1}$
	\item $1u$ = $1,66055.10^{-27}Kg$ et $1eV = 1,602.10^{-19}J$
	\item Masse d’un proton : $m(^1_1p) = 1,0073u$ ; Masse d’un neutron $m(^1_0n) = 1,0087u$
\end{itemize}





\end{document}
