\documentclass[12pt]{article}
\usepackage[a4paper, margin=.30in]{geometry}
%\usepackage{array}
\usepackage{fancybox}

\usepackage{graphicx, subfig, wrapfig, makecell,multirow, mhchem }
\newcommand\headerMe[2]{\noindent{}#1\hfill#2}
\renewcommand \thesection{\Roman{section}}

\newcolumntype{M}[1]{>{\raggedright}m{#1}}




\begin{document}

\headerMe{Royaume du Maroc}{année scolaire \emph{2024-2025}}\\
\headerMe{Ministère de l'Éducation nationale, }{  Professeur :\emph{Zakaria Haouzan}}\\
\headerMe{du Préscolaire et des Sports}{Établissement : \emph{Lycée SKHOR qualifiant}}\\

\begin{center}
Devoir surveillé N°2 \\
Durée 2h00\\
\underline{2-BAC Section des sciences expérimentales: Option de sciences physiques}\\
  \emph{semestre 1}

    \vspace{.2cm}
\hrulefill
\Large{Fiche Pédagogique}
\hrulefill\\
\end{center}
%end Headerss------------------------


%__________________Chimie ______________________-
%%%%%%%+_+_+_+_+_+_+_+_+_Partie1
\section[A]{Introduction }
\hspace{0.5cm}Le programme d'études de la matière physique chimie vise à croître un ensemble de compétences visant à développer la personnalité de l'apprenant. Ces compétences peuvent être classées en Compétences transversales communes et Compétences qualitatives associées aux différentes parties du programme.
\section{cadre de référence }
 \hspace{0.5cm}L'épreuve a été réalisée en adoptant des modes proches à des situations d'apprentissages et des situations problèmes, qui permettent de compléter les connaissances et les compétences contenues dans les instructions pédagogiques et dans le programme de la matière physique chimie et aussi dans le cadre de référence de l'examen national. 
 \\Tout en respectant les rapports d'importance précisés dans les tableaux suivants :
 \begin{center}
\begin{tabular}{|c||c||c|}
\hline
    \textbf{Restitution des Connaissances} & \textbf{Application des Connaissances} & \textbf{Situation Problème }\\
    \hline 
    $50\%$ & $25\%$ & $25\%$\\
    \hline
\end{tabular} 
\end{center}

\section{tableau de spécification}
 \begin{center}
\begin{tabular}{||c|c||c|c|c|c|}
\hline
     \multicolumn{2}{||c||}{\bf{   \hfill  Niveau d'habileté  \hfill } }
	& \makecell{Restitution \\des Connaissances} &\makecell{Application \\des Connaissances} & \makecell{Situation Problème} & la somme \\\hline

	%  
	\multirow{1}{*}{\makecell{Transformations \\nucléaires 62\%}}
	& \makecell{Décroissance\\radioactive}  & \makecell{9\%\\5Q - 4,5pts}  &\makecell{4\%\\1Q - 0,75pt} & \makecell{4\%\\2Q - 2,5pt} & 

	\multirow{3}{*}{\makecell{62\%\\13pts\\14Q\\75min}}\\\cline{2-5}
& \makecell{Noyaux,\\masse\\et énergie}  & \makecell{23\%\\2Q - 1,5pts}  &\makecell{11\%\\4Q - 3,5pt } & \makecell{11\%\\-} &\\\hline

%& \makecell{Les Ondes \\lumineuse}  & \makecell{10\%\\3Q - 3pts}  &\makecell{5\%\\1Q - 1pt } & \makecell{5\%\\2Q - 1pt} &\\\hline

\multirow{2}{*}{ \makecell{Les\\Transformations\\non totales d'un\\d'un\\système\\chimique 47\%}}

& \makecell{Transfo\\chimiques\\dans les\\deux sens}  & \makecell{\\20\%\\6Q - 4pts}  &\makecell{9\% \\2pts - 2Q } & \makecell{9\%\\2pts - 2Q} &

\multirow{1}{*}{\makecell{38\%\\7pts\\10Q\\45min }}\\

& \makecell{État\\d’équilibre\\d’un\\système\\chimique}  & \makecell{}  &\makecell{} & \makecell{} &\\\hline



     \multicolumn{2}{||c||}{\bf{   \hfill  --  \hfill } }
& \makecell{50\%\\14Q - 11pts}  &\makecell{25\%\\6Q - 5pts } & \makecell{25\%\\6Q - 5pts} &\\\hline



\end{tabular} 
\end{center}

\newpage
\begin{center}
    \shadowbox{\bf{ Devoir surveillé $N^{\circ}$2 Semestre I} }
\end{center}
 \begin{center}

     \begin{tabular}{|c||c||c|}
    \hline
         \multicolumn{3}{||c||}{\bf{   \hfill  Chimie  \hfill (7pts)} }\\
         \hline
         \multicolumn{3}{||c||}{\bf{Partie1: Transformations non totales d’un système chimique\dotfill} }\\
\hline
    \textbf{$N^{\circ}$Question } & \textbf{Réponse } & \textbf{Note }\\
    \hline
    $1.$ &
         \makecell{ Définition de Bronsted-Lowry }
    & $0,25pts$\\\hline
 %Q2
     $2.$ &


         \makecell{$\ce{HCOOH_{(aq)} + H_2O_{(aq)} <=> HCOO^-_{(aq)} + H_3O^+_{(aq)} }$}


       & $0,25pts$\\\hline  
 %Q3
     $3$ &
       \makecell{ ${x_f} = [H_3O^+].V_1 = 10^{-PH_1}.V_1 = 2. 10^{-3}.mol$}
    & $0,75pts$\\\hline  

 %Q4
     $4$ &
         \makecell{Le taux d’avancement final $\tau_1 = \frac{10^{-PH_1}}{C_1} = 0,04 = 4\%$  }
    & $0.75pt$\\\hline  
 %Q5
     $5$ &
       \makecell{la constante d’équilibre  $K_1 = \frac{10^{-PH_2}}{C_1 - 10^{-PH_1}} = 1,65.10^{-4}$ }
    & $0,5pt$\\\hline  

         \multicolumn{3}{||c||}{\bf{Partie1: Transformations non totales d’un système chimique\dotfill} }\\
\hline
    $1.$ &
         \makecell{  la conductivité : $\sigma = \lambda_1.[H_3O^+] + \lambda_2.[CH_3COO^-] = [H_3O^+](\lambda_1 + \lambda_2)$ }
    & $0,5pts$\\\hline
 %Q2
     $2.$ &


         \makecell{Montrer que : $[H_3O^+] = \frac{\sigma}{\lambda_1 + \lambda_2}$}


       & $0,5pts$\\\hline  
 %Q3
     $3$ &
       \makecell{La solution devient un peu moins acide  $PH_2 = -log([H_3O^+]) = 3,04$. }
    & $0,75pts$\\\hline  

 %Q4
     $4$ &
         \makecell{la valeur du taux d’avancement final : $\tau_2 = \frac{[H_3O^+]}{C_2} = 17\%$}
    & $0.25pt$\\\hline  
 %Q5
     $5$ &
       \makecell{la constante d’équilibre  $K_2 = K_1 = \frac{10^{-PH_2}}{C_1 - 10^{-PH_1}} = 1,65.10^{-4}$ }
    & $1pt$\\\hline  
 %Q6
     $6$ &
       \makecell{ l’effet de la dilution sur le taux
d’avancement final  $\tau_2 > \tau_1$ mais $K_1 = K_2$}

    & $1,5pt$\\\hline  



%%Partie 2 : 
         %\multicolumn{3}{||c||}{\bf{Partie 2 : Suivi d’une transformation chimique\dotfill (2pts)}}\\
%\hline
%$1.$ &
         %\makecell{\\ % table dont forget 
             %\begin{tabular}{|c|c|c|c|c|c|}
    %\hline
    %\multicolumn{2}{|c|}{Equation de la réaction}& \multicolumn{4}{c|}{2CuO + C $\rightarrow$ 2Cu + $CO_2$}\\\hline
    %états  & avancement& \multicolumn{4}{|c|}{quantité de Matière en mol}\\\hline
    %Etat initial          &    0        &  12.38 &  1.4&  0              &  0 \\\hline
                 %\makecell{Etat de \\transformation}&    $x$      & $ 12.38 - 2x$ & $ 1.4 - x$ & $ 2x$  & $x$ \\\hline
    %Etat final            &    $x_{max}$& $ 12.38 - 2x_{max}$ & $1.4 - x_{max}$ & $2x_{max}$  & $x_{max}$ \\\hline
   %% \cline{2-4}\
%\end{tabular}
         %\\$\; $ }  
    %& $1pt$\\\hline
 %%Q2
   %$2$ &
         %\makecell{ l’avancement maximal $x_{max} = 1.4mol$ et le réactif limitant le carbone C(s) 
 %}
    %& $0.5pt$\\\hline  
 %%Q3
   %$3$ &
         %\makecell{ bilan de matière dans l’état final :\\ $n_f(CuO) = 9.58 mol$ et $n_f(C) = 0 mol$ et $n_f(Cu) = 2.8 mol$, $n_f(CO_2) = 1.4mol$ 
 %}
    %& $0.5pt$\\\hline  
%Physique : 
    %Partie 1 : 
\end{tabular} 
\end{center}

\begin{center}
  \begin{tabular}{|c||c||c|}
    \hline
         \multicolumn{3}{||c||}{\bf{   \hfill  Physique  \hfill (13pts)} }\\
         \hline
         \multicolumn{3}{||c||}{\bf{Partie 1 :L’étude d’un nucléide d’azote 13 \dotfill (6pts)} }\\
\hline
    \textbf{$N^{\circ}$Question } & \textbf{Réponse } & \textbf{Note }\\
    \hline
    $1$ &
         \makecell{
           $ce{^{13}_7N -> ^{13}_6C + ^0_{+1}e}$
}
    & $1pt$\\\hline
 %Q2
 $2$ &
         \makecell{ la composition du noyau d’azote 13: A=13, N=6, Z=7 }
    & $0,75pt$\\\hline
 %Q3
 $3$ &
         \makecell{
           l’énergie de liaison : $E_l({13}^N) = [Z.m_p + Nm_n - m({13}^N)].c^2$\\ 
           A.N $E_l({13}^N) = 90,523Mev$
      }
    & $0,5pt$\\\hline

 %Q4
 $4$ &
         \makecell{
           l’énergie de liaison par nucléon: $\xi({^13}N) = \frac{E_l}{A} = 6,96Mev $
      }
    & $0,5pt$\\\hline
 %Q5
 $5$ &
         \makecell{
           le noyau le plus stable c'est: carbone 13
      }
    & $0,25pt$\\\hline
 %Q6
 $6.a$ &
         \makecell{
           $\ce{^{16}_8O + ^1_1p -> ^{13}_7N + ^4_2He}$
      }
    & $1pt$\\\hline
 %Q6.b
 $6.b$ &
         \makecell{
           l’énergie produite par cette réaction nucléaire. : \\
           $\Delta{E} = (m(^4He) + m(^{13}N) - (m({^16}O) + m(^1p)) ).c^2$
      }
    & $2pt$\\\hline
      %Partie 2 : -----
\multicolumn{3}{||c||}{\bf{Partie 2 :  datation par le carbone 14 \dotfill (3pts)} }\\
\hline
%1
 $1$ &
 \makecell{  la signification physique du temps de demi-vie}
    & $0,75pt$\\\hline
%2
 $2$ &
      \makecell{l’activité radioactive à l’origine de ce morceau de bois.: \\ 
      $a_0 =  \lambda.N_0 = \lambda.\frac{m_0}{M}.N_A = \frac{ln(2)}{t_{1/2}}.\frac{m_0.N_A}{M} = 1,49.Bq $
      }
    & $0,75pt$\\\hline
%2
 $3$ &
      \makecell{ montrer que : $t = \frac{t{1/2}}{ln(2)}.ln(\frac{m_0}{m}) = 12,36.10^{3}ans$ }
    & $1,5pt$\\\hline

\multicolumn{3}{||c||}{\bf{Partie 2 : Etude d’un stimulateur cardiaque.\dotfill (4pts)} }\\
\hline
%1
 $1$ &
 \makecell{le noyau le plus stable $^{238}Pu$ }
    & $1pt$\\\hline
%2
 $2.1$ &
 \makecell{$\ce{^{238}_{94}Pu -> ^{234}_{92}U +  ^4_2He}$}
    & $1pt$\\\hline
 $2.2$ &
 \makecell{l’énergie libérée $E_{lib} = -5,6Mev$ }
    & $1pt$\\\hline

 $3$ &
 \makecell{
Le patient aura environ 85 ans lorsqu'il faudra changer son \\stimulateur cardiaque
car :\\
$t = \frac{t{1/2}}{ln(2)}.ln(\frac{a_0}{a}) = 45,12ans$
\\
L'âge du patient lors du changement sera donc :
Âge = 40 + 45,2 = 85,2 ans
}
 
    & $1pt$\\\hline

  \end{tabular}
  \end{center}


\end{document}
