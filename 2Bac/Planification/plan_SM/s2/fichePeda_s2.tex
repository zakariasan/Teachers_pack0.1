\documentclass[12pt]{article}
\usepackage[a4paper, margin=.20in]{geometry}
%\usepackage{array}
\usepackage{graphicx, subfig, wrapfig, makecell,fancyhdr,xcolor }
\newcommand\headerMe[2]{\noindent{}#1\hfill#2}
\renewcommand \thesection{\Roman{section}}

\newcolumntype{M}[1]{>{\raggedright}m{#1}}

\chead{\includegraphics[width = 0.1\textwidth]{./img/logoMin.png}}
\cfoot{helllo}

\begin{document}

\begin{center}
\includegraphics[width = 0.18\textwidth]{./img/logoMin.png}
\vspace{-3cm}
\end{center}
\headerMe{Matière : Physique-Chimie}{Établissement : \emph{Lycée SKHOR qualifiant}}\\
\headerMe{ Niveau : 2BAC }{  Professeur :\emph{Zakaria Haouzan}}\\
%\headerMe{Niveau : 2BAC-SM-X}{Heure : 5H}\\

\begin{center}
	\vspace{0.5cm}
%\underline{Section des sciences expérimentales: Option de sciences physiques}\\
\underline{Section des sciences mathématiques: Option sciences mathématiques (A) et (B)}
\\
    %\vspace{.2cm}
\hrulefill
\Large{Planification annuelle
du programme de la matière chimie physique}
\hrulefill\\
\Large{Semestre 02}
\hrulefill\\

\end{center}
%end Headerss------------------------


%__________________Chimie ______________________-
%%%%%%%+_+_+_+_+_+_+_+_+_Partie1
\begin{center}
	 \begin{tabular}{||p{0.15\textwidth}||p{0.5\textwidth}||p{0.1\textwidth}||p{0.1\textwidth}|}

     \hline
 %\multicolumn{4}{|c|}{Déroulement}\\\hline

     %Header ====================================================
\makecell{La période de\\réalisation} & \makecell{Le contenude programme } & \multicolumn{2}{|c|}{L’enveloppe horaire }  \\\hline
  %Sem01===========================================================
\makecell{
\color{red}{Semaine 1}\\De 03-02-2025\\à 08-02-2025
}   &

     \makecell{\bf{ Applications}: Production d’ondes
     \\ \bf{}électromagnétiques et communication.} 
    &8H
     & \\\cline{2-3} & 
\makecell{
  Exercices: Applications: Production d’ondes
\\électromagnétiques et communication.     } & 2H  & 
% SEM01 =======================================================
\\\hline
\hline
\makecell{
\color{red}{Semaine 2}\\De 17-02-2025\\à 22-02-2025
}&
\makecell{
  \bf{Evolution} spontanée d'un système chimique. 
}& 2H& \\\cline{2-3}&

\makecell{
  \bf{} Transformations spontanées dans les piles et \\ \bf{}récupération de
l’énergie
} & 6H  & \\\hline \hline

%sem 3=======================================================
\makecell{
\color{red}{Semaine 3}\\De 24-02-2024\\à 1-03-2025
}&
\makecell{
  \bf{} Lois de Newton.
}& 4H& \\\cline{2-3}&
\makecell{Exercices:  Lois de Newton.}& 2H& \\\cline{2-3}&
\makecell{\bf{Devoir} $N^{\circ}1$ \emph{Semestre $N^{\circ}2$}} &2H&\\\hline

%SEM 4:====================================================== 

\makecell{
\color{red}{Semaine 4}\\De 03-03-2025\\à 08-03-2025
} &

\makecell{
  \bf{}Exemples de transformations forces} & 6H & \\\cline{2-3} &
\makecell{
  Exercices:lois de Newton 
  \\Exercices: Exemples de transformations forces}& 2H &\\\hline

% sem 5:===============================================

\makecell{
\color{red}{Semaine 5}\\De 10-03-2025\\à 15-03-2025}&
\makecell{ 
  \bf{Applications}: Chute verticale d'un solide
}& 4H &\\\cline{2-3} &
\makecell{
  Exercices :  Chute verticale d'un solide} & 2H &
\\\hline\hline

% sem 6:=========================================

\makecell{
\color{red}{Semaine 6}\\De 17-03-2025\\à 22-03-2025}&
\makecell{Vacances }& 8 jours&\\\hline



% sem 7:=================================================== 

\makecell{
  \color{red}{Semaine 7}\\De 24-03-2025\\à 29-03-2025
}&
\makecell{ 
  \bf{Applications}: Mouvements plans
  }& 6H &\\\cline{2-3} &
\makecell{
  Exercices : Mouvements plans } & 2H &
\\\hline\hline
% sem 8:=========================================

\makecell{
\color{red}{Semaine 8}\\De 31-03-2025\\à 05-04-2025}&
\makecell{Vacances }& 4 jours&\\\hline



% sem 9:=================================================== 

\makecell{
  \color{red}{Semaine 9}\\De 07-04-2025\\à 12-04-2025
}&
\makecell{ 
  \bf{Applications}: Satellites artificiels et planètes
  }& 2H &\\\cline{2-3} &
\makecell{
  Exercices : Satellites artificiels et planètes} & 2H &\\\cline{2-3} &

\makecell{ \bf{} Les réactions d’estérification et d’hydrolyse.
} & 4H &\\\hline \hline

\end{tabular}\end{center}


% =====================================================
\begin{center}
	 \begin{tabular}{||p{0.15\textwidth}||p{0.5\textwidth}||p{0.1\textwidth}||p{0.1\textwidth}|}
\hline
 %\multicolumn{4}{|c|}{Déroulement}\\\hline

\makecell{La période de\\réalisation} & \makecell{Le contenu de programme } & \multicolumn{2}{|c|}{L’enveloppe horaire }  \\\hline

% sem 10:=================================================== 

\makecell{
  \color{red}{Semaine 10}\\De 14-04-2025\\à 19-04-2025
}&
\makecell{ \bf{} Les réactions d’estérification et d’hydrolyse.
} & 2H & \\\cline{2-3} &
\makecell{ 
Exercices :  Les réactions d’estérification et d’hydrolyse.
  }& 2H &\\\cline{2-3} &
\makecell{
  \bf{}Relation quantitatif entre la somme des moments \\et \bf{}l'accélération angulaire} & 4H &\\\hline \hline



% sem 11:============================================================ 

\makecell{
\color{red}{Semaine 11}\\De 21-04-2025\\à 26-04-2025} &
\makecell{
  \bf{} Systèmes oscillants.
}& 6H &\\\cline{2-3} &

\makecell{\bf{Devoir} $N^{\circ}2$ \emph{Semestre $N^{\circ}2$}} &2H&\\\hline

% sem 12=========================================================== 

\makecell{
\color{red}{Semaine 12}\\De 28-04-2025\\à 03-05-2025
}&
\makecell{
  \bf{}Systèmes oscillants. 
} & 3H &\\\cline{2-3} &
\makecell{
  Exercices: Systèmes oscillants.
} & 5H&\\\hline\hline

% sem 13=================================================
\makecell{
\color{red}{Semaine 13}\\De 05-04-2025\\à 10-05-2025
} &
\makecell{
  \bf{}Contrôle de l’évolution de systèmes chimiques.
}
& 6H & \\\cline{2-3} &
	\makecell{
    Exercices: Contrôle de l’évolution de systèmes chimiques.
  }& 2H & \\\hline\hline

% sem 14 =============================================================
\makecell{
\color{red}{Semaine 14}\\De 12-05-2025\\à 17-05-2025
}	&

\makecell{
  \bf{} Aspects énergétiques.
} &  & \\\cline{2-3} &
\makecell{
  Atome et mécanique de Newton:
} & & \\\cline{2-3}
&
\makecell{
\bf{Devoir} $N^{\circ}2$ \emph{Semestre $N^{\circ}1$}
}
& 2H &\\\hline\hline

\end{tabular}
% sem 13 ============================================================== 
%\makecell{
%\color{red}{Semaine 13}\\De 01-12-2024\\à 07-12-2024
%}	&
%\makecell{
%%  \bf{Dipôle RC}
%} & 5H&\\\cline{2-3} &
%\makecell{
 %\ Exercices: Dipôle RC



% sem 14 =============================================================== 


% sem 15  ============================================================== 

%\end{tabular}\end{center}

%%__________________Chimie ______________________-
%%%%%%%%+_+_+_+_+_+_+_+_+_Partie1
%\begin{center}
	 %\begin{tabular}{||p{0.15\textwidth}||p{0.5\textwidth}||p{0.1\textwidth}||p{0.1\textwidth}|}
%\hline
 %%\multicolumn{4}{|c|}{Déroulement}\\\hline


%% sem 19

%\makecell{
%\color{red}{Semaine 20}\\De 16-01-2023\\à 22-01-2023}


\end{center}


\end{document}
