\documentclass[12pt]{article}
\usepackage[a4paper, margin=.20in]{geometry}
%\usepackage{array}
\usepackage{graphicx, subfig, wrapfig, makecell,fancyhdr,xcolor }
\newcommand\headerMe[2]{\noindent{}#1\hfill#2}
\renewcommand \thesection{\Roman{section}}

\newcolumntype{M}[1]{>{\raggedright}m{#1}}

\chead{\includegraphics[width = 0.1\textwidth]{./img/logoMin.png}}
\cfoot{helllo}

\begin{document}

\begin{center}
\includegraphics[width = 0.18\textwidth]{./img/logoMin.png}
\vspace{-3cm}
\end{center}
\headerMe{Matière : Physique-Chimie}{Établissement : \emph{Lycée SKHOR qualifiant}}\\
\headerMe{ Niveau : 2BAC-X }{  Professeur :\emph{Zakaria Haouzan}}\\
%\headerMe{Niveau : 2BAC-SM-X}{Heure : 5H}\\

\begin{center}
	\vspace{0.5cm}
%\underline{Section des sciences expérimentales: Option de sciences physiques}\\
\underline{Section des sciences mathématiques: Option Sciences physiques
}
\\
    %\vspace{.2cm}
\hrulefill
\Large{Planification annuelle
du programme de la matière chimie physique}
\hrulefill\\
\end{center}
%end Headerss------------------------


%__________________Chimie ______________________-
%%%%%%%+_+_+_+_+_+_+_+_+_Partie1
\begin{center}
	 \begin{tabular}{||p{0.15\textwidth}||p{0.5\textwidth}||p{0.1\textwidth}||p{0.1\textwidth}|}

     \hline
 %\multicolumn{4}{|c|}{Déroulement}\\\hline

     %Header ====================================================
\makecell{La période de\\réalisation} & \makecell{Le contenude programme } & \multicolumn{2}{|c|}{L’enveloppe horaire }  \\\hline
  %Sem01===========================================================
\makecell{
\color{red}{Semaine 1}\\De 09-09-2024\\à 14-09-2024
\\\color{red}{Semaine 2}\\De 16-09-2024\\à 21-09-2024
}   &

\makecell{Pendant cette période, nous réalisons\\-Contrat didactique
\\- Révision générale
\\- Examens diagnostiques
\\- Soutien pédagogique
}
    &
18H
    &
% SEM03 =======================================================
\\\hline
\hline
\makecell{
\color{red}{Semaine 3}\\De 23-09-2024\\à 28-09-2024
}&
\makecell{
\bf {Les} questions qui se posent au physicien
\\ \bf{Les} questions qui se posent au chimiste
\\ \bf{Les} transformations lentes et les transformations\\\bf{rapides}
}& 3H &\\\hline\hline

%sem 4=======================================================
\makecell{
\color{red}{Semaine 4}\\De 30-09-2024\\à 05-10-2024
}&
\makecell{
  \bf{Le} suivi temporel d'une transformation\\chimique - La vitesse de réaction
}& 7H& \\\cline{2-3}&
\makecell{
  Exercices: suivi temporel d'une transformation\\chimique - La vitesse de réaction
} & 2H  & +3H\\\hline \hline

%SEM 5:====================================================== 

\makecell{
\color{red}{Semaine 5}\\De 06-10-2024\\à 12-10-2024
} &

\makecell{
  \bf{Les } Ondes mécaniques progressives. 
} & 4H & \\\cline{2-3} &
\makecell{
  Exercices:Ondes mécaniques progressives. 
}& 1H &+2h\\\cline{2-3} &

\makecell{\bf{Les} Ondes mécaniques progressives périodiques. } 
  &3H&
	\\\cline{2-3} & 
Exercices : Ondes mécaniques progressives périodiques. & 1H &\\\hline\hline

% sem 6%sem 6:===============================================

\makecell{
\color{red}{Semaine 6}\\De 13-10-2024\\à 19-10-2024}&
\makecell{
  \bf{Les} Ondes mécaniques progressives périodiques.
} & 1H & \\\cline{2-3} &

Exercices : Ondes mécaniques progressives périodiques. & 1H &\\\cline{2-3} &
\makecell{
  \bf{La} Propagation d'une onde lumineuse
} & 4H   &

\\\hline\hline

% sem 7:=========================================

\makecell{
\color{red}{Semaine 7}\\De 20-10-2024\\à 27-10-2024}&
\makecell{Vacances d'automne}& 8 jours&\\\hline



% sem 8:=================================================== 

\makecell{
\color{red}{Semaine 8}\\De 27-10-2024\\à 02-11-2024
}&
\makecell{Exercices: La Propagation d'une onde lumineuse
} & 1H &\\\cline{2-3}
&\makecell{Révision} & 3H &\\\cline{2-3}
&\makecell{\bf{Devoir} $N^{\circ}1$ \emph{Semestre $N^{\circ}1$}} &2H&\\\hline

\end{tabular}\end{center}


% =====================================================#############
\begin{center}
	 \begin{tabular}{||p{0.15\textwidth}||p{0.5\textwidth}||p{0.1\textwidth}||p{0.1\textwidth}|}
\hline
 %\multicolumn{4}{|c|}{Déroulement}\\\hline

\makecell{La période de\\réalisation} & \makecell{Le contenu de programme } & \multicolumn{2}{|c|}{L’enveloppe horaire }  \\\hline

% sem 9:============================================================ 

\makecell{
\color{red}{Semaine 9}\\De 03-11-2024\\à 09-11-2024} &
\makecell{
  \bf{Transformations} chimiques qui ont lieu dans\\les deux sens.
}& 3H &\\\cline{2-3} &

\makecell{
  Exercices:Transformations chimiques qui ont lieu dans\\les deux sens.
} & 1H&\\\cline{2-3} 

&\makecell{
  \bf{L'état} d'équilibre d'un système chimique
} & 2H &\\\cline{2-3}
\hline\hline

% sem 10=========================================================== 

\makecell{
\color{red}{Semaine 10}\\De 10-11-2024\\à 16-11-2024
}&
\makecell{
  \bf{L'état} d'équilibre d'un système chimique
} & 2H &\\\cline{2-3} &
\makecell{
  Exercices: L'état d'équilibre d'un système chimique
} & 1H&\\\cline{2-3} &
\makecell{
  \bf{Décroissance} radioactive\\
} & 3H &
\\\cline{2-3}
\hline\hline

% sem 11=================================================
\makecell{
\color{red}{Semaine 11}\\De 17-10-2024\\à 23-11-2024
} &
\makecell{
  Exercices: Décroissance radioactive
}
& 1H & \\\cline{2-3} &
	\makecell{
    \bf{Noyaux}, masse et énergie
  }& 5H & \\\hline\hline

% sem 12 =============================================================
\makecell{
\color{red}{Semaine 12}\\De 24-11-2024\\à 30-11-2024
}	&

\makecell{
  \bf{Noyaux}, masse et énergie
} & 3H & \\\cline{2-3} &
\makecell{
  Exercices: Noyaux, masse et énergie
} & 2H & \\\cline{2-3}
& corriger le Devoir N 1 & 1H &\\\hline\hline

% sem 13 ============================================================== 
\makecell{
\color{red}{Semaine 13}\\De 01-12-2024\\à 07-12-2024
}	&
\makecell{
  \bf{Dipôle RC}
} & 5H&\\\cline{2-3} &
\makecell{
  Exercices: Dipôle RC
} & 1H & \\\hline\hline



% sem 14 =============================================================== 

\makecell{
\color{red}{Semaine 14}\\De 08-12-2024\\à 14-12-2024} 

&\makecell{
  Révision
} & 4H &\\\cline{2-3}
&\makecell{\bf{Devoir} $N^{\circ}2$ \emph{Semestre $N^{\circ}1$}} &2H
& \\\hline\hline

% sem 15  ============================================================== 

\makecell{
\color{red}{Semaine 15}\\De 15-12-2024\\à 21-12-2024}
&\makecell{\bf{Les} transformations chimiques liées à des\\réactions acido-basiques}&6H&\\\hline\hline

% sem 16 

\makecell{
\color{red}{Semaine 16}\\De 22-12-2024\\à 28-12-2024}
&\makecell{Exercices: Les transformations chimiques liées à des\\réactions}&2H&\\\cline{2-3}
&\makecell{\bf{Dipôle} RL}&4H& + 2h \\\cline{2-3}


&\makecell{ 
corriger le Devoir N 2 
}&1H&\\\hline\hline


% sem 17================================================================= 

\makecell{
\color{red}{Semaine 17}\\De 30-12-2024\\à 04-01-2025} 
&\makecell{Exercices: Dipôle RL }& 2H &\\\cline{2-3}

&\makecell{\bf{Les} oscillations libres d'un circuit RLC}&4H
&\\\hline\hline


% sem 18================================================================= 

\makecell{
\color{red}{Semaine 18}\\De 06-01-2025\\à 11-01-2025}
&\makecell{\bf{Les} oscillations libres d'un circuit RLC}& 2H &\\\cline{2-3}
&\makecell{Exercices: Les oscillations libres d'un circuit RLC}&2H&\\\hline
\hline

% sem 19

\makecell{
\color{red}{Semaine 19}\\De 12-01-2025\\à 18-01-2025}
&\makecell{\bf{Devoir} $N^{\circ}3$ \emph{Semestre $N^{\circ}1$}} &2H&\\\cline{2-3}
&\makecell{\bf{Le circuit} RLC série en régime sinusoïdal forcé} &6H&\\\cline{2-3}
&\makecell{ corriger le Devoir N 3  }&1H&\\\hline

%\end{tabular}\end{center}

%%__________________Chimie ______________________-
%%%%%%%%+_+_+_+_+_+_+_+_+_Partie1
%\begin{center}
	 %\begin{tabular}{||p{0.15\textwidth}||p{0.5\textwidth}||p{0.1\textwidth}||p{0.1\textwidth}|}
%\hline
 %%\multicolumn{4}{|c|}{Déroulement}\\\hline


%% sem 19

%\makecell{
%\color{red}{Semaine 20}\\De 16-01-2023\\à 22-01-2023}
%&\makecell{\bf{Le circuit} RLC série en régime sinusoïdal forcé} &6H&\\\cline{2-3}


\end{tabular}
\end{center}


\end{document}
