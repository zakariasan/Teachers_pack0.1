\documentclass[12pt]{article}
\usepackage[a4paper, margin=.20in]{geometry}
%\usepackage{array}
\usepackage{graphicx, subfig, wrapfig, makecell,fancyhdr,xcolor }
\newcommand\headerMe[2]{\noindent{}#1\hfill#2}
\renewcommand \thesection{\Roman{section}}

\newcolumntype{M}[1]{>{\raggedright}m{#1}}

\chead{\includegraphics[width = 0.1\textwidth]{./img/logoMin.png}}
\cfoot{helllo}

\begin{document}

\begin{center}
\includegraphics[width = 0.18\textwidth]{./img/logoMin.png}
\vspace{-3cm}
\end{center}
\headerMe{Matière : Physique-Chimie}{Établissement : \emph{Lycée SKHOR qualifiant}}\\
\headerMe{ Niveau : 2BAC-SM-X }{  Professeur :\emph{Zakaria Haouzan}}\\
%\headerMe{Niveau : 2BAC-SM-X}{Heure : 5H}\\

\begin{center}
	\vspace{0.5cm}
\underline{Section des sciences expérimentales: Option de sciences physiques}\\
\underline{Section des sciences mathématiques: Option sciences mathématiques (A) et (B)}
\\
    %\vspace{.2cm}
\hrulefill
\Large{Planification annuelle
du programme de la matière chimie physique}
\hrulefill\\
\end{center}
%end Headerss------------------------


%__________________Chimie ______________________-
%%%%%%%+_+_+_+_+_+_+_+_+_Partie1
\begin{center}
	 \begin{tabular}{||p{0.15\textwidth}||p{0.5\textwidth}||p{0.1\textwidth}||p{0.1\textwidth}|}
\hline
 %\multicolumn{4}{|c|}{Déroulement}\\\hline

\makecell{La période de\\réalisation} & \makecell{Le contenu
de programme } & \multicolumn{2}{|c|}{L’enveloppe horaire }  \\\hline


\makecell{
\color{red}{Semaine 1}\\De 05-09-2022\\à 11-09-2022
\\\color{red}{Semaine 2}\\De 12-09-2022\\à 18-09-2022
%\\\color{red}{Semaine 1}\\De 05-09-2022\\à 12-09-2022
}	  &
\makecell{Pendant cette période, nous réalisons\\-le contrat didactique
\\- Révision générale
\\- Examens diagnostiques
\\- Soutien pédagogique
\\-Les questions qui se posent au physicien
\\-Les questions qui se posent au chimiste
}
							  &
18H
							  &
							  % SEM3
				\\\hline
				\hline
\makecell{
\color{red}{Semaine 3}\\De 19-09-2022\\à 25-09-2022
}
							  &

							  \makecell{\bf{Les} ondes mécaniques progressives}
							  & 4H &\\\cline{2-3}
							  &
	Exercices Les ondes mécaniques progressives & 1H &\\\cline{2-3}
													   &
	\bf{Les} ondes mécaniques progressives périodiques& 1H &\\\hline
\hline
%sem 4 

\makecell{
\color{red}{Semaine 4}\\De 26-09-2022\\à 02-10-2022
}&
\bf{Les} ondes mécaniques progressives périodiques
														&3H&\\\cline{2-3}
														
														&
\makecell{Exercices: Les ondes mécaniques\\progressives périodiques}
														&1H&\\\cline{2-3}
														& 
	\makecell{\bf{La } propagation des ondes lumineuses } &2H& \\\hline
\hline
%sem 5: 

\makecell{
\color{red}{Semaine 5}\\De 03-10-2022\\à 09-10-2022
}&
	\makecell{\bf{La } propagation des ondes lumineuses } &2H& \\\cline{2-3}
&
	\makecell{Exercices: La  propagation des ondes lumineuses}&1H&\\\cline{2-3}
															  &
	\makecell{\bf{Les} transformations lentes et les transformations\\\bf{rapides}} &2H&
																					\\\cline{2-3} & \makecell{\bf{Le} suivi temporel d'une transformation\\chimique -
																					La vitesse de réaction }&1H&\\\hline\hline

% sem 6%sem 6: 

\makecell{
\color{red}{Semaine 6}\\De 10-10-2022\\à 16-10-2022}&
\makecell{\bf{Le} suivi temporel d'une transformation\\chimique -La vitesse de réaction }&6H&\\\hline\hline

% sem 7%sem 7: 

\makecell{
\color{red}{Semaine 7}\\De 17-10-2022\\à 23-10-2022}&
\makecell{Exercices: suivi temporel d'une transformation\\chimique -La vitesse de réaction }&2H&\\\cline{2-3}
																							&\makecell{Révision} &2H&\\\cline{2-3}
																							&\makecell{\bf{Devoir} $N^{\circ}1$ \emph{Semestre $N^{\circ}1$}} &2H&\\\hline

% sem 7%sem 8: 

\makecell{
\color{red}{Semaine 8}\\De 24-10-2022\\à 30-10-2022}&
\makecell{Vacances d'automne}& 8 jours&\\\hline


\end{tabular}\end{center}
% Dev 2

%__________________Chimie ______________________-
%%%%%%%+_+_+_+_+_+_+_+_+_Partie1
\begin{center}
	 \begin{tabular}{||p{0.15\textwidth}||p{0.5\textwidth}||p{0.1\textwidth}||p{0.1\textwidth}|}
\hline
 %\multicolumn{4}{|c|}{Déroulement}\\\hline

\makecell{La période de\\réalisation} & \makecell{Le contenu
de programme } & \multicolumn{2}{|c|}{L’enveloppe horaire }  \\\hline


\makecell{
\color{red}{Semaine 9}\\De 31-10-2022\\à 06-11-2022
}	  &
\makecell{\bf{Décroissance} radioactive\\
	Exercices: Décroissance radioactive
}
							  &
4H
							  &
							  % SEM3
				\\\cline{2-3}
							  &
	\makecell{\bf{Noyaux}, masse et énergie}&2H& \\\hline\hline
% sem 10 
\makecell{
\color{red}{Semaine 10}\\De 07-11-2022\\à 13-11-2022
}	&
	\makecell{\bf{Noyaux}, masse et énergie}&6H& \\\hline
% sem 11 
\makecell{
\color{red}{Semaine 11}\\De 14-11-2022\\à 20-11-2022
}	&
	\makecell{Exercices: Noyaux, masse et énergie}
&2H& \\\cline{2-3}
&\makecell{\bf{Dipôle RC}} & 3H&\\\cline{2-3}
& corriger le Devoir N 1 & 1H&\\\hline\hline

% sem 12 

\makecell{
\color{red}{Semaine 12}\\De 21-11-2022\\à 27-11-2022} 
& \makecell{\bf{Dipôle RC} \\ Exercices: Dipôle RC} & 3H&\\\cline{2-3} 
& \makecell{\bf{Transformations} chimiques qui ont lieu dans\\les deux sens.}&3H&\\\hline
\hline

% sem 13 

\makecell{
\color{red}{Semaine 13}\\De 28-11-2022\\à 04-12-2022} 
& \makecell{ Exercices:Transformations chimiques qui ont lieu dans\\les deux sens.} & 1H&\\\cline{2-3} 
&\makecell{\bf{L'état} d'équilibre d'un système chimique}&4H&\\\cline{2-3}
& \makecell{Exercices:L'état d'équilibre d'un système chimique }&1H&\\\hline
\hline
% sem 14 

\makecell{
\color{red}{Semaine 14}\\De 05-12-2022\\à 11-12-2022} 
&\makecell{Révision} &2H&\\\cline{2-3}													
&\makecell{\bf{Devoir} $N^{\circ}2$ \emph{Semestre $N^{\circ}1$}} &2H&\\\cline{2-3}
&\makecell{\bf{Dipôle} RL}&2H& \\\hline\hline

% sem 15 

\makecell{
\color{red}{Semaine 15}\\De 12-12-2022\\à 18-12-2022} 
&\makecell{\bf{Dipôle} RL\\Exercices: Dipôle RL }&4H&\\\cline{2-3}
&\makecell{\bf{Les} oscillations libres d'un circuit RLC}&2H&\\\hline
\hline


% sem 16 

\makecell{
\color{red}{Semaine 16}\\De 19-12-2022\\à 25-12-2022}
&\makecell{\bf{Les} oscillations libres d'un circuit RLC}&4H&\\\cline{2-3}
&\makecell{Exercices: Les oscillations libres d'un circuit RLC}&1H&\\\cline{2-3}
&\makecell{ corriger le Devoir N 2  }&1H&\\\hline
\hline

% sem 16 

\makecell{
\color{red}{Semaine 17}\\De 26-12-2022\\à 01-01-2023}
&\makecell{Exercices: Les oscillations libres d'un circuit RLC}&1H&\\\cline{2-3}
&\makecell{\bf{Les} transformations chimiques liées à des\\réactions acido-basiques}&5H&\\\hline\hline

% sem 18 

\makecell{
\color{red}{Semaine 18}\\De 02-01-2023\\à 08-01-2023}
&\makecell{\bf{Les} transformations chimiques liées à des\\réactions
acido-basiques}&1H&\\\cline{2-3}
&\makecell{Exercices: Les transformations chimiques liées à des\\réactions}&2H&\\\cline{2-3}
&\makecell{Révision}&3H&\\\hline\hline


% sem 19

\makecell{
\color{red}{Semaine 19}\\De 09-01-2023\\à 15-01-2023}
&\makecell{\bf{Devoir} $N^{\circ}3$ \emph{Semestre $N^{\circ}1$}} &2H&\\\cline{2-3}
&\makecell{\bf{Le circuit} RLC série en régime sinusoïdal forcé} &1H&\\\cline{2-3}
&\makecell{ corriger le Devoir N 3  }&1H&\\\hline

%\end{tabular}\end{center}

%%__________________Chimie ______________________-
%%%%%%%%+_+_+_+_+_+_+_+_+_Partie1
%\begin{center}
	 %\begin{tabular}{||p{0.15\textwidth}||p{0.5\textwidth}||p{0.1\textwidth}||p{0.1\textwidth}|}
%\hline
 %%\multicolumn{4}{|c|}{Déroulement}\\\hline


%% sem 19

%\makecell{
%\color{red}{Semaine 20}\\De 16-01-2023\\à 22-01-2023}
%&\makecell{\bf{Le circuit} RLC série en régime sinusoïdal forcé} &6H&\\\cline{2-3}


\end{tabular}
\end{center}


\end{document}
