\documentclass[12pt]{article}
\usepackage[a4paper, margin=.30in]{geometry}
%\usepackage{array}
\usepackage{fancybox}

\usepackage{graphicx, subfig, wrapfig, makecell }
\usepackage{multirow}

\newcommand\headerMe[2]{\noindent{}#1\hfill#2}
\renewcommand \thesection{\Roman{section}}

\newcolumntype{M}[1]{>{\raggedright}m{#1}}




\begin{document}

\headerMe{Royaume du Maroc}{année scolaire \emph{2024-2025}}\\
\headerMe{Ministère de l'Éducation nationale, }{  Professeur :\emph{Zakaria Haouzan}}\\
\headerMe{du Préscolaire et des Sports}{Établissement : \emph{Lycée SKHOR qualifiant}}\\

\begin{center}
%Evaluation Diagnostique \\
%Durée 1h45
\hrulefill
\shadowbox{\bf{Rapport de l’évaluation diagnostique }}
\hrulefill\\
\end{center}
%end Headerss------------------------


%__________________Chimie ______________________-
%%%%%%%+_+_+_+_+_+_+_+_+_Partie1
\section[A]{Introduction }
\hspace{0.5cm} Pour constituer une vision globale sur l’état d’avancement des apprenants en Physique chimie, et pour établir le profil de l’ensemble de la classe, une évaluation diagnostique s’impose au début de chaque année scolaire.

Il s’agit de situer les apprenants par rapport aux apprentissages prévus dans le nouveau programme, de détecter leurs pré-requis et pré-acquis, de repérer leurs difficultés d’apprentissage et de déterminer leurs savoirs, savoir-faire et savoir-être relatifs aux différentes compétences requises pour entamer et s’entreprendre les nouveaux apprentissages.

L’évaluation diagnostique en matière Physique chimie pour le niveau 2Bac-PC , visera les disciplines suivantes : en Physique 3 partie Mécanique , électrodynamique , Optique et Chimie générale .
\section{ Objectifs de l’évaluation diagnostique :}
\begin{itemize}
	\item Etre capable de déterminer les points forts et les points faibles dans les apprentissages antérieurs des apprenants.
	\item Déterminer les difficultés et les obstacles d’apprentissage et motiver les apprenants à les surmonter

	\item Investir les résultats de l’évaluation diagnostique pour planifier les activités de soutien
	\item Adopter ces résultats pour l’orientation et le conseil 

\end{itemize}

\section{Informations générales sur l’évaluation diagnostique :  }
Le tableau ci-dessous et la fiche pédagogique résume la planification de l’évaluation diagnostique et le nombre des apprenants présents :
\begin{center}
  \begin{tabular}{|c|c|c|}
	  \hline
	  La classe & Date de la procédure & Nombre des apprenants présents\\\hline
	  2Bac-PC3 & 18/09/2024 & 7\\\hline
\end{tabular}
\end{center}

Cette évaluation inclus diverses questions : Questions vrai ou faux et Questions à choix multiples (QCM) dans des Situations problèmes.
\section{Analyse des résultats de l’évaluation diagnostique : }

\begin{center}
  \begin{tabular}{|c|c|c|c|}
	  \hline
La classe & \makecell{Les élèves de niveau faible \\ 0,20} & \makecell{Les élèves moyens \\ ]20,60]}& \makecell{Les élèves brillants \\  ]60,100]} \\\hline
2Bac PC3 & \makecell{2 élèves \\ 29\%} & \makecell{5 élèves \\ 71\%}& \makecell{0 élève \\ 0\%} \\\hline
\end{tabular}
\end{center}


Après que les apprenants aient passé ce test , on peut dire que les résultats obtenus sont en général positifs, mais certaines observations doivent être mentionnées, telles que :

\begin{itemize}
	\item La plupart des apprenants confondent entre les grandeurs physiques et leurs unités dans le système international (Puissance mécanique et électrique, le moment d'une force ...)
	\item Tous les apprenants sauf un ont des problèmes au niveau de la puissance instantanée dans le cas de la rotaion 

	\item  La plupart des apprenants ont oublié les concepts liés à la quantité de matière et la géométrie moléculaire . 
	\item Certains apprenants ont oublié : comment relier le voltmètre et l’ampèremètre pour mesurer la tension et l’intensité de courant,  la loi d’ohm, l’expression de la puissance électrique.
\end{itemize}

\section*{Conclusion : }
\hspace{2cm}
Cette évaluation diagnostique a mis en lumière plusieurs lacunes à la fois dans les notions de base et dans la capacité des élèves à s’exprimer clairement sur des sujets scientifiques. Bien que certains élèves montrent des aptitudes moyennes, ils nécessitent un soutien supplémentaire pour approfondir leur compréhension. D’autres, en revanche, ont besoin de reprendre les bases pour éviter d’être dépassés par le rythme du programme.


L’évaluation diagnostique a permis de dégager une vue d’ensemble des difficultés rencontrées par les élèves, tant sur le plan conceptuel que sur celui des compétences en communication. Pour garantir une progression harmonieuse, il est crucial de personnaliser l’enseignement en fonction des besoins des élèves et d’encourager une approche collaborative. Les solutions proposées visent à renforcer à la fois les notions fondamentales et les compétences d’expression, afin d’aider chaque élève à réussir dans le programme de physique-chimie.
 
 \vspace{9cm}
 SIGNATURE DU PROFESSEUR \hspace{3cm} DIRECTEUR  \hspace{3cm} INSPECTEUR

 \vspace{3.85cm}
\emph{Pièces jointes : Evaluation Diagnostique + Fiches pédagogiques.}
\end{document}
