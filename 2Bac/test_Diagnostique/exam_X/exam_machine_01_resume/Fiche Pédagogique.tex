\documentclass[12pt]{article}
\usepackage[a4paper, margin=.30in]{geometry}
%\usepackage{array}
\usepackage{fancybox}

\usepackage{graphicx, subfig, wrapfig, makecell }
\usepackage{multirow}

\newcommand\headerMe[2]{\noindent{}#1\hfill#2}
\renewcommand \thesection{\Roman{section}}

\newcolumntype{M}[1]{>{\raggedright}m{#1}}




\begin{document}

\headerMe{Royaume du Maroc}{année scolaire \emph{2022-2023}}\\
\headerMe{Ministère de l'Éducation nationale, }{  Professeur :\emph{Zakaria Haouzan}}\\
\headerMe{du Préscolaire et des Sports}{Établissement : \emph{Lycée SKHOR qualifiant}}\\

\begin{center}
Evaluation Diagnostique \\
Durée 1h45
\\
    \vspace{.2cm}
\hrulefill
\Large{Fiche Pédagogique}
\hrulefill\\
\end{center}
%end Headerss------------------------


%__________________Chimie ______________________-
%%%%%%%+_+_+_+_+_+_+_+_+_Partie1
\section[A]{Introduction }
\hspace{0.5cm}Le programme d'études de la matière physique chimie vise à croître un ensemble de compétences visant à développer la personnalité de l'apprenant. Ces compétences peuvent être classées en Compétences transversales communes et Compétences qualitatives associées aux différentes parties du programme.
\section{cadre de référence }
 \hspace{0.5cm}
L'évaluation diagnostique s'appuie sur l'analyse des programmes des années scolaires passées et de l'année scolaire en cours pour déterminer les apprentissages fondamentaux antérieurs sur lesquels s'appuient les apprentissages ciblés au niveau actuel.

Cette évaluation diagnostique vise à atteindre les objectifs suivants :
\begin{itemize}
	\item Permettre aux enseignants de déterminer avec précision les forces et les faiblesses des acquis des élèves.
	\item Identifier les domaines prioritaires et soutenir les activités prévues.
	\item Permettre à la Direction de l'éducation de fournir des données diagnostiques exactes sur la réussite scolaire des élèves.
\end{itemize}
 %\begin{center}
%\begin{tabular}{|c||c||c|}
%\hline
    %\textbf{Restitution des Connaissances} & \textbf{Application des Connaissances} & \textbf{Situation Problème }\\
    %\hline
    %$50\%$ & $25\%$ & $25\%$\\
    %\hline
%\end{tabular} 
%\end{center}

\section{tableau de spécification}
 \begin{center}
\begin{tabular}{||c|c|c|c|c|}
\hline
\multicolumn{2}{||c|}{\makecell{Domaine}} & \makecell{Taux d'importance du domaine} & \multicolumn{2}{|c|}{Importance du niveau de compétence }\\\cline{4-5}
\multicolumn{2}{||c|}{}  &  &\makecell{Connaissances \\49\%} &\makecell{Application\\51\%}\\\hline
\multirow{3}{*}{\makecell{Physique\\70\%}}  & Mécanique & 43\% & 21\% & 22\%\\\cline{2-5}
											& électrodynamique & 22\% & 11\% & 11\%\\\cline{2-5}
											& Optique   & 5\% & 2\% & 3\%\\\hline
\multicolumn{2}{||c|}{\makecell{Chimie 30\%}}    & 30\% &15\% & 16\% \\\hline\hline
\multicolumn{2}{||c|}{\makecell{total}}     & 100\%     &49\% & 51\% \\\hline
	
    \hline
\end{tabular} 
\end{center}
\section*{Nombre d'indicateur pour chaque domaine}
 \begin{center}
\begin{tabular}{|c|c|c|c|c|c|}
\hline
\makecell{Domaine} & \makecell{Mécanique\\43\%} & \makecell{électrodynamique \\22\%} & \makecell{Optique\\5\%} & \makecell{Chimie \\30\%} & \makecell{Nombre d'indicateur} \\\hline

\makecell{Connaissances 49\%} & 8 & 4 & 1 & 8 & 20\\\hline
\makecell{Application 51\%}   & 8 & 4 & 1 & 9 & 21\\\hline
\makecell{Total 100\%} & 16 & 8 & 2 & 17 & 43\\\hline


\end{tabular} 
\end{center}
\begin{center}
	\shadowbox{\bf{ Evaluation Diagnostique } }
\end{center}



\newpage
\begin{center}
  \begin{tabular}{|c||c||c|}
    \hline
	\multicolumn{3}{||c||}{\bf{   \hfill  Physique 70\%  \hfill (62pts)} }\\
         \hline
         \multicolumn{3}{||c||}{\bf{Partie 1 : Mécanique et énergie \dotfill (42pts)} }\\
\hline
    \textbf{$N^{\circ}$Q } & \textbf{Réponse } & \textbf{Note }\\
    \hline
    $1$ & \makecell{
		La relation entre la vitesse linéaire et la vitesse
angulaire est\\ (a) $V=R.\omega$}  & $+2pt$\\\hline
%Q2    
$2$ & \makecell{Unité de la puissance d’une force est :(c) Watt} & $+2pt$\\\hline
%Q3:
$3$ & \makecell{(b)Le travail de la somme des forces est nul } & $+2pt$\\\hline
 %Q4:
$4$ & \makecell{(a) La fréquence f s’exprime en Hertz ( Hz ) } & $+2pt$\\\hline
 %Q5:
$5$ & \makecell{(a) $\omega = 20\pi rad/s$} & $+2pt$\\\hline
 %Q6:
$6$ & \makecell{(b) $W_{AB} = F.AB.cos(\alpha)$} & $+2pt$\\\hline
 %Q7:
$7$ & \makecell{ (a) Le travail d’une force constante,lors du
déplacement \\de son point d’application entre A
et B.\\ ne dépend pas du chemin suivi entre A et
B } & $+2pt$\\\hline
 %Q8:
$8$ & \makecell{(b)La puissance instantanée $P=\vec{F}.\vec{v}$} & $+2pt$\\\hline
 %Q9:
$9$ & \makecell{(b) L’énergie cinétique EC d’un solide en mouvement \\de translation est $E_c =\frac{1}{2}.m.v^2  $} & $+2pt$\\\hline
 %Q10:
$10$ & \makecell{(b) $E_pp = mgz + C$} & $+2pt$\\\hline
 %Q11:
$11$ & \makecell{(a) augmente } & $+2pt$\\\hline
 %Q12.a:
$12.a$ & \makecell{(ii) $W_{AB} = -mgh$ } & $+4pt$\\\hline
 %Q12.b:
$12.b$ & \makecell{(i) $W_{AB} = -200J$ } & $+4pt$\\\hline
 %Q12.c:
$12.c$ & \makecell{(ii) $P = 20J$ } & $+4pt$\\\hline
 %Q12.d:
$12.d$ & \makecell{(ii) $Ec_{AB} = 100J$ } & $+4pt$\\\hline
 %Q12.e:
$12.e$ & \makecell{(iv) $\Delta{Ec_{AB}} = -200J$ } & $+4pt$\\\hline

      %Partie 2 : -----
\multicolumn{3}{||c||}{\bf{Partie 2 : électrodynamique \dotfill (17pts)} }\\
\hline
%1
 $1$ & \makecell{(c) $P_e = U_{AB}.I$} & $+2pt$\\\hline
%2
$2$ & \makecell{ (a,c)$W_j=U_{AB}.I.\Delta{t}$ et $W_j=R.I^2.\Delta{t}$ }& $+4pt$\\\hline
%3 
		  $3$ & \makecell{ (a) $U_{PN}=E-rI$}& $+2pt$\\\hline
%4 
$4$ & \makecell{ (a) $Pu=E'I$}& $+2pt$\\\hline
%5 
$5$ & \makecell{ (c) $\rho=\frac{W_u}{W_r} = \frac{Pu}{Pr}$}& $+2pt$\\\hline
%6 
$6$ & \makecell{ (b,c) }& $+2pt$\\\hline
%7 
$7$ & \makecell{ (b) faux}& $+1pt$\\\hline
%8
$8$ & \makecell{ (a,c) }& $+2pt$\\\hline

      %Partie 2 : -----
\multicolumn{3}{||c||}{\bf{Partie 3 : Optique \dotfill (3pts)} }\\
\hline
%1
 $1$ & \makecell{(a) vrai  }& $+1pt$\\\hline
%2
$2$ & \makecell{(b) $n_1sin(i_1)=n_2sin(i_2)$  }& $+2pt$\\\hline
  \hline
	\multicolumn{3}{||c||}{\bf{   \hfill  Chimie 30\%  \hfill (38pts)} }\\
         \hline
\hline
%Q1    
$1$ & \makecell{(a)La quantité de matière $n=\frac{m}{M}$  } & $+2pt$\\\hline
%Q2    
$2$ & \makecell{(a)P.V=n.R.T  } & $+2pt$\\\hline
%Q3    
$3$ & \makecell{(a)T(K) = T(c)+273,15  } & $+2pt$\\\hline
%Q4    
$4$ & \makecell{(d) c.V=n  } & $+2pt$\\\hline
%Q5    
$5$ & \makecell{(b) c=0,5mol/L  } & $+2pt$\\\hline
%Q6    
$6$ & \makecell{(a)Différentes  } & $+2pt$\\\hline
%Q7   
$7$ & \makecell{(a)  } & $+2pt$\\\hline
%Q8   
$8$ & \makecell{(c)  $\sigma = 0,14 S/m$ } & $+2pt$\\\hline
%Q9   
$9$ & \makecell{(a)  } & $+2pt$\\\hline
%Q10   
$10.a$ & \makecell{(i) $x_max =9mmol$  } & $+3pt$\\\hline
%Q10.b   
$10.b$ & \makecell{(i) Fe  } & $+3pt$\\\hline
%Q10.c   
$10.c$ & \makecell{(ii) V=216mL  } & $+4pt$\\\hline
%Q11   
$11$ & \makecell{(b) $C_nH{2n+2}$ } & $+2pt$\\\hline
%Q12   
$12$ & \makecell{ (b) 2-méthylpropan-1- ol } & $+2pt$\\\hline
%Q13   
$13$ & \makecell{(c) -OH  } & $+2pt$\\\hline
%Q14 
$14$ & \makecell{(b) La concentration massique g/L} & $+2pt$\\\hline
%Q15   
$15$ & \makecell{(b)  } & $+2pt$\\\hline

  \end{tabular}
  \end{center}


\end{document}
