\documentclass[12pt]{article}
%\usepackage{array}

\usepackage[a4paper, margin=.29in]{geometry}
\usepackage{graphicx, subfig, wrapfig, makecell,fancyhdr,xcolor }
\newcommand\headerMe[2]{\noindent{}#1\hfill#2}
\renewcommand \thesection{\Roman{section}}
\usepackage{amsmath,amssymb}
\usepackage{graphicx}
\usepackage{array}
\usepackage{tabularx}
\usepackage{multirow}
\usepackage{colortbl}
\usepackage{booktabs}
\usepackage{enumitem}

\definecolor{lightgray}{rgb}{0.9,0.9,0.9}
\definecolor{lightblue}{rgb}{0.8,0.85,1.0}
\newcolumntype{M}[1]{>{\raggedright}m{#1}}
\begin{document}

\begin{center}
\includegraphics[width = 0.18\textwidth]{./img/logoMin.png}
\vspace{-3cm}
\end{center}
\headerMe{Matière : Physique-Chimie}{Établissement : \emph{Lycée SKHOR qualifiant}}\\
\headerMe{ Unité :  Sens d'évolution d'un système chimique}{  Professeur :\emph{Zakaria Haouzan}}\\
\headerMe{Niveau : 2BAC-SM-X}{Heure : 4H}\\

\begin{center}
	%\vspace{1cm}
\underline{Leçon N°3: Transformations forcées - L'électrolyse }\\
Durée 4h00
\\
    \vspace{.2cm}
\hrulefill
\Large{Fiche Pédagogique}
\hrulefill\\
\end{center}
%end Headerss------------------------


%__________________Chimie ______________________-
%%%%%%%+_+_+_+_+_+_+_+_+_Partie1

\begin{tabular}{|p{4.5cm}|p{4.5cm}|p{4.5cm}|p{3cm}|}
\hline
\rowcolor{lightblue}
  \textbf{Prérequis} & \textbf{Compétences visées}& \textbf{Savoir et savoir-faire} & \textbf{Outils didactiques} \\
\hline
\hline
\begin{itemize}[leftmargin=*]
  \item Connaître le concept de transformation spontanée
  \item Maîtriser les réactions d'oxydoréduction
  \item Connaître les couples Ox/Red
  \item Comprendre le fonctionnement des piles électrochimiques
  \item Connaître la notion de transfert d'électrons
\end{itemize}
&
\begin{itemize}[leftmargin=*]
  \item Comprendre la différence entre transformation spontanée et transformation forcée
  \item Maîtriser le concept d'électrolyse
  \item Relier les phénomènes de la vie quotidienne aux concepts d'électrolyse
  \item Résoudre des problèmes en rapport avec les transformations forcées
  \item Utiliser la méthode scientifique pour analyser les phénomènes d'électrolyse
\end{itemize} 
&
\begin{itemize}[leftmargin=*]
\item Distinguer une transformation spontanée d'une transformation forcée
\item Identifier le sens d'évolution d'un système chimique
\item Déterminer les réactions aux électrodes dans une électrolyse
\item Comprendre le rôle du générateur dans une électrolyse
\item Analyser des applications industrielles de l'électrolyse
\end{itemize}
&
\begin{itemize}[leftmargin=*]
  \item Générateur de courant continu
  \item Électrodes de graphite, cuivre et zinc
  \item Solutions de bromure de cuivre, chlorure de sodium, sulfate de cuivre
Tubes en U
      \item Ordinateur avec simulation d'électrolyse
(si disponible)
\end{itemize}

  \\
\hline
\end{tabular}
%\vspace{1cm}

\begin{center}
\colorbox{lightblue}{\parbox{15cm}{\centering\textbf{Situation-problème}}}
\end{center}

\begin{tabular}{|p{16cm}|}
\hline

Dans une usine de bijouterie, on souhaite recouvrir des objets en fer d'une fine couche d'or pour leur donner l'apparence de bijoux en or massif. Un technicien propose d'utiliser une pile électrochimique pour ce faire, mais son collègue affirme qu'une pile ne peut pas réaliser ce dépôt et qu'il faut utiliser une autre méthode.


\begin{enumerate}
  \item  Quelle est la différence entre une pile électrochimique et un électrolyseur?
  \item Pourquoi ne peut-on pas utiliser une pile pour réaliser un dépôt métallique?
  \item Quel dispositif permettrait de réaliser ce dépôt et comment fonctionne-t-il?
\end{enumerate} \\
\hline
\end{tabular}

\vspace{1cm}

\begin{center}
\colorbox{lightblue}{\parbox{15cm}{\centering\textbf{Déroulement}}}
\end{center}

\begin{tabularx}{\textwidth}{|p{3.5cm}|X|X|p{2.5cm}|}
\hline
\rowcolor{lightgray}
\textbf{Éléments du cours} & \textbf{Activités de l'enseignant} & \textbf{Activités de l'apprenant} & \textbf{Évaluation} \\
\hline
\textbf{I. Rappel sur les transformations spontanées} & 
\begin{itemize}[leftmargin=*]
    \item Le professeur pose la situation-problème.
    \item Le professeur demande aux apprenants de répondre aux questions de la situation-problème.
    \item Écrit les hypothèses proposées par les apprenants.
    \item Garde les hypothèses convenues pour vérifier pendant le cours.
    \item Le professeur rappelle le concept de transformation spontanée à l'aide de l'exemple de la réaction entre le cuivre métal et le dibrome.
\end{itemize} & 
\begin{itemize}[leftmargin=*]
\item L'apprenant analyse la situation et formule des hypothèses
\item   \textbf{Exemple des hypothèses attendues:}
\begin{itemize}
    \item Une pile électrochimique utilise une réaction spontanée pour produire de l'électricité.
    \item Pour réaliser un dépôt métallique, il faut forcer les ions métalliques à se réduire, ce qui nécessite un apport d'énergie.
\end{itemize}

\end{itemize} & 
Évaluation diagnostique \\
\hline

\textbf{II. Transformation spontanée vs transformation forcée} 
& 
\begin{itemize}[leftmargin=*]
   \item Réalise l'expérience de la réaction entre $\text{Cu}_{(s)}$ et $\text{Br}_{2(aq)}$.
    \item Fait constater que la réaction est spontanée dans le sens $\text{Cu} + \text{Br}_2 \rightarrow \text{Cu}^{2+} + 2\text{Br}^-$.
    \item Pose la question: "Que se passerait-il si on mélangeait initialement les ions $\text{Cu}^{2+}$ et les ions $\text{Br}^-$?"
    \item Introduit la notion de transformation forcée.
\end{itemize} & 
\begin{itemize}[leftmargin=*]
\item Observe l'expérience et constate:
    \begin{itemize}
        \item la disparition de la coloration du dibrome
        \item la disparition du métal cuivre
        \item l'apparition d'une coloration bleue due aux ions $\text{Cu}^{2+}$
    \end{itemize}
    \item Écrit l'équation de la réaction: $\text{Cu}_{(s)} + \text{Br}_{2(aq)} \rightarrow \text{Cu}^{2+}_{(aq)} + 2\text{Br}^-_{(aq)}$
    \item Calcule le quotient de réaction initial et le compare à la constante d'équilibre.
    \item Conclut que la réaction inverse ($\text{Cu}^{2+} + 2\text{Br}^- \rightarrow \text{Cu} + \text{Br}_2$) n'est pas spontanée.
    \item Comprend qu'il faut apporter de l'énergie électrique pour forcer cette réaction non spontanée.
\end{itemize}

& 
Évaluation formative \\
\hline

\end{tabularx}

\begin{tabularx}{\textwidth}{|p{3.5cm}|X|X|p{2.5cm}|}
\hline
\rowcolor{lightgray}
\textbf{Éléments du cours} & \textbf{Activités de l'enseignant} & \textbf{Activités de l'apprenant} & \textbf{Évaluation} \\
\hline\hline
\textbf{III. L'électrolyse: définition et principe} & 
\begin{itemize}[leftmargin=*]
     \item Définit l'électrolyse comme une transformation forcée qui se déroule dans le sens opposé à l'évolution spontanée.
    \item Réalise le montage d'électrolyse d'une solution de bromure de cuivre.
    \item Identifie l'anode et la cathode dans un électrolyseur.
    \item Fait observer les phénomènes aux électrodes.
\end{itemize} & 
\begin{itemize}[leftmargin=*]
     \item Note la définition de l'électrolyse.
    \item Observe le dépôt de cuivre sur la cathode et la formation de dibrome à l'anode.
    \item Identifie l'anode comme électrode reliée au pôle positif et la cathode au pôle négatif.
    \item Compare avec le cas d'une pile où les polarités sont inversées.
    \item Écrit les demi-équations électroniques et l'équation bilan:
    \begin{itemize}
        \item À l'anode: $2\text{Br}^- \rightarrow \text{Br}_2 + 2\text{e}^-$
        \item À la cathode: $\text{Cu}^{2+} + 2\text{e}^- \rightarrow \text{Cu}$
        \item Bilan: $\text{Cu}^{2+} + 2\text{Br}^- \rightarrow \text{Cu} + \text{Br}_2$
    \end{itemize}
\end{itemize} & 
Évaluation formative \\
\hline

\textbf{IV. Exemples d'électrolyses}
 & 
\begin{itemize}[leftmargin=*]
    \item Réalise l'électrolyse d'une solution de chlorure de sodium.
    \item Fait observer les dégagements gazeux aux électrodes.
    \item Guide l'interprétation des observations.
    \item Introduit la notion d'électrolyse à anode soluble.

\end{itemize} & 
\begin{itemize}[leftmargin=*]
      \item Observe le dégagement de dichlore à l'anode et de dihydrogène à la cathode.
    \item Identifie les couples redox mis en jeu: $\text{Cl}_2/\text{Cl}^-$, $\text{Na}^+/\text{Na}$, $\text{H}_2\text{O}/\text{H}_2$ et $\text{O}_2/\text{H}_2\text{O}$.
    \item Écrit les demi-équations possibles à chaque électrode:
    \begin{itemize}
        \item À l'anode: $2\text{Cl}^- \rightarrow \text{Cl}_2 + 2\text{e}^-$ ou $2\text{H}_2\text{O} \rightarrow \text{O}_2 + 4\text{H}^+ + 4\text{e}^-$
        \item À la cathode: $\text{Na}^+ + \text{e}^- \rightarrow \text{Na}$ ou $2\text{H}_2\text{O} + 2\text{e}^- \rightarrow \text{H}_2 + 2\text{HO}^-$
    \end{itemize}
    \item Détermine les réactions effectives d'après les observations.
    \item Comprend le principe de l'électrolyse à anode soluble et son intérêt (dépôt métallique, purification).
\end{itemize}
& 
Évaluation formative \\
\hline

\end{tabularx}
\begin{tabularx}{\textwidth}{|p{3.5cm}|X|X|p{2.5cm}|}
\hline
\rowcolor{lightgray}
\textbf{Éléments du cours} & \textbf{Activités de l'enseignant} & \textbf{Activités de l'apprenant} & \textbf{Évaluation} \\
\hline\hline


\textbf{V. Applications industrielles de l'électrolyse} & 
\begin{itemize}[leftmargin=*]
       \item Présente différentes applications de l'électrolyse.
    \item Montre l'exemple concret du placage de métaux précieux.
    \item Explique le principe de la recharge d'un accumulateur.
    \item Fait le lien avec la situation-problème initiale.
\end{itemize} & 
\begin{itemize}[leftmargin=*]
       \item Identifie les applications industrielles de l'électrolyse
    \item Comprend que le placage en bijouterie utilise le principe de l'électrolyse.
    \item Explique le fonctionnement d'un accumulateur en mode décharge (pile) et en mode charge (électrolyse).
    \item Fait le lien avec la situation-problème initiale et propose une solution.
\end{itemize} & 
Évaluation sommative \\
\hline
\end{tabularx}

\begin{center}
\colorbox{lightblue}{\parbox{15cm}{\centering\textbf{Activité pratique }}}
\end{center}


\begin{center}
\fcolorbox{black}{lightgray}{%
\begin{minipage}{0.95\textwidth}
\textbf{Objectif:} Réaliser une électrolyse à anode soluble pour comprendre le principe du placage métallique.

\textbf{Matériel:}
\begin{itemize}
    \item Générateur de courant continu
    \item Électrode de cuivre (anode)
    \item Objet métallique à plaquer (cathode)
    \item Solution de sulfate de cuivre $\text{CuSO}_4$
    \item Fils de connexion
    \item Chronomètre
    \item Balance de précision
\end{itemize}

\textbf{Protocole:}
\begin{enumerate}
    \item Peser la cathode (objet à plaquer) avant l'expérience
    \item Monter le circuit électrique: relier l'anode au pôle positif et la cathode au pôle négatif du générateur
    \item Immerger les électrodes dans la solution de sulfate de cuivre
    \item Mettre en marche le générateur avec une tension d'environ 4V pendant 10 minutes
    \item Arrêter le générateur, retirer la cathode, la rincer, la sécher et la peser à nouveau
\end{enumerate}

\textbf{Questions:}
\begin{enumerate}
    \item Écrire les demi-équations aux électrodes et l'équation bilan de la réaction
    \item Calculer la masse de cuivre déposée et vérifier si elle correspond à la valeur théorique
    \item Expliquer pourquoi on appelle ce type d'électrolyse "électrolyse à anode soluble"
    \item Citer des applications pratiques de ce type d'électrolyse
\end{enumerate}
\end{minipage}}
\end{center}




\end{document}
