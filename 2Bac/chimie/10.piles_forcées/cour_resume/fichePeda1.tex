\documentclass[12pt]{article}
\usepackage[a4paper, margin=.20in]{geometry}
%\usepackage{array}
\usepackage{graphicx, subfig, wrapfig, makecell,fancyhdr,xcolor }
\newcommand\headerMe[2]{\noindent{}#1\hfill#2}
\renewcommand \thesection{\Roman{section}}

\newcolumntype{M}[1]{>{\raggedright}m{#1}}

\chead{\includegraphics[width = 0.1\textwidth]{./img/logoMin.png}}
\cfoot{helllo}

\begin{document}

\begin{center}
\includegraphics[width = 0.18\textwidth]{./img/logoMin.png}
\vspace{-3cm}
\end{center}
\headerMe{Matière : Chimie}{Établissement : \emph{Lycée SKHOR qualifiant}}\\
\headerMe{ Unité :  Sens d'évolution d'un \\système chimique}{  Professeur :\emph{Zakaria Haouzan}}\\
\headerMe{Niveau : 2BAC-SM-pc}{Heure : 6H}\\

\begin{center}
	%\vspace{1cm}
	\underline{Leçon $N^{\circ}$10: Transformations forcées - L’électrolyse}\\
Durée 2h00
\\
    \vspace{.2cm}
\hrulefill
\Large{Fiche Pédagogique}
\hrulefill\\
\end{center}
%end Headerss------------------------


%__________________Chimie ______________________-
%%%%%%%+_+_+_+_+_+_+_+_+_Partie1
\section{Objectifs Pédagogiques}
\begin{itemize}
    \item Différencier les transformations spontanées et forcées.
    \item Identifier les rôles de l'anode et de la cathode dans l'électrolyse.
    \item Analyser les réactions d'oxydo-réduction lors de l'électrolyse.
    \item Explorer les applications industrielles de l'électrolyse.
\end{itemize}

\section{Situation Problématique Initiale}
\textbf{Contexte :}  
Vous êtes un chimiste chargé de récupérer du cuivre à partir d’une solution usée contenant des ions Cu\(^{2+}\) et Br\(^-\). Spontanément, le cuivre métallique réagit avec le dibrome (Br\(_{2}\)) pour former Cu\(^{2+}\) et Br\(^-\). Comment inverser ce processus pour déposer du cuivre sur un objet et régénérer le dibrome ? Quelle source d’énergie utiliser ?

\textbf{Rôle de l’enseignant :}
\begin{itemize}
    \item Poser des questions : 
    \begin{itemize}
        \item  Que se passe-t-il spontanément entre Cu et Br\(_{2}\) ? 
        \item  Comment obliger la réaction inverse à se produire ? 
    \end{itemize}
    \item Encourager le rappel des constantes d’équilibre.
\end{itemize}

\textbf{Rôle des élèves :}
\begin{itemize}
    \item Discuter : Pourquoi Cu + Br\(_{2}\) est-il spontané ? Que signifie K = 1,2 × 10\(^{25}\) ?
    \item Proposer : Peut-on utiliser l’électricité pour inverser la réaction ?
\end{itemize}

\section{Déroulement de la Leçon}

\subsection{Phase 1 : Transformations Spontanées (1h)}
\textbf{Activités de l’enseignant :}
\begin{itemize}
    \item Réviser : Cu\(_{(s)}\) + Br\(_{2(aq)} \rightleftharpoons\) Cu\(^{2+}_{(aq)}\) + 2Br\(^-_{(aq)}\).
    \item Questionner : Pourquoi cette réaction est-elle spontanée ? Que signifie Q\(_{r,i}\) < K ? 
    \item Démontrer (ou décrire) : Mélange de tournure de cuivre et solution de Br\(_{2}\).
\end{itemize}

\textbf{Activités des élèves :}
\begin{itemize}
    \item Répondre :  K élevé indique une réaction totale dans le sens direct. 
    \item Observer : Décoloration de Br\(_{2}\), bleu des ions Cu\(^{2+}\).
    \item Demander : La réaction inverse est-elle possible spontanément ? 
\end{itemize}

\textbf{Matériel :} Tournure de cuivre, solution de Br\(_{2}\) (10\(^{-2}\) mol/L), tube à essai.

\subsection{Phase 2 : Introduction aux Transformations Forcées (1h)}
\textbf{Activités de l’enseignant :}
\begin{itemize}
    \item Définir :  Une transformation forcée va à l’encontre de l’évolution spontanée, nécessitant une énergie externe. 
    \item Questionner : Comment forcer Cu\(^{2+}\) + 2Br\(^-\) à former Cu et Br\(_{2}\) ?
    \item Introduire l’électrolyse comme solution.
\end{itemize}

\textbf{Activités des élèves :}
\begin{itemize}
    \item Hypothèse :  Quel montage pour fournir de l’énergie électrique ?
    \item Demander : Que deviennent les ions pendant l’électrolyse ?
\end{itemize}

\subsection{Phase 3 : Expérience - Électrolyse de CuBr\(_{2}\) (2h)}
\textbf{Activités de l’enseignant :}
\begin{itemize}
    \item Préparer un tube en U avec CuBr\(_{2}\) et électrodes en graphite.
    \item Questionner :  Que se passe-t-il pour U > 1,2 V ? 
    \item Expliquer :
    \begin{itemize}
        \item Anode : 2Br\(^-\) \(\rightleftharpoons\) Br\(_{2}\) + 2e\(^-\)
        \item Cathode : Cu\(^{2+}\) + 2e\(^-\) \(\rightleftharpoons\) Cu
    \end{itemize}
\end{itemize}

\textbf{Activités des élèves :}
\begin{itemize}
    \item Observer : Cu se dépose à la cathode, Br\(_{2}\) (orange) près de l’anode.
    \item Demander :  Pourquoi Cu à la cathode et Br\(_{2}\) à l’anode ?
    \item Écrire : Cu\(^{2+}\) + 2Br\(^-\) \(\rightleftharpoons\) Cu + Br\(_{2}\).
\end{itemize}

\textbf{Matériel :} Tube en U, solution de CuBr\(_{2}\), électrodes graphite, alimentation DC (>1,2 V), fils, lunettes de sécurité.

\textbf{Étapes :}
\begin{enumerate}
    \item Remplir le tube avec CuBr\(_{2}\).
    \item Placer les électrodes dans chaque bras.
    \item Connecter l’anode (+) et la cathode (-).
    \item Appliquer U > 1,2 V pendant 15 min.
    \item Noter les observations.
\end{enumerate}

\subsection{Phase 4 : Électrolyse de NaCl (1h)}
\textbf{Activités de l’enseignant :}
\begin{itemize}
    \item Réaliser l’électrolyse de NaCl.
    \item Questionner : Quels gaz ou produits attendre ?
    \item Analyser :
    \begin{itemize}
        \item Anode : 2Cl\(^-\) \(\rightleftharpoons\) Cl\(_{2}\) + 2e\(^-\)
        \item Cathode : 2H\(_{2}\)O + 2e\(^-\) \(\rightleftharpoons\) H\(_{2}\) + 2OH\(^-\)
    \end{itemize}
\end{itemize}

\textbf{Activités des élèves :}
\begin{itemize}
    \item Prédire : Le sodium métallique se forme-t-il ?
    \item Observer : Cl\(_{2}\) (jaune-vert) à l’anode, H\(_{2}\) et OH\(^-\) à la cathode.
    \item Écrire : 2H\(_{2}\)O + 2Cl\(^-\) \(\longrightarrow\) Cl\(_{2}\) + H\(_{2}\) + 2OH\(^-\).
\end{itemize}

\textbf{Matériel :} Solution NaCl, tube en U, électrodes graphite, alimentation DC, indicateur pH.

\subsection{Phase 5 : Anode Soluble et Applications (1h)}
\textbf{Activités de l’enseignant :}
\begin{itemize}
    \item Démontrer : Électrolyse avec anode Cu et cathode Fe.
    \item Expliquer : Transfert Cu \(\rightarrow\) Cu\(^{2+}\) (anode) puis Cu\(^{2+}\) \(\rightarrow\) Cu (cathode).
    \item Lister les applications (purification, placage).
\end{itemize}

\textbf{Activités des élèves :}
\begin{itemize}
    \item Observer : Cu disparaît de l’anode et se dépose sur la cathode.
    \item Demander : Pourquoi la couleur de la solution reste-t-elle constante ?
    \item Citer 3 applications industrielles.
\end{itemize}

\textbf{Matériel :} Anode Cu, cathode Fe (clé), solution CuSO\(_{4}\), alimentation DC.

\section{Évaluation}
\textbf{Questions de l’enseignant :}
\begin{itemize}
    \item  Quelle est la différence entre transformations spontanées et forcées ?
    \item  Quels facteurs déterminent les réactions dans l’électrolyse ? 
\end{itemize}

\textbf{Tâches des élèves :}
\begin{itemize}
    \item Résoudre :  Prédire les produits de l’électrolyse de ZnCl\(_{2}\).
    \item Réfléchir : Impact de l’électrolyse dans la vie quotidienne. 
\end{itemize}

\section{Consignes de Sécurité}
\begin{itemize}
    \item Manipuler Br\(_{2}\) et Cl\(_{2}\) sous ventilation.
    \item Porter des lunettes et gants.
    \item Éliminer les déchets chimiques correctement.
\end{itemize}



\end{document}
