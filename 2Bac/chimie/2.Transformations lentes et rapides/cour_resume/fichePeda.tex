\documentclass[12pt]{article}
\usepackage[a4paper, margin=.20in]{geometry}
%\usepackage{array}
\usepackage{graphicx, subfig, wrapfig, makecell,fancyhdr,xcolor }
\newcommand\headerMe[2]{\noindent{}#1\hfill#2}
\renewcommand \thesection{\Roman{section}}

\newcolumntype{M}[1]{>{\raggedright}m{#1}}

\chead{\includegraphics[width = 0.1\textwidth]{./img/logoMin.png}}
\cfoot{helllo}

\begin{document}

\begin{center}
\includegraphics[width = 0.18\textwidth]{./img/logoMin.png}
\vspace{-3cm}
\end{center}
\headerMe{Matière : Physique-Chimie}{Établissement : \emph{Lycée SKHOR qualifiant}}\\
\headerMe{ Unité : Transformations lentes et rapides\\d’un système chimique }{  Professeur :\emph{Zakaria Haouzan}}\\
\headerMe{Niveau : 2BAC-SM-X}{Heure : 2H}\\

\begin{center}
	%\vspace{1cm}
	\underline{Leçon $N^{\circ}$2: Transformations lentes et rapides}\\
Durée 2h00
\\
    \vspace{.2cm}
\hrulefill
\Large{Fiche Pédagogique}
\hrulefill\\
\end{center}
%end Headerss------------------------


%__________________Chimie ______________________-
%%%%%%%+_+_+_+_+_+_+_+_+_Partie1
 \begin{center}
	 \begin{tabular}{|p{0.2\textwidth}||p{0.3\textwidth}||p{0.3\textwidth}||p{0.1\textwidth}|}
\hline
\textbf{Prérequis} & \textbf{Compétences visées } & \textbf{Savoir et savoir-faire}&\textbf{Outils didactiques }\\
    \hline

---Connaitre Réaction d’oxydation : Transfere des électrons

---Définition oxydante et réducteur

---les couples Ox/Red

---L’équation bilan de la réaction ox/Red


				   &
				 ---Relier les phénomènes de la
vie quotidienne aux concepts
et principes desTransformations lentes et rapides.

---Résoudre un problème en rapport avec les Transformations lentes et rapides.

---Utiliser la méthode scientifique à différents stades afin d'analyser les différents problèmes liés aux Transformations lentes et rapides.

---Acquisition d'une
méthodologie de recherche
Méthodologie d'action Autoapprentissage

 & 
--- Écrire l’équation de la réaction associée à une transformation d’oxydoréduction

--- Mettre en évidence l’influence du facteur temps dans le déroulement d’un certain nombre de
transformations d’oxydoréduction

--- Connaitre l’influence de la concentration des réactifs et de la température sur le déroulement temporel d’une réaction.

--- Interpréter cette influence au niveau microscopique.


 & 

---Ordinateur  

---simulation data-show 

---tube à essaies

---solution de nitrate d’argent

---d’acide chlorhydrique

---solution de chlorure de fer III

---solution d’hydroxyde de sodium

---l’eau oxygénée

---solution d’iodure de potassium
\\
    \hline
\end{tabular} 
\end{center}
\section*{Situation-problème :}
La combustion d’un gaz dans l’air est une transformation rapide et la formation d’une
couche de rouille sur une surface métallique est une transformation lente et les deux
transformations sont des réactions d’oxydoréduction.

\begin{enumerate}
	\item Qu’est-ce qu’une réaction d’oxydoréduction ?
	\item Qu’est-ce qu’une transformation rapide et une transformation lente ?
	\item Peut-on accélérer ou ralentir une réaction chimique ?

\end{enumerate}

\begin{center}
	 \begin{tabular}{|p{0.2\textwidth}||p{0.3\textwidth}||p{0.3\textwidth}||p{0.1\textwidth}|}
\hline
\multicolumn{4}{|c|}{Déroulement}\\\hline
Eléments du & \multicolumn{2}{c||}{Activités didactiques} &  \\\cline{2-3}
cours & Enseignant & Apprenant & Evaluation\\\hline

\color{red}{I -Rappels sur les couples Ox/Red :}	 

\vspace{0.5cm}
\color{blue}I.1-exercices d’application :
\vspace{0.5cm}

	  &
---Le professeur pose la situation-problème.

---Le professeur Demande aux apprenants de répondre aux questions de la situation-problème.

---Ecrire les hypothèses proposées par les apprenants.

---Garde les hypothèses convenues pour vérifier pendant
du cours.

---Le professeur donne des Rappels sur les couples Ox/Red :
				  &
				  -L’apprenant analyse la situation déclenchante
et formule des hypothèses.

\textbf{Exemple des hypothèses attendues :}

---Une réaction d’oxydoréduction est caractérisée par un transfert d’électrons entre l’oxydant d’un couple
ox1 /red1 et le réducteur d’un autre couple ox2 /red2 .

---Une transformation rapide est une transformation qui se fait en une courte durée de
telle façon qu’on ne peut pas suivre son évolution en fonction du temps avec l’œil ou
avec les appareils de mesure.

---On appelle facteur cinétique tout paramètre capable d’influer sur la vitesse d’une transformation chimique.

\vspace{0.5cm}

				  &
				  Evaluation
diagnostique\\\hline


%II- partie 2 
\color{red}{II- Transformations lentes et transformations rapides :}
\vspace{0.5cm}

\color{blue}1-Transformations rapides:
\vspace{0.5cm}

\color{blue}2-Transformations lentes


				  &
				  ---On verse dans un tube à essaies une solution de nitrate d’argent $(Ag^+ + NO^{3-})$ ,puis on lui ajoute une solution
d’acide chlorhydrique $(H_3O^+ + Cl^- )$

--- On verse dans un tube à essaies une solution d’iodure de potassium $(K^+ + I^- )$ puis on lui ajoute un
peu d’eau oxygénée $(H_2O_2 )$ acidifiée avec quelques gouttes d’acide sulfurique $H_2SO_4$
%\emph{Décrire le mouvement du point S}

---Le professeur pose la question suivante : Qu’observez-vous ? quel est le nom du composé produit ?

---Cette réaction peut-elle être suivie à l’œil nu ? conclure
%\emph{Comparer le mouvement de deux points M et N du milieu de propagation dans chacun des cas }


%-Le professeur la corde avec un stroboscope de fréquence réglable Ne :

%\emph{Quelles sont les fréquences Ne des éclairs qui donnent l’immobilité apparente de la corde ? en
%déduire la fréquence maximale }



				  &
   %               --- L’apprenant répond la question :  Le point S a un mouvement rectiligne sinusoïdal
%\vspace{0.5cm}

%--- L’apprenant répond la question : les deux points M et N vibrent en phase
%si la distance qui les sépare est un multiple de la longueur d’onde

%\vspace{0.5cm}

%---L’apprenant répond la question : L’immobilité apparente est obtenue lorsque la fréquence de l’onde N est un multiple de la
%fréquence des éclairs Ne : N= k Ne
%\vspace{0.2cm}
---Répondre aux questionnaires orientées

---On constate la formation d’un précipité blanc de chlorure d’argent AgCl (qui noirci à la lumière) selon
une réaction rapide dont l’équation s’écrit: $Ag^+ + Cl^- \rightarrow AgCl$


--- Il y’a formation progressive du diiode I2 caractérisé par sa coloration brune

--- On constate que la couleur
du mélange réactionnel évolue progressivement du jaune au jaune foncé puis prend une coloration brune
qui devient de plus en plus foncée en fonction du temps .

---Donc la réaction est une réaction lente


---Les élèves écrivent une conclu-
sion dans le cahier.
				  & 
	Évaluation formative\\\hline		  
\end{tabular}
\end{center}

%Page 3 in parte 2

\begin{center}
	 \begin{tabular}{|p{0.2\textwidth}||p{0.3\textwidth}||p{0.3\textwidth}||p{0.1\textwidth}|}
\hline
\multicolumn{4}{|c|}{Déroulement}\\\hline
Eléments du & \multicolumn{2}{c||}{Activités didactiques} &  \\\cline{2-3}
cours & Enseignant & Apprenant & Evaluation\\\hline

%II- partie 2 
\color{red}{III Les facteurs cinétiques}

\vspace{0.5cm}
\color{blue}1 Définition:
\vspace{0.5cm}
\color{blue}2 Influence des facteur cinétique sur la vitesse de la réaction:

\vspace{0.5cm}
\color{red} Quelques application des facteurs cinétiques	
	  &
Activité : 

---Verser dans deux tubes à essais A et B, 10,0 ml des ions permanganate $MnO^{-}_4$ en milieu acide ${H_2C_2O_4}_{(aq)}$ à $0,50 mol/L$.

---À un instant choisi comme origine,
Ajouter en même temps 3 ml d’une
solution acidifiée de permanganate de
potassium à 0,50 mol/L dans chacun des
tubes à essais


---Les ions permanganates $MnO^-_{4(aq)}$ sont violets en solution aqueuse , la solution d’acide oxalique
est incolore ainsi que celle d’acide sulfurique qui sert à acidifier le mélange réactionnel. les ions manganèse $( Mn^{2+}_{(aq)}$ ) sont incolores en solution aqueuse.

\vspace{0.2cm}
-Le professeur pose la question suivante : 


---Écrire l’équation bilan de la réaction 

---Cette réaction est-elle une réaction d'oxydoréduction ? Justifier.

---Qu’observez-vous ? comparer les durées de décoloration (la disparition de la couleur) de chaque mélange

---Que peut-on en déduire ?

---Interpréter ces résultats au niveau microscopique

				  &

---\textbf{Interprétation : }


---L’équation bilan: 

$2MnO^-_4 + 6H^+ + {5H_2C_2O}_4 \rightarrow 2Mn^{2+}  + 10CO_2 + 8H_2O$

---Cette réaction est une réaction d’oxydoréduction car il y a un transfert d’électrons entre les deux
réactifs

---On observe que la disparition de la couleur violette (la décoloration) est plus rapide dans le tube
à essais B à 60°C

---On constate que La vitesse de disparition des ions $MnO_4^- (aq)$ est plus grande quand la
température est plus élevée .Donc la température est un facteur cinétique : Plus la température
d’un mélange réactionnel est grande, plus la réaction est rapide.

---La température est une grandeur liée à l’agitation moléculaire. autrement dit plus la
température est élevée, plus les réactifs sont agités. il est donc logique que le nombre de chocs
efficaces par unité de temps ( par seconde ) soit plus grande et que La vitesse de la réaction soit plus rapide.

---Les élèves écrivent une conclusion dans le cahier.
\vspace{0.5cm}

				  & 
	Évaluation formative\\\hline 
\end{tabular}
\end{center}






\end{document}
