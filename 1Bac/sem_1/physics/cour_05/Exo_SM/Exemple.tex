\documentclass[12pt, french]{article}

\usepackage{fancyhdr, fancybox, lastpage}
\usepackage[most]{tcolorbox}
\usepackage[a4paper, margin={0.3in, .75in}]{geometry}
\usepackage{wrapfig}
\pagestyle{fancy}
\renewcommand\headrulewidth{1pt}
\renewcommand\footrulewidth{1pt}
\fancyhf{}
\rhead{ \em{Zakaria Haouzan}}
\lhead[C]{\em{1ére année baccalauréat Sciences Mathématiques}}
\chead[C]{}
\rfoot[C]{}
\lfoot[R]{}
\cfoot[]{\em{Page \thepage / \pageref{LastPage}}}


\newtcolorbox{Box2}[2][]{
                lower separated=false,
                colback=white,
colframe=white!20!black,fonttitle=\bfseries,
colbacktitle=white!30!gray,
coltitle=black,
enhanced,
attach boxed title to top left={yshift=-0.1in,xshift=0.15in},
title=#2,#1}


\begin{document}
\begin{center}
   \shadowbox {\bf{Energie thermique et Transfert thermique (Sc. Math) }}
\end{center}


%%_________________________Exercice ! :"_________________________Exercice
   \begin{Box2}{Exercice 1 : }
  On admet que dans un calorimètre, seul le vase intérieur (masse $m_1 = 300g$, capacité thermique massique $C_1=0,38.kJ.kg^{-1}K{-1}$ ) et l’agitateur (masse $m_2 = 50 g$, capacité thermique massique $C_2=0,90.kJ.kg^{-1}K^{-1}$) sont
susceptibles de participer aux échanges thermiques avec le contenu de l’appareil.
     \begin{enumerate}

         \item Calculer la capacité thermique $\mu$ du calorimètre.

         \item Ce dernier contient $400 g$ d’éthanol à la température $t_1 = 17,5 ^{\circ} C$ . on y verse $200 g$ d’eau à la
           température $t_2 = 24,7^{\circ}C$ et on note la température lorsque l’équilibre thermique est réalisé, soit $t_e=20,6^{\circ}C$.
En déduire la valeur de la capacité thermique massique C de l’éthanol.
\\Donnée : \\Capacité thermique massique ce de l’eau : $4,19 kJ.kg^{-1}K^{-1}$.
     \end{enumerate}
   \end{Box2}


%%_________________________Exercice !2 :"_________________________Exercice
\begin{Box2}{Exercice 2 : }

  Un calorimètre contient $100 g$ d’eau à $18^{\circ}C$. On y verse $80 g$ d’eau à $60^{\circ}C$.
  \begin{enumerate}
    \item Quelle serait la température d'équilibre si la capacité thermique du calorimètre et de ces accessoires était
négligeable ?
\item La température d’équilibre est en fait $35,9^{\circ}C$. En déduire la capacité thermique du calorimètre et de ses
accessoires.
\item On considère de nouveau le calorimètre qui contient $100 g$ d’eau à $18^{\circ}C$. On y plonge un morceau de
cuivre de masse $20 g$ initialement placé dans de l’eau en ébullition. La température d'équilibre s’établit à
$19,4^{\circ}C$. Calculer la capacité thermique massique du cuivre.
\item  On considère encore le même calorimètre contenant $100 g$ d’eau à $18^{\circ}C$. On y plonge maintenant un
morceau d’aluminium de masse $30,2 g$ à la température de $100^{\circ}C$ et de capacité thermique massique $920J.kg^{-1}K^{-1}$.
Déterminer la température d'équilibre.
\item  L’état initial restant le même : le calorimètre contenant $100 g$ d’eau à $18^{\circ}C$, on y introduit un glaçon de
masse $25 g$ à $0^{\circ}C$. Calculer la température d'équilibre.
\item  L’état initial est encore : le calorimètre contenant $100 g$ d’eau à $18^{\circ}C$, on y introduit un glaçon de masse
$25g$ provenant d’un congélateur à la température de $-18^{\circ}C$. Quelle est la température d’équilibre ?
\\Données :
\\- Capacité thermique massique de l’eau : $C_e = 4,19 kJ.kg^{-1}K^{-1}$.
\\- Capacité thermique massique de la glace : $C_g = 2,10.10^3 J.kg^{-1}.K^{-1}$.
      \\- Chaleur latente de fusion de la glace à $0^{\circ}$C : $L_f = 3,34.10^5 J.kg^{-1}$
\end{enumerate}
\end{Box2}

\vspace{2cm}
\begin{center}
   \Large{ \em{Exercices Supplémentaires}}
\end{center}

%%_________________________Exercice ! 3:"_________________________Exercice

\begin{Box2}{Exercice 3 :}
Un calorimètre contient de l’eau à la température $t1 = 18,3^{\circ}C$ ; sa capacité thermique totale a pour valeur
$\mu=1350 J.K^{-1}$.
On y introduit un bloc de glace, de masse $m = 42 g$, prélevé dans le compartiment surgélation d’un
réfrigérateur à la température $t_2 = -25,5^{\circ}$C. Il y a fusion complète de la glace et la température d’équilibre
est $t = 5,6^{\circ}$C.
On recommence l’expérience (même calorimètre, même quantité d’eau initiale, même température), mais
on introduit cette fois un glaçon de masse m’=35 g, à la température de $0^{\circ}C$. La nouvelle température est
t’=8,8°C.
Déduire des deux expériences précédentes :
  \begin{enumerate}
    \item  La chaleur latente de fusion Lf de la glace ;
    \item La capacité thermique massique Cs de la glace.
    \item On introduit un nouveau glaçon, de masse 43 g, à la température -25,5°C, dans l’eau du calorimètre à la
température t’ issue de la dernière expérience.
      \begin{enumerate}
      \item Quelle est la température atteinte à l’équilibre thermique ?
\item Reste-t-il de la glace ? Si oui, quelle est sa masse ?
\\Donnée : \\Capacité thermique massique de l’eau : $C_e = 4,19 kJ.kg^{-1}.K^{-1}$
      \end{enumerate}
  \end{enumerate}
\end{Box2}




%%_________________________Exercice 4 : _________________________Exercice


\begin{Box2}{Exercice 4 : }
On place 200 mL de solution d’acide chlorhydrique de concentration 0,4 mol/L dans un vase de Dewar de
capacité thermique $\mu = 150 J.K^{-1}$.
Une solution aqueuse d’hydroxyde de sodium, de concentration 1 mol/L, est versée progressivement dans
la solution chlorhydrique, tandis qu’on relève, après chaque addition, la température dans le calorimètre.
Initialement, les solutions d’acide chlorhydrique et d’hydroxyde de sodium sont à la même température t1 = 16,1°C. La température du calorimètre s’élève régulièrement jusqu’à t2 = 19,5°C, puis décroît lentement.
  \begin{enumerate}
\item    Ecrire l’équation bilan de la réaction qui se produit dans le calorimètre et interpréter qualitativement les
phénomènes physiques observés. Pour quel volume V de solution d’hydroxyde de sodium versé observe-ton la température maximale t2 ?
\item En déduire la chaleur de la réaction entre une mole d’ions $H3O^+$et une mole d’ions $OH^-$.
\item  Quelle est la température t3 lorsque l’on a versé 150 mL de solution d’hydroxyde de sodium ?
\\Données :
\\- Les capacités thermiques massiques des solutions d’acide chlorhydrique et d’hydroxyde de sodium sont
égales : $C = 4,2 kJ.kg^{-1}K^{-1}$
\\- Les masses volumiques de ces solutions sont égales : $\rho = 103 kg/m^3$
  \end{enumerate}
\end{Box2}




%%_________________________Exercice 6 : _________________________Exercice
\end{document}
