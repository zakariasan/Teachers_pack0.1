\documentclass[12pt, french]{article}

\usepackage{fancyhdr, fancybox, lastpage}
\usepackage[most]{tcolorbox}
\usepackage[a4paper, margin={0.3in, .75in}]{geometry}

\pagestyle{fancy}
\renewcommand\headrulewidth{1pt}
\renewcommand\footrulewidth{1pt}
\fancyhf{}
\rhead{ \em{Zakaria Haouzan}}
\lhead[C]{\em{ 1ére Année Baccalauréat Sciences Expérimentales
}}
\chead[C]{}
\rfoot[C]{}
\lfoot[R]{}
\cfoot[]{\em{Page \thepage / \pageref{LastPage}}}


\newtcolorbox{Box2}[2][]{
                lower separated=false,
                colback=white,
colframe=white!20!black,fonttitle=\bfseries,
colbacktitle=white!30!gray,
coltitle=black,
enhanced,
attach boxed title to top left={yshift=-0.1in,xshift=0.15in},
title=#2,#1}


\begin{document}
\begin{center}
   \shadowbox {\bf{LA CONCENTRATION ET SOLUTIONS ÉLECTROLYTIQUES}}
\end{center}


%%_________________________Exercice ! :"_________________________Exercice
   \begin{Box2}{Exercice 1 : }
      1. Donner la formule statique et la formules ionique des composés ioniques suivants :
      \begin{itemize}
         \item  Sulfates de cuivre 2 ($SO_4^{2-}, Cu^{2+}$)
         \item  Chlorure de potassium ($Cl^-,K^+$)
         \item  Chlorure de baryum ($Cl^-, Ba^{2+}$)
         \item  Hydroxyde de sodium ($OH^-, Na^+$)
         \item  Carbonate de sodium ($CO_3^{2-}, Na^+$)
      \end{itemize}
      2. Indiquer le nom et les ions présentés dans les composés ioniques dont les formules statiques sont  : $FeCl_2 - Na_2SO_4 - AlF_3 - Fe_2(CO_3)_3 - CuCl_2 - ZnCl_2 - ZnSO_4 - AlF_3 - Fe(NO_3)_3 $
      \\
 3. Ecrire les équations de dissolutions des composés ioniques précédents ( question 2)

   \end{Box2}


%%_________________________Exercice !2 :"_________________________Exercice
\begin{Box2}{Exercice 2 : }
On désir préparer un volume V = 250ml d’une solution de sulfate d’aluminium , de concentration massique $C_m = 17, 12g/l$ .

1. Quelle masse de sulfate d’aluminium doit on utiliser ?\\
2. Quelle est la concentration molaire C de la solution obtenue , en sulfate d’aluminium ?\\
3. Quelles sont les concentrations molaires de cette solution en anion sulfate et en cation
aluminium ?\\
Données :\\
masses molaires des éléments :$ M(Al) = 27g/mol , M(O) = 16g/mol , M(S) = 32g/mol$
\end{Box2}

%%_________________________Exercice ! 3:"_________________________Exercice
\begin{Box2}{Exercice 3 :}
1. On fait dissoudre $m = 51,3 g$ de sulfate d'aluminium $Al_2(SO_4)_3$ (composé ionique) dans $500 mL$ d'eau.\\
1.1. Préciser les 3 étapes de cette dissolution.\\
1.2. Ecrire l'équation de dissolution.\\
1.3. Calculer la concentration de soluté apporté.\\
1.4. Calculer la concentration molaire volumique de chaque espèce d'ions dans la solution.\\
Données :Masses molaires atomiques : M(Al) = 27g/mol , M(S) = 32g/mol , M(O)=16g/mol.
\\2. A partir de la solution précédente, on veut préparer V' $= 100 mL$ de sulfate d'aluminium de concentration C'$= 0,15 mol / L.$
Préciser la façon d'opérer (quelques calculs et certains appareils sont nécessaires).

\end{Box2}

%%_________________________Exercice 4 : _________________________Exercice
\begin{Box2}{Exercice 4 : }
   L'eau, pour être potable, doit avoir une concentration maximum de nitrate ($NO^{3-}$) de$1,6.10^{-4} mol/L.$

L'analyse d'une eau minérale montre qu'elle contient $0,12 mg$ de nitrate dans $200 ml.$\\
1. Effectuer les calculs permettant de montrer si cette eau est potable.\\
2. Que devient cette concentration si l'on ajoute 50 mL d'eau distillée aux 200 mL ?
\end{Box2}

\vspace{2cm}
\begin{center}
   \Large{ \em{Exercices Supplémentaires}}
\end{center}
%%_________________________Exercice 5 : _________________________Exercice
\begin{Box2}{Exercice 5 : }
On dispose de trois solution A,B,et C de même concentration molaire en soluté apporté , 
$0, 100mol/l.$ A est un solution de chlorure de sodium , B un solution de sulfate de sodium et C
une solution de chlorure de zinc .\\
1. Écrire les équations de dissolution dans l’eau qu’ont eu lieu lors de la préparation de
chacune des solution .\\
2. On mélange un volume $V_A = 100ml$ de solution A , un volume $V_B = 50ml$ de solution B
et un volume $V_C = 50ml$ de solution C . Aucun transformation chimique ne se produit .\\
a. Quelles sont les espèces chimiques présentes dans la solution (S) obtenue ?\\
b. Déterminer leurs concentrations effectives .\\
\end{Box2}

%%_________________________Exercice 6 : _________________________Exercice

\begin{Box2}{Exercice 6 :}
Le sel de Mohr est un solide de formule $FeSO_4, (NH_4)_2SO_4, 6H_2O$. On souhaite préparer une solution $S_0$ de sel de Mohr de volume $V_0=200,0 mL$de concentration molaire apportée $C_0 = 1,50 . 10 mol / L.$
On dilue ensuite cette solution pour obtenir un volume $V_1 = 100,0 mL$ de solution $S_1$ dans laquelle la
   concentration massique des ions fer II est égale à $C_{m1}=0,209 g / L.$
1. Calculer la masse molaire du sel de Mohr.\\
2. Ecrire l’équation de la dissolution dans l’eau et préciser le nom des ions.\\
3. Indiquer les tests chimiques permettant de mettre en évidence, dans cette solution, le cation métallique et
l’anion.\\
4. Décrire soigneusement la préparation de la solution $S_0$.\\
5. Quelles sont les concentrations molaires effectives de tous les ions présents dans la solution $S_0$.\\
   6. Quelle est la concentration massique des ions der II dans la solution $S_0$ ? Indiquer succinctement le mode
opératoire pour obtenir la solution S1.

\end{Box2}


%%_________________________Exercice 7 : _________________________Exercice

\begin{Box2}{Exercice 7 : }
Un flacon de déboucheur pour évier porte les indications suivantes :
   \begin{itemize}
      \item Produit corrosif.
      \item Contient de l’hydroxyde de sodium (soude caustique).
      \item d=1,2
      \item Solution à 20\%.
   \end{itemize}
Le pourcentage indiqué représente le pourcentage massique d’hydroxyde de sodium (NaOH) contenu dans
le produit.\\
1. Calculer la masse d’hydroxyde de sodium contenu dans $500 mL$ de produit.\\
2. En déduire la concentration $C_0$ en soluté hydroxyde de sodium de la solution commerciale.\\
3. On désire préparer un volume $V_1$ de solution $S_1$ de déboucheur 20 fois moins concentré que la solution commerciale.\\
3.a. Quelle est la valeur de la concentration $C_1$ de la solution ?\\
3.b. Quelle est la quantité de matière d’hydroxyde de sodium contenu dans $250 mL$ de solution $S_1$?\\
3.c. Quel volume de solution commerciale a-t-il fallu prélever pour avoir cette quantité de matière
d’hydroxyde de sodium ?\\
\end{Box2}

%%_________________________Exercice 8 : _________________________Exercice
\end{document}
