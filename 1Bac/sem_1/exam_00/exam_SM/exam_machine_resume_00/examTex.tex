\documentclass[12pt]{article}
\usepackage[a4paper, margin=.30in]{geometry}
%\usepackage{array}
\usepackage{graphicx, subfig, wrapfig, makecell }
\newcommand\headerMe[2]{\noindent{}#1\hfill#2}
\renewcommand \thesection{\Roman{section}}

\newcolumntype{M}[1]{>{\raggedright}m{#1}}




\begin{document}

\headerMe{Royaume du Maroc}{année scolaire \emph{2021-2022}}\\
\headerMe{Ministère de l'Éducation nationale, }{  Professeur :\emph{Zakaria Haouzan}}\\
\headerMe{du Préscolaire et des Sports}{Établissement : \emph{Lycée SKHOR qualifiant}}\\

\begin{center}
Devoir surveillé N°1 \\
1BAC Sciences Mathématiques\\
Durée 2h00
\\
    \vspace{.2cm}
\hrulefill
\Large{Fiche Pédagogique}
\hrulefill\\
\end{center}
%end Headerss------------------------


%__________________Chimie ______________________-
%%%%%%%+_+_+_+_+_+_+_+_+_Partie1
\section[A]{Introduction }
\hspace{0.5cm}Le programme d'études de la matière physique chimie vise à croître un ensemble de compétences visant à développer la personnalité de l'apprenant. Ces compétences peuvent être classées en Compétences transversales communes et Compétences qualitatives associées aux différentes parties du programme.
\section{cadre de référence }
 \hspace{0.5cm}L'épreuve a été réalisée en adoptant des modes proches à des situations d'apprentissages et des situations problèmes, qui permettent de compléter les connaissances et les compétences contenues dans les instructions pédagogiques et dans le programme de la matière physique chimie et aussi dans le cadre de référence de l'examen national. 
 \\Tout en respectant les rapports d'importance précisés dans les tableaux suivants :
 \begin{center}
\begin{tabular}{|c||c||c|}
\hline
    \textbf{Restitution des Connaissances} & \textbf{Application des Connaissances} & \textbf{Situation Problème }\\
    \hline
    $50\%$ & $25\%$ & $25\%$\\
    \hline
\end{tabular} 
\end{center}

 \begin{center}
     \begin{tabular}{|M{11cm}||c|}
\hline
    \thead{ \textbf{Eléments du programme et Objectifs}} &\thead{ \textbf{ Pourcentage de l'élément}} \\
    \hline
    \bf{1.Mouvement de rotation d'un corps solide non déformable
autour d'un axe fixe.} &     
    \\{  -Définir le mouvement de rotation   }&
    \\ \makecell[l]{ -Trouver le paramètre de point d'un corps solide en rotation \\autour d'un axe fixe     }&\makecell[c]{30\%}

         \\ -{ Connaître l'expression de la vitesse angulaire et son unité  }&
         \\ -{ Connaître la relation entre la vitesse angulaire et la vitesse linéaire d'un point sur le corps }&
         \\ -{ Connaître les caractéristiques du mouvement de rotation uniforme} &
         \\ - {Exploiter les équations de rotation uniforme}&
    \\\hline
    \bf{2.Travail et puissance d'une force.} &\\
         -{Connaître les effets mécaniques sur un corps solide soumis à des forces dont les points d'influence sont transmis } &\\
        -{   Connaître l'expression de travail d'une force appliquée à un objet en déplacement.} &{45\%}\\
        -{  Connaître le travail moteur et travail résistant  }&\\
        -{  Connaître l'expression de travail de poids d'un corps dans un champ gravitationnel  }&\\
        -{  Connaitre que le travail de poids est indépendant du chemin suivi  }&\\
        -{  Connaître l'expression du travail d'une force dont le couple est constant  }&\\
        -{    Connaître l'expression de travail d'une force double dont le couple est constant}&\\
        -{   Connaître l'expression de la puissance moyenne et de la puissance instantanée d'une force ou d'un groupe de forces dans le cas du déplacement rectiligne et dans le cas de rotation. }&\\
        -{   Connaître l'unité de puissance }&\\
    \hline
    \bf{3. Importance de la mesure des quantités de matière dans la vie courante. }&\\
        -{Connaître quelques techniques de mesure }&{5\%}\\
     \hline
     \bf{4. Les grandeurs physiques liées aux quantités de matière. }&\\
        - {Choisir le matériel de laboratoire en fonction d'un objectif précis et l'utiliser correctement.}&{20\%}\\
         -{ Connaître l'utilisation des documents pour connaître les dangers des matériaux utilisés. }&\\
         -{  Connaître le modèle de gaz parfait et son équation}&\\
         -{  Détermination de la quantité de matière dans un solide en fonction de sa masse}&\\
\hline
\end{tabular} 
\end{center}
\vspace{4cm}
\section{tableau de spécification}
 \begin{center}
\begin{tabular}{|c||c|c|c|c|}
\hline
    niveau d'habileté & \makecell{Restitution \\des Connaissances} &\makecell{Application \\des Connaissances} & \makecell{Situation Problème} & la somme \\
\hline
    \makecell{Mouvement de rotation \\d'un corps solide\\ non déformable\\
autour d'un axe fixe.} & \makecell{15\%\\3pts}  & \makecell{7.5\%\\1.5pts}  &\makecell{7.5\%\\1.5pts } & \makecell{30\%\\6pts} \\\hline
    \makecell{Travail et puissance \\d'une force.}
    &\makecell{22.5\%\\4.5pts}  & \makecell{11.25\%\\2.25pts}  &\makecell{11.25\%\\2.25pts } & \makecell{45\%\\9pts} \\\hline
    \makecell{Importance de la mesure\\ des quantités de \\matière dans la \\vie courante}
&\makecell{2.5\%\\0.5pts}  & \makecell{1.25\%\\0.25pts}  &\makecell{1.25\%\\0.25pts } & \makecell{5\%\\1pt} \\\hline
    \makecell{ Les grandeurs physiques \\liées aux quantités \\de matière.}
&\makecell{10\%\\2pts}  & \makecell{5\%\\1pt}  &\makecell{5\%\\1pt } & \makecell{20\%\\4pts} \\\hline
 
&\makecell{50\%\\10pts}  & \makecell{25\%\\5pts}  &\makecell{25\%\\5pts } & \makecell{100\%\\20pts} \\\hline

\end{tabular} 
\end{center}




\end{document}
