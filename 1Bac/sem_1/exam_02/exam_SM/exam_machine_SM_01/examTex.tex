\documentclass[12pt]{article}
\usepackage[a4paper, total={7.5in, 10in}]{geometry}
%\usepackage{array}
\usepackage{graphicx, subfig, wrapfig, fancyhdr, lastpage }
\newcommand\headerMe[2]{\noindent{}#1\hfill#2}
\usepackage[mathscr]{euscript}

\pagestyle{fancy}
\fancyhf{}

\cfoot{\em{Page \thepage \hspace{1pt} / \pageref{LastPage}}}
\begin{document}

\headerMe{Royaume du Maroc}{année scolaire \emph{2021-2022}}\\
\headerMe{Ministère de l'Éducation nationale, }{  Professeur :\emph{Zakaria Haouzan}}\\
\headerMe{du Préscolaire et des Sports}{Établissement : \emph{Lycée SKHOR qualifiant}}\\

\begin{center}
Devoir surveillé N°3 \\
    1BAC Sciences Mathématiques\\
Durée 2h00
\\
    \vspace{.2cm}
\hrulefill
\Large{Chimie 6pts}
\hrulefill\\

    \emph{Les deux parties sont indépendantes}
\end{center}
%end Headerss------------------------

\section*{Partie 1 : Conductance et conductivité molaire ionique \dotfill(3pts) }
    \indent La conductance d’une solution de chlorure de sodium, de concentration $C_1 = 0,15 mol.L^{-1}$ , est$\;\;\;\;\;\;\;\;\;$ $G_1$ = $2,188.10^{-2}S$. On mesure la conductance $G_2$ d’une deuxième solution de chlorure de sodium avec le même conductimètre. On obtient $G_2 = 2,947.10^{-2} S$.

\begin{enumerate}
    \item Calculer la concentration molaire $C_2$ de cette deuxième solution. La température du laboratoire et des solutions est de 25°C.\dotfill(1pt)
    \item La constante de la cellule du conductimètre est $k = 86, 7m$ . La distance entre les électrodes de la cellule est $L = 12,0 mm$. Calculer l’aire S de chaque électrode.
    \item Calculer la conductivité $\sigma$ de la première solution.\dotfill(1pt)
    \item La conductivité molaire ionique de l’ion sodium $Na^+$ est $\lambda_{Na^+} = 50,1.10^{-4} S.m^2.mol^{-1}$ . Déterminer la conductivité molaire ionique $\lambda_{Cl^-}$ de l’ion chlorure $Cl^-$.\dotfill(1pt)
\end{enumerate}

\section*{Partie 2 : Mesure d’une conductance \dotfill(3pts)}

   Aux bornes d’une cellule plongée dans une solution de chlorure de potassium et branchée sur un
générateur alternatif, on a mesuré une tension efficace de 13,7 V et une intensité efficace de 89,3mA.

\begin{enumerate}
    \item Calculer la résistance R de la portion d’électrolyte comprise entre les électrodes.\dotfill(1pt)
    \item Calculer la conductance G en S.\dotfill(1pt)
    \item La conductivité de cette solution est de 0,512 mS/cm à 20°C. Calculer la valeur de la constante k de cellule définie par : $G=K\sigma$.\dotfill(1pt)
\end{enumerate}


%_____________________________________PHYSIque Partie 22222____________________________________________________________________________
\begin{center}
    \vspace{0cm}
\hrulefill
\Large{Physique 14pts}
\hrulefill\\
    \emph{Les deux parties sont indépendantes}
\end{center}
%end Headerss------------------------
%_________________partie 2  : gravitation universelle :)

\section*{Partie 1 :Travail et énergie interne  \dotfill(6pts)}

Choisir la proposition vraie :
On fournit 50J à un système, par travail et le système cède au milieu extérieur 100J sous forme
d’énergie thermique.
  \begin{enumerate}
    \item L’énergie reçue par le système est :  (a) W = -50J \hspace{1cm}(b)W = 50J \dotfill(1pt)
    \item l’énergie cédée par le système au milieu extérieur est : (a) Q = -100J \hspace{1cm} (b) Q = 100J\dotfill(1pt)

    \item la variation de l’énergie interne est :
        (a) $\Delta$U = -150J
          \hspace{1cm}(b) $\Delta$U = -50J
          \hspace{1cm}(c) $\Delta$U = 50J\dotfill(1pt)

      \item On considère un système énergétiquement isolé et siège des frottements\dotfill(1pt)

        \begin{enumerate}
          \item l’énergie mécanique de ce système est constante
          \item l’énergie interne de ce système ne varie pas
\item le système s’échauffe.
        \end{enumerate}
        \item Répondre vrai ou faux en justifiant votre réponse :\dotfill(2pt)

          \begin{enumerate}
            \item Le travail d’une force ne peut que faire varier l’énergie cinétique d’un système.
              \item Il est possible d’élever la température d’un corps sans chauffage.
                \item Les particules constituant un solide cristallin sont immobiles dans un réseau cristallin.
                \item Vaporiser un liquide augmente le désordre des molécules qui le constituant.
          \end{enumerate}
  \end{enumerate}


\section*{Partie 2 :Energie thermique et Transfert thermique \dotfill(8pts)}

  Un calorimètre contient $100 g$ d’eau à $18^{\circ}C$. On y verse $80 g$ d’eau à $60^{\circ}C$.
  \begin{enumerate}
    \item Quelle serait la température d'équilibre si la capacité thermique du calorimètre et de ces accessoires était
      négligeable ?\dotfill(1pt)
\item La température d’équilibre est en fait $35,9^{\circ}C$. En déduire la capacité thermique du calorimètre et de ses
accessoires.\dotfill(1pt)

\item On considère de nouveau le calorimètre qui contient $100 g$ d’eau à $18^{\circ}C$. On y plonge un morceau de
cuivre de masse $20 g$ initialement placé dans de l’eau en ébullition. La température d'équilibre s’établit à
$19,4^{\circ}C$. Calculer la capacité thermique massique du cuivre.\dotfill(1pt)

\item  On considère encore le même calorimètre contenant $100 g$ d’eau à $18^{\circ}C$. On y plonge maintenant un
morceau d’aluminium de masse $30,2 g$ à la température de $100^{\circ}C$ et de capacité thermique massique $920J.kg^{-1}K^{-1}$.
Déterminer la température d'équilibre.\dotfill(1pt)

\item  L’état initial restant le même : le calorimètre contenant $100 g$ d’eau à $18^{\circ}C$, on y introduit un glaçon de
masse $25 g$ à $0^{\circ}C$. Calculer la température d'équilibre.\dotfill(1pt)

\item  L’état initial est encore : le calorimètre contenant $100 g$ d’eau à $18^{\circ}C$, on y introduit un glaçon de masse
$25g$ provenant d’un congélateur à la température de $-18^{\circ}C$. Quelle est la température d’équilibre ?\dotfill(3pt)
\\Données :
\\- Capacité thermique massique de l’eau : $C_e = 4,19 kJ.kg^{-1}K^{-1}$.
\\- Capacité thermique massique de la glace : $C_g = 2,10.10^3 J.kg^{-1}.K^{-1}$.
      \\- Chaleur latente de fusion de la glace à $0^{\circ}$C : $L_f = 3,34.10^5 J.kg^{-1}$
      \end{enumerate}
\end{document}
