\documentclass[12pt, french]{article}

\usepackage{fancyhdr, fancybox, lastpage , xcolor, 
            enumerate,
            amsmath,fontenc, mhchem}
\usepackage[most]{tcolorbox}
\usepackage[a4paper, margin={0.3in, .75in}]{geometry}
\usepackage{multirow}

\pagestyle{fancy}
\renewcommand\headrulewidth{1pt}
\renewcommand\footrulewidth{1pt}
\fancyhf{}
\rhead{ \em{Zakaria Haouzan}}
\lhead[C]{\em{1ére année baccalauréat Sciences Mathématiques}}
\chead[C]{}
\rfoot[C]{}
\lfoot[R]{}
\cfoot[]{\em{Page \thepage / \pageref{LastPage}}}


\newtcolorbox{Box2}[2][]{
                lower separated=false,
                colback=white,
colframe=white!20!black,fonttitle=\bfseries,
colbacktitle=white!30!gray,
coltitle=black,
enhanced,
attach boxed title to top left={yshift=-0.1in,xshift=0.15in},
title=#2,#1}


\begin{document}
\begin{center}

   \shadowbox {\bf{Les Réaction d’oxydoréduction $-->$ 0X/Red }}
\end{center}


%%_________________________Exercice ! :"_________________________Exercice
   \begin{Box2}{Exercice 1 :couples oxydant / réducteur } 

Ecrire les demi-équations des couples oxydant / réducteur suivants : 
$NO_{3 (aq)}/HNO_{2 (aq)}$ ;  $Ag_2O_{(s)}/Ag_{(s)}$ ; 
      $CO_{2 (g)}/H_2C_2O_{4 (aq)}$ ; 
$ClO_{(aq)}^{3-}/Cl_{2 (g)}$ ; 
      $O_{2(g)}/H_2O_{2 (aq)}$ ; 
      $S_{(S)}/H_2S_{(aq)}$ ;
      $FeO_{4 (aq)}^{2-}/Fe_{(aq)}^{3+}$ ;
      $Fe_3O_{4 (s)}/ Fe_{(aq)}^{2+}$ ; 
$HIO_{(aq)}/I_{(aq)}^-$ ; 
      $MnO_{2 (aq)}/ Mn_{(aq)}^{2+}$  ; 
      $HgO_{(S)}/Hg_(l)$


   \end{Box2}

%%_________________________Exercice !2 :"_________________________Exercice
\begin{Box2}{Exercice 2 : grain de plomb et sulfate de cuivre
 }
   On plonge un grain de plomb (Pb) dans $V = 0,15L$ d'une solution de concentration $C = 10^{-2} mol/L$ de sulfate de cuivre
$( Cu^{2+}_{(aq)} + SO^{2-}_{4 (aq)} )$.

Au cours de la réaction, il se forme des ions plomb Pb ainsi qu'un dépôt métallique, la masse de ce dépôt obtenu à la fin de la
réaction est $m = 0,72 g$. Le grain de plomb a disparu.

   1. Identifier les couples redox mis en jeu et écrire l'équation de la réaction.

   2. Établir un tableau d'avancement du système chimique.

   3. Quel est le réactif limitant du système chimique ? En déduire l'avancement maximal de la réaction.

   4. Quelle est la concentration finale des ions plomb $Pb^{2+}$?

5. Déterminer la masse initiale du grain de plomb.

Données : M(Pb) = 207 g.mol, M(Cu) = 63,5 g.mol.

\end{Box2}

%%_________________________Exercice 3 : _________________________Exercice
\begin{Box2}{Exercice 3 :  sulfure d’hydrogène }
On fait barboter pendant quelques minutes du sulfure d’hydrogène de formule $H_2S$ dans 50 ml d’une solution de
chlorure de fer III de concentration C=0,5 mol/L. Un précipité jaune de soufre S apparaît. L’addition de la soude à
   la solution obtenue par filtration donne in précipité vert d’hydroxyde de fer II caractéristique des ions $Fe^{2+}$ .

1. Interpréter ces observations en écrivant les demi équations des réactions qui viennent d’avoir lieu.

2. Donner les deux couples redox mis en jeu dans la première réaction.

   3. Calculer le volume de $H_2S$ nécessaire pour réduire tout les ions $Fe^{2+}$ .

   4. Quelle est la concentration de la solution obtenue en ions $Fe^{2+}$.

   5. calculer la masse de soufre ( S ) formé au cours de cette réaction
\end{Box2}

%%_________________________Exercice 4 : _________________________Exercice
\begin{Box2}{Exercice 4 : grenaille métallique de zinc }
Une grenaille métallique de zinc de masse $m=0,56g$ réagit avec une solution d'acide chlorhydrique de
concentration C=5mol/L.

1. Ecrire les formules des couples mis en jeux.

   2. Ecrire les demi équations correspondantes.

   3. Etablir l'équation de la réaction d'oxydoréduction.

   4. Calculer la quantité de matière initiale de zinc .

   5.a. Quel est le volume nécessaire de la solution d'acide chlorhydrique pour faire disparaitre complètement la
grenaille de zinc?

   b.Quel est le gaz formé au cours de cette transformation?

   c. Quel est le volume du gaz dégagé à la fin de la réaction , sachant que le volume molaire $V_M=25L/mol$

   d. Décrire une méthode opératoire permettant de mesurer le volume du gaz échappé.  

\end{Box2}

%\vspace{3cm}
\begin{center}
   \Large{ \em{Exercices Supplémentaires}}
\end{center}
%%_________________________Exercice 6 : _________________________Exercice

%%_________________________Exercice 7 : _________________________Exercice

\begin{Box2}{Exercice 5 :dosage du permanganate de potassium }
 
   On prépare une solution $S_1$ de permanganate de potassium $(K^+_{(aq)} + MnO^-_{4 (aq)})$ de coloration violette en dissolvant une
   masse m de $KMnO_{4 (s)}$ dans un volume $V=100mL$ d'eau,( acidifiée par quelques gouttes d'acide sulfurique).

Pour déterminer la concentration de la solution $S_1$, on prélève à l'aide d'une pipette un volume $V_1=10mL$ de cette solution qu'on introduit dans un bécher et on lui ajoute progressivement une solution $S_2$ d'acide oxalique $H_2C_2O_4$ de
concentration $C_2=0,4mol/L$.

1. Comment s'appelle cette étude expérimentale qui a pour objet la détermination de la concentration de la solution $S_1$ ?

2. Donner le schéma du dispositif expérimental utilisé dans cette étude en nommant ses différents constituants.

3. Comment s'appelle la solution dont on doit déterminer la concentration ? et comment s'appelle la solution ajouté?

4. Ecrire l'équation de la réaction qui se produit durant cette étude sachant que:
l'acide oxalique est réducteur du couple $CO_2/H_2C_2O_4$ et l' ion permanganate est oxydant du couple $MnO_4^-/Mn^{2+}$.

5. Construire le tableau d'avancement de cette réaction et en déduire la relation d'équivalence.

6. Comment repérer l'équivalence dans cette étude?

   7. Quel est le réactif limitant avant l'équivalence et quel est celui limitant après l'équivalence?

   8. Sachant que le volume ajouté à l'équivalence est : $V_{2eq}=12,5mL$, déterminer la concentration $C_1$ de la solution $S_1$.

9. Déterminer la masse m utilisée pour préparer la solution $S_1$.

10. Pour diluer la solution $S_1$ , quel volume d'eau doit- on ajouter à $90mL$ de la solution $S_1$ pour que sa concentration devient $ C'=0.1 mol/L $?

on donne : g=10N/kg - M(K)=39,1g/mol - M(Mn)=54,9g/mol - M(O)=16g/mol
\end{Box2}

%%______________

%%_________________________Exercice ! 3:"_________________________Exercice
\begin{Box2}{Exercice 3 : Hydrogénocarbonate de sodium}
 
On ajoute une masse m=2,8g  de fer Fe à un volume V=25mL d’une solution d’acide chlorhydrique
($H^+_{(aq)} + Cl^-_{(aq)}$ ) de concentration $C=1mol/L$ , il résulte de la réaction qui se produit la formation des ions ferreux $Fe^{2+}$ et le dégagement du gaz dihydrogène $H_2$.

1. Ecrire l’équation de la réaction qui se produit et déterminer les quantités de matière initiales des réactifs.

2. Construire le tableau d’avancement et déterminer le réactif limitant.

3. Déterminer la masse du fer Fe restant à la fin de la réaction.

4. Quel est le volume de $H_2$ qui résulte de la réaction ?

   5. Déterminer la masse du fer disparu (qui a réagit) .

   6. Quelle est la masse du fer initiale qu’on devrait utiliser pour que le mélange initial soit stœchiométrique ?

On donne : M(Fe)= 56g/mol

\end{Box2}






\end{document}
