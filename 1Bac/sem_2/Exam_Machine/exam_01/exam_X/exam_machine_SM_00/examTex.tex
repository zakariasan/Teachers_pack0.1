\documentclass[12pt]{article}
\usepackage[a4paper, total={7.5in, 10in}]{geometry}
%\usepackage{array}

\usepackage[siunitx, RPvoltages]{circuitikz}
\usepackage{graphicx, subfig, wrapfig, fancyhdr, lastpage, multicol ,color}
\newcommand\headerMe[2]{\noindent{}#1\hfill#2}
\usepackage[mathscr]{euscript}

\setlength{\columnseprule}{1pt}
\def\columnseprulecolor{\color{blue}}


\pagestyle{fancy}
\fancyhf{}

\cfoot{\em{Page \thepage \hspace{1pt} / \pageref{LastPage}}}
\begin{document}

\headerMe{Royaume du Maroc}{année scolaire \emph{2021-2022}}\\
\headerMe{Ministère de l'Éducation nationale, }{  Professeur :\emph{Zakaria Haouzan}}\\
\headerMe{du Préscolaire et des Sports}{Établissement : \emph{Lycée SKHOR qualifiant}}\\

\begin{center}
Devoir Surveillé  N°2 \\
    Filière 1Bac Sciences Expérimentales\\
Durée 2h00
\\
    \vspace{.2cm}
\hrulefill
\Large{Chimie 7pts/42min}
\hrulefill\\

    %\emph{Les questions parties sont indépendantes}
\end{center}
%end Headerss------------------------
%__________________Chimie ______________________-
%%%%%%%+_+_+_+_+_+_+_+_+_Partie1

 \section*{Partie 1 :Les comprimés effervescents de Vitamine B5 \dotfill(3.5pts) }
 
 Les comprimés effervescents de Vitamine B5, contiennent acide pantothénique $C_9H_{17}NO_5$ et le
   pantothénate de sodium $NaC_9H_{16}NO_5$ est le sel de sodium de la vitamine B5 , ce dernier est
employé comme additif alimentaire.

\begin{enumerate}

  \item Écrire l’équation de dissolution de pantothénate de sodium dans l’eau. \dotfill(0.5pt)

  \item Identifier le couple acide / base mettant en jeu l’acide pantothénique et écrire la demi-équation acido-basique correspondante.\dotfill(1pt)

\item  On fait réagir une masse m = 3,00 g d’acide pantothénique avec 150 mL d’une solution
d’hydroxyde de sodium $(Na^+, HO^-)$de concentration $C=2,50.10^{-1} mol.L^{-1}$.
    \begin{enumerate}
      \item Identifier les couples acide / base mis en jeu, puis écrire l’équation de la réaction envisagée.\dotfill(1pt)

      \itemÉtablir un tableau d’avancement et déterminer l’avancement maximal de la réaction. Quel est
        le réactif limitant ?\dotfill(1pt)
    \end{enumerate}
    On donne : M(H)=1g/mol , M(C)=12g/mol , M(N)=14g/mol , M(O)=16g/mol
\end{enumerate}
\section*{Partie 2 : L’eau de javel \dotfill(3.5pts) }

L’eau de javel est une solution aqueuse d’hypochlorite de sodium de formule $(Na^+_{(aq)} + ClO^-_{(aq)} )$.
La formule chimique d’une solution aqueuse d’acide chlorhydrique $(H_3O^+,Cl^-)$
\begin{enumerate}
  \item Écrire les demi-équations électroniques des deux couples suivants : $ClO^-/Cl_2$ et $Cl_2 /Cl^-$. \dotfill(0.5pt)
    
  \item  Écrire l’équation de la réaction entre les ions chlorure et hypochlorite. \dotfill(1pt)
  
  \item Soit $250mL$ d’eau de Javel contenant une quantité de matière d’ions hypochlorite $n(ClO^-)=0,41mol$ a été mélangée avec un détartrant à base d’acide chlorhydrique dans une pièce de volume $V=3,5 m^3$ .
    \begin{enumerate}
 \item Établir le tableau d’avancement relatif à la transformation chimique précédente. On considèrera que les ions $H^+_{(aq)}$ et $Cl^-_{(aq)}$ ont été introduits en excès. \dotfill(0.75pt)

 \item Calculer la quantité de matière $n$ du gaz toxique produite. \dotfill(0.75pt)
 \item En déduire le volume V de gaz toxique dégagé à 20°C et à pression atmosphérique normale. \dotfill(0.5pt)
   \end{enumerate}
\end{enumerate}

 %_____________________________________PHYSIque Partie 22222____________________________________________________________________________
\begin{center}
    \vspace{2cm}
\hrulefill
\Large{Physique 13pts - 78min}
\hrulefill\\
    \emph{Les parties sont indépendantes}
\end{center}
%end Headerss------------------------
 \section*{Partie 1 :  Comportement globale d’un circuit électrique\dotfill(6pts)}
 On dispose d’un circuit électrique comprenant, un générateur linéaire de caractéristique $(E = 12V , r = 1\Omega )$, un conducteur ohmique de résistance $R = 10\Omega$ et un électrolyseur $(E' = 4V, r')$.L’ensemble des dipôles est en série.
\begin{enumerate}
  \item Schématiser le circuit en y incluant un ampèremètre mesurant l’intensité qui traverse le conducteur
    ohmique et un voltmètre qui mesure la tension aux bornes de l’électrolyseur.\dotfill(1pt)
  \item L’intensité de courant ne varie pas au cours de l’expérience et a une valeur de 500mA pour une
    durée de fonctionnement de 12 minutes.
    \begin{enumerate}
      \item En déduire l’énergie dissipée par effet joule par le conducteur ohmique.\dotfill(1pt)
      \item Calculer la résistance interne r' de l’électrolyseur à l’aide de l’intensité de courant I.\dotfill(1pt)
    \end{enumerate}
    \item On a changé le conducteur ohmique par un nouveau conducteur ohmique. On a maintenant une intensité de 0,35 A qui traverse le circuit.
    \begin{enumerate}
      \item Calculer la valeur de l’énergie totale produite par le générateur en 20 minutes.\dotfill(1pt)
      \item Calculer la valeur de l’énergie électrique fournie au circuit par le générateur en 20 minutes.\dotfill(1pt)
        \item Calculer la nouvelle résistance du nouveau conducteur ohmique et en déduire l’énergie dissipée
          par effet joule par l’ensemble des dipôles récepteurs de ce circuit.\dotfill(1pt)
    \end{enumerate}
\end{enumerate}

 \section*{Partie 2 : Bilan énergétique \dotfill(7pts)}

   \begin{wrapfigure}[6]{r}{0.5\textwidth}
      \vspace{-1cm}
      \begin{center}
   \begin{circuitikz}[european, voltage shift=0.5]
    \draw (0,0)
    to[american voltage source,  i=$I_0$, v=$U_0$] (0,3)
    to[short] (1,3)
    to[R=$R_{Shunt}$, i=$I_0$, v=$U_S$] (3,3)
    to[short, -o] (4,3);
    \draw (0,0) to[short, -o] (4,0);
    \draw [dashed, gray] (4,3) to [Telmech=M,n=motor] ++(0,-3);
  \end{circuitikz}
  \end{center}
\end{wrapfigure}
   On considère le montage suivant constitué :
\\-d’Un générateur de force électromotrice E et de résistance interne r et un intérrupteur .
\\-d’un moteur de force électromotrice E’=2,4V et de résistance interne $r'=2\Omega$ et d’un fil inextensible enroulé sur la poulie du moteur et auquel est suspendu à l’autre extrémité un corps de masse m=50g.
\\-d’un conducteur ohmique de résistance $R=30\Omega$.

On ferme l’intérrupteur et il passe dans le circuit un courant électrique d’intensité I=0,1A.
 \begin{enumerate} 
   \item  Déterminer la puissane $P_J$ déssipée par effet joule dans l’ensemble : ( le conduceur ohmique + le moteur).\dotfill(1pt)

    \item  Calculer la puissance utile du moteur électrique.\dotfill(1pt)

    \item  En déduire la puissance $P_e$ fournie par le générateur au reste du circuit.\dotfill(1pt)

    \item  Sachant que la puissance totale déssipée dans tout le circuit par effet joule est égale à  0,36W .
      \begin{enumerate}
        \item  Déteminer la valeur de la puissance déssipée par effet joule dans le moteur.\dotfill(1pt)

        \item  En déduire la valeur de la résistance du générateur.\dotfill(1pt)
          \end{enumerate}
\item Déterminer la valeur de la force électromotrice du générateur puis retrouver l’intensité du courant en utilisant la loi de
  pouillet.\dotfill(2pt)
 \end{enumerate}
\end{document}
