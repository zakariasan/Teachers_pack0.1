 \documentclass[12pt]{article}
\usepackage[a4paper, margin=.30in]{geometry}

\usepackage{array}
\usepackage{graphicx, subfig, wrapfig, fancyhdr, lastpage,makecell }
\newcommand\headerMe[2]{\noindent{}#1\hfill#2}
\usepackage[mathscr]{euscript}



\pagestyle{fancy}
\fancyhf{}

\rfoot{\em{Page \thepage \hspace{1pt} / \pageref{LastPage}}}
\begin{document}

\headerMe{Royaume du Maroc}{année scolaire \emph{2024-2025}}\\
\headerMe{Ministère de l'Éducation nationale, }{  Professeur :\emph{Zakaria Haouzan}}\\
\headerMe{du Préscolaire et des Sports}{Établissement : \emph{Lycée SKHOR qualifiant}}\\

\begin{center}

    \vspace{-1.5cm}
Devoir  N°1 \\
   Filière Tronc Commun Scientifique\\
Durée 2h00
\\
\hrulefill
\Large{Chimie 7pts - 70min}
\hrulefill\\

    %\emph{Les Trois parties sont indépendantes}
\end{center}
%end Headerss------------------------
 
    \vspace{-1.2cm}
    
\section*{Partie 1 :les espèces chimiques \dotfill (1.25pts) }
	
	1. Compléter le tableau suivant:\dotfill(0,75pts)
\begin{center}
\begin{tabular}{ | c | c | c | }
	\hline
	\textbf{Espèce chimique }& \textbf{test} & \textbf{résultat} \\\hline 
 Présence d’eau $H_2O$ & .................... & ....................... \\\hline  
 
Présence de glucose  & ...............+ Chauffage& ........................\\\hline 
\end{tabular}
\end{center}

2. Quelle est la différence entre espèce chimique naturelle et espèce chimique synthétique ?

\section*{Partie 2 :Extraction et analyse qualitative de l’huile essentielle de la menthe \dotfill (5,75 pts) }
\hspace{0.5cm}
La menthone est l’un des constituants de certaines espèces de menthe, dont la menthe poivrée. Son
odeur et sa saveur fraîche, analogues à celles de la menthe, en font un arôme très utilisé dans les
produits alimentaires.

\begin{wrapfigure}[15]{r}{0.32\textwidth}
	\vspace{-0.6cm}
	\begin{center}
    \includegraphics[width=0.32\textwidth]{./img/hydro.png}
	\includegraphics[width=0.2\textwidth]{./img/CCM.png}
\end{center}
\end{wrapfigure}
L’extraction de l’huile essentielle de menthe poivrée s’effectue en utilisant le montage ci-dessous :

\begin{enumerate}
	\item[I] \underline{\textbf{Première étape : Extraction de l’huile essentielle}}
		\begin{enumerate}
	\item[1.1.] donner le nom de ce montage, et donner son principe. \dotfill(0,5pts)
	\item [1.2.]Recopier les numéros des parties du montage et les nommer.\dotfill(1,25pts)
	\item [1.3.]Quel est le rôle de l’élément (5). \dotfill(0,25pts)
	\item [1.4.]Quel est le rôle des grains de pierre ponce\dotfill(0,25pts)

		\end{enumerate}
	\item[II] \underline{\textbf{Deuxième étape :séparation de deux phases. }}
		\begin{enumerate}
			\item [2.]On transvase le contenu de l’erlenmeyer dans une ampoule à décanter. On ajoute 10mL d’ un solvant
convenable pour la décantation. On agite le contenu de l’ampoule rigoureusement puis, on enlève le bouchon
de l’ampoule et on laisse décanter son contenu.

Le tableau ci-dessous donne quelques propriétés des solvants :
\begin{center}
\begin{tabular}{ | c | c | c | c | }
	\hline
							& Eau & Toléne & éthanol  \\\hline 
	Densité				    & 1        & 0,87          & 0,79\\\hline  
	Miscililité avec l’eau  &  ---- & Non miscible  & miscible\\\hline  
	Solubilité de la menthone & Peu soluble & Très soluble & Très soluble\\\hline  
\end{tabular}
\end{center}

\item[2.1]Choisir le solvant convenable pour cette extraction. Justifier.\dotfill(0.5pts)
\item[2.2]Dessiner sur votre copie l’ampoule à décanter et donner les noms des deux phases.\dotfill(0,5pts)

		\end{enumerate}
		\vspace{0.5cm}	
	\item[II] \underline{\textbf{Troisième étape : identification de l’espèce extraite.}}

%\begin{wrapfigure}[5]{r}{0.36\textwidth}
    %\vspace{-1.8cm}
    %\includegraphics[width=0.36\textwidth]{./img/CCM.png}
%\end{wrapfigure}


		\begin{enumerate}
			\item[3].On réalise une chromathgraphie sur couche mince de l’huile essentielle extraite des clous de girofle. On
dépose quatre gouttes sur la plaque chromatographique.\\\textbf{(H):}L’huile essentielle extraite des clous de girofle. ; 
\\\textbf{(E):}Eugenol commercial. ; \\\textbf{(A:)}L’acétyle eugénol.
\\\textbf{(F:)}L’huile essentielle préparé à partir de feuilles de giroflier.
\\Après révélation on a obtenu le chromatogramme ci-contre.

\item[3.1.]Est-ce que l’huile essentielle (H) extraite des girofle est pure, justifier.\dotfill(0,5pts)
\item[3.2.]Que designe les deux lignes ( C ) et (D).\dotfill(0,5pts)
\item[3.3.]Quelles sont les espèces présentes dans cette huile essentielle (H) extraite des clous de girofle?\dotfill(0,5pts)

\item[3.4.]Calculer les rapports frontaux de l’eugenol commercial et de l’L’acétyle
eugénol.\dotfill(1pts) 
		\end{enumerate}
\end{enumerate}

%__________________Chimie ______________________-
%%%%%%%+_+_+_+_+_+_+_+_+_Partie1

%_____________________________________PHYSIque Partie 22222____________________________________________________________________________
\begin{center}
    %\vspace{2cm}
\hrulefill
\Large{Physique 13pts - 36min}
\hrulefill\\
    \emph{Les deux parties sont indépendantes}
\end{center}
%end Headerss------------------------

 \section*{Partie 1 :la Gravitation universelle \dotfill(10,25 pts)}

	 I. Compléter le tableau ci-dessous :\dotfill(2,25pts)
\begin{center}
\begin{tabular}{ | c | c | c | c | }
	\hline
			Distance		& Valeur en mètre(m) & Ecriture scientifique &Ordre de grandeur  \\\hline 
Diamètre d’une cellule $259\mu m$				    & \dotfill        &\dotfill &\dotfill \\\hline  
	Epaisseur d’une feuille $0,0321cm$  & \dotfill& \dotfill  & \dotfill\\\hline  
	\makecell{Distance entre \\Rabat et Agadir $7850 Km$} & \dotfill & \dotfill &\dotfill \\\hline  
\end{tabular}
\end{center}
 \begin{enumerate}

\item[II]. Soient deux corps ponctuels A et B de masses respectives $m_A = 10Kg$ et $m_B=20Kg$
distants de : $d =10m$.

\item Enoncer la loi de gravitation universelle\dotfill(1pts)
\item Donner les caractéristiques des deux forces de gravitation universelles               $\vec{F_{A/B}}$ et $\vec{F_{B/A}}$\dotfill(2pt)
\item Représenter sur le schéma ci-contre les  $\vec{F_{A/B}}$ et $\vec{F_{B/A}}$ en utilisant une échelle adapté.\dotfill(2pt)
\item[III.] A une altitude h de la surface de la terre, l’intensité de la pesanteur $g_0$ est donnée par la formule suivante :$g = G.\frac{M_T}{(R_T + h)^2}$.

\item En déduire l’expression de l’intensité du champ de pesanteur
$g_0$ la surface de la terre $(h=0)$ en fonction de :G,$M_T$,$R_T$.

\item Déduire la relation $g=g_0.\frac{R_T^2}{(R_T + h)^2}$.\dotfill(1.5pts)
\item Montrer que lorsque $h = \sqrt{2}.R_T$ On a $P=\frac{P_0}{9}$.\dotfill(1.5pts)
 \end{enumerate}



 \section*{Partie 2 :Exemples d’actions mécaniques\dotfill(2,75 pts)}

%\begin{wrapfigure}{r}{0.36\textwidth}
    %\vspace{-1.8cm}
    %\includegraphics[width=0.36\textwidth]{./img/plan pi.png}
%\end{wrapfigure}

Un solide (S) de masse $m=100g$ est au repos sur un plan $\pi$ incliné par rapport à l’horizontale d’un angle $\alpha$ sans frottement.
\begin{enumerate}
	\item Faire le bilan des forces appliquées sur le solide (S).\dotfill(1.5pt)
	\item Représenter, sans échelle, ces forces sur le schéma ci-dessus.\dotfill(1,25pt)
\end{enumerate}











\end{document}
