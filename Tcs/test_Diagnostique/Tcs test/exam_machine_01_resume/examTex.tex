\documentclass[12pt]{article}
\usepackage[a4paper, margin=.30in]{geometry}
%\usepackage{array}
\usepackage{fancybox}

\usepackage{graphicx, subfig, wrapfig, makecell }
\usepackage{multirow}

\newcommand\headerMe[2]{\noindent{}#1\hfill#2}
\renewcommand \thesection{\Roman{section}}

\newcolumntype{M}[1]{>{\raggedright}m{#1}}




\begin{document}

\headerMe{Royaume du Maroc}{année scolaire \emph{2024-2025}}\\
\headerMe{Ministère de l'Éducation nationale, }{  Professeur :\emph{Zakaria Haouzan}}\\
\headerMe{du Préscolaire et des Sports}{Établissement : \emph{Lycée SKHOR qualifiant}}\\

\begin{center}
Evaluation Diagnostique \\
Filière Tronc Commun Scientifique\\
Durée 1h45
\\
    \vspace{.2cm}
\hrulefill
\Large{Fiche Pédagogique}
\hrulefill\\
\end{center}
%end Headerss------------------------


%__________________Chimie ______________________-
%%%%%%%+_+_+_+_+_+_+_+_+_Partie1
\section[A]{Introduction }
\hspace{0.5cm}Le programme d'études de la matière physique chimie vise à croître un ensemble de compétences visant à développer la personnalité de l'apprenant. Ces compétences peuvent être classées en Compétences transversales communes et Compétences qualitatives associées aux différentes parties du programme.
\section{cadre de référence }
 \hspace{0.5cm}
L'évaluation diagnostique s'appuie sur l'analyse des programmes des années scolaires passées et de l'année scolaire en cours pour déterminer les apprentissages fondamentaux antérieurs sur lesquels s'appuient les apprentissages ciblés au niveau actuel.

Cette évaluation diagnostique vise à atteindre les objectifs suivants :
\begin{itemize}
	\item Permettre aux enseignants de déterminer avec précision les forces et les faiblesses des acquis des élèves.
	\item Identifier les domaines prioritaires et soutenir les activités prévues.
	\item Permettre à la Direction de l'éducation de fournir des données diagnostiques exactes sur la réussite scolaire des élèves.
\end{itemize}

\section{tableau de spécification}
 \begin{center}
\begin{tabular}{||c|c|c|c|c|}
\hline
\multicolumn{2}{||c|}{\makecell{Domaine}} & \makecell{Taux d'importance du domaine} & \multicolumn{2}{|c|}{Importance du niveau de compétence }\\\cline{4-5}
\multicolumn{2}{||c|}{}  &  &\makecell{Connaissances \\49\%} &\makecell{Application\\51\%}\\\hline
\multirow{3}{*}{\makecell{Physique\\70\%}}  & Mécanique & 31\% & 16,4\% & 14,6\%\\\cline{2-5}
											& électricité & 36\% & 16,4\% & 19,6\%\\\hline
\multicolumn{2}{||c|}{\makecell{Chimie 30\%}}    & 33\% &16\% & 17\% \\\hline\hline
\multicolumn{2}{||c|}{\makecell{total}}     & 100\%     &49\% & 51\% \\\hline
	
    \hline
\end{tabular} 
\end{center}
\section*{Nombre d'indicateur pour chaque domaine}
 \begin{center}
\begin{tabular}{|c|c|c|c|c|}
\hline
\makecell{Domaine} & \makecell{Mécanique\\31\%} & \makecell{électricité \\36\%} & \makecell{Chimie \\33\%} & \makecell{Nombre d'indicateur} \\\hline

\makecell{Connaissances 49\%} & 3 & 4 & 4  & 11\\\hline
\makecell{Application 51\%}   & 4 & 5 & 4  & 13\\\hline
\makecell{Total 100\%} & 7 & 9 & 8  & 24\\\hline


\end{tabular} 
\end{center}
\begin{center}
	\shadowbox{\bf{ Evaluation Diagnostique } }
\end{center}



\newpage
\begin{center}
  \begin{tabular}{|c||c||c|}
    \hline
	\multicolumn{3}{||c||}{\bf{   \hfill  Physique 70\%  \hfill (34pts)} }\\
         \hline
         \multicolumn{3}{||c||}{\bf{Partie 1 : Mécanique et énergie \dotfill (19pts)} }\\
\hline
    \textbf{$N^{\circ}$Q } & \textbf{Réponse } & \textbf{Note }\\
    \hline
    $1$ & \makecell{(Vrai) 
La masse est une grandeur fixe elle ne
dépend pas du lieu. }  & $1pt$\\\hline
%Q2    
$2$ & \makecell{ (Vrai) 
La valeur de l’intensité du poids est une grandeur fixe elle dépond du lieu. } & $1pt$\\\hline
%Q3:
$3$ & \makecell{(Vrai) Le poids est la force exercée par la terre sur un corps.} & $1pt$\\\hline
 %Q4:
$4$ & \makecell{(Faux) La relation entre le poids et la masse est  P=m/g} & $1pt$\\\hline
 %Q5:
$5$ & \makecell{(Faux) 
Dans un mouvement de translation la trajectoire, d’un corps est une droit. } & $1pt$\\\hline
 %Q6:
$6$ & \makecell{(Vrai) Dans un mouvement rectiligne uniforme, la vitesse est constante.} & $1pt$\\\hline
 %Q7:
$7$ & \makecell{ (Faux) La valeur de la vitesse augmente dans un mouvement rectiligne retardé. } & $1pt$\\\hline
 %Q8:
$8$ & \makecell{ La vitesse de la voiture est : V(voiture) = 10 m/s\\
La vitesse de la voiture V(voiture) = 36 km/h\\
La vitesse du conducteur du camion V(camion) = 16,66 m/s\\
La vitesse du conducteur du camion V(camion) = 60 Km/h\\
Le camion a dépassé la vitesse limite
} & $5pt$\\\hline
 %Q9:
$9$ & \makecell{ L’intensité du poids est P=25N\\
	L’intensité de la force exercée par la table sur la boule est F=25N
} & $2pt$\\\hline
 %Q10:
$10$ & \makecell{F1 et F2 vérifient la relation suivante: $\vec{F_1} + \vec{F_2} = \vec{0}$ } & $3pt$\\\hline
 %Q11:
$11$ & \makecell{Son poids est P = 2N } & $1pt$\\\hline
 %Q12.a:
$12$ & \makecell{ la vitesse d’un objet est constante, le mouvement est dit Uniforme } & $1pt$\\\hline
 %Q12.b:
      %Partie 2 : -----
\multicolumn{3}{||c||}{\bf{Partie 2 : électricité \dotfill (15pts)} }\\
\hline
%1
 $1$ & \makecell{la valeur de la résistance utilisée R=0,1Ohm } & $2pt$\\\hline
%2
$2$ & \makecell{
Dans un circuit en série, quand on ajoute une résistance,\\alors l’intensité du courant diminue }& $1pt$\\\hline
%3 
		  $3$ & \makecell{
Dans les appareils de chauffage, une résistance \\permet de produire de la chaleur }& $1pt$\\\hline
%4 
$4$ & \makecell{
L’énergie consommée par un appareil \\de chauffage est donnée par :$E=R.I^2t$ }& $2pts$\\\hline
%5 
$5$ & \makecell{a relation de la puissance électrique reçue \\par un appareil en courant continu est: P=UI }& $1pt$\\\hline
%6 
$6$ & \makecell{l’intensité du courant vaut I = 0,3A }& $1pt$\\\hline
%7 
$7$ & \makecell{La tension aux bornes du conducteur ohmique est continu.\\
L’intensité du courant traversant le conducteur est I=2.5A
}& $4pts$\\\hline
%8
$8$ & \makecell{l’intensité efficace du courant traversant d’une lampe I=0,33A }& $1pt$\\\hline

$9$ & \makecell{l’énergie consommé par la lampe pour une durée(3H)  E=135Wh }& $1pt$\\\hline

$10$ & \makecell{E en J, P en W et t en s }& $1pt$\\\hline

  \hline
	\multicolumn{3}{||c||}{\bf{   \hfill  Chimie 30\%  \hfill (16pts)} }\\
         \hline
\hline
%Q1    
$1$ & \makecell{Pour savoir si un morceau de pain contient\\de l’eau, on utilise l’espèce chimique suivante :: le sulfate de cuivre anhydre } & $1pt$\\\hline
%Q2    
$2$ & \makecell{Pour obtenir simplement une eau limpide à partir\\ d’une eau boueuse, on peut réaliser : filtration } & $1pt$\\\hline
%Q3    
$3$ & \makecell{Support : N 7 , Entonnoir : N 3 , Filtrat : N 5, Mélange hétérogène N 1,\\
Filtre : N 2, Baguette:N 6
} & $6pt$\\\hline
%Q4    
$4$ & \makecell{pH = 8,2. Indique si l’eau est plutôt :Basique  } & $1pt$\\\hline
%Q5    
$5$ & \makecell{Lorsqu’on dilue une solution acide, le PH de cette solution : augmente } & $1pt$\\\hline
%Q6    
$6$ & \makecell{
La charge de l’ion Al3+ est:q=+3e } & $1pt$\\\hline
%Q7   
$7$ & \makecell{
Les matériaux organiques sont composés principalement par : le Carbone et l’hydrogène } & $1pt$\\\hline
%Q8   
$8$ & \makecell{Les constituants de l’atome sont : Les électrons et le noyau } & $1pt$\\\hline
%Q9   
$9$ & \makecell{Détermine la masse en cuivre utilisé pour fabriquer la casserole. m=2,3Kg } & $3pts$\\\hline
  \end{tabular}
  \end{center}


\end{document}
