\documentclass[12pt]{article}
\usepackage[a4paper, margin=.30in]{geometry}
%\usepackage{array}
\usepackage{fancybox}

\usepackage{graphicx, subfig, wrapfig, makecell }
\usepackage{multirow}

\newcommand\headerMe[2]{\noindent{}#1\hfill#2}
\renewcommand \thesection{\Roman{section}}

\newcolumntype{M}[1]{>{\raggedright}m{#1}}




\begin{document}

\headerMe{Royaume du Maroc}{année scolaire \emph{2024-2025}}\\
\headerMe{Ministère de l'Éducation nationale, }{  Professeur :\emph{Zakaria Haouzan}}\\
\headerMe{du Préscolaire et des Sports}{Établissement : \emph{Lycée SKHOR qualifiant}}\\

\begin{center}
%Evaluation Diagnostique \\
%Durée 1h45
\hrulefill
\shadowbox{\bf{Rapport de l’évaluation diagnostique }}
\hrulefill\\
\end{center}
%end Headerss------------------------


%__________________Chimie ______________________-
%%%%%%%+_+_+_+_+_+_+_+_+_Partie1
\section[A]{Introduction }
\hspace{0.5cm} Pour constituer une vision globale sur l’état d’avancement des apprenants en Physique chimie, et pour établir le profil de l’ensemble de la classe, une évaluation diagnostique s’impose au début de chaque année scolaire.

Il s’agit de situer les apprenants par rapport aux apprentissages prévus dans le nouveau programme, de détecter leurs pré-requis et pré-acquis, de repérer leurs difficultés d’apprentissage et de déterminer leurs savoirs, savoir-faire et savoir-être relatifs aux différentes compétences requises pour entamer et s’entreprendre les nouveaux apprentissages.

L’évaluation diagnostique en matière Physique chimie pour le niveau 2Bac-SM (2ème année baccalauréat Sciences Mathématiques) , visera les disciplines suivantes : en Physique 3 partie Mécanique , électrodynamique , Optique et Chimie générale .
\section{ Objectifs de l’évaluation diagnostique :}
\begin{itemize}
	\item Etre capable de déterminer les points forts et les points faibles dans les apprentissages antérieurs des apprenants.
	\item Déterminer les difficultés et les obstacles d’apprentissage et motiver les apprenants à les surmonter

	\item Investir les résultats de l’évaluation diagnostique pour planifier les activités de soutien
	\item Adopter ces résultats pour l’orientation et le conseil 

\end{itemize}

\section{Informations générales sur l’évaluation diagnostique :  }
Le tableau ci-dessous et la fiche pédagogique résume la planification de l’évaluation diagnostique et le nombre des apprenants présents :
\begin{center}
  \begin{tabular}{|c|c|c|}
	  \hline
	  La classe & Date de la procédure & Nombre des apprenants présents\\\hline
 Tronc Commun Scientifique (Tcs-2) & 19/09/2024 & 28\\\hline
\end{tabular}
\end{center}

Cette évaluation inclus diverses questions : Questions vrai ou faux et Questions à choix multiples (QCM) dans des Situations problèmes.
\section{Analyse des résultats de l’évaluation diagnostique : }

\begin{center}
  \begin{tabular}{|c|c|c|c|}
	  \hline
	  La classe & \makecell{Les élèves de niveau faible \\ 0,16 } & \makecell{Les élèves moyens \\ ]16,34]}& \makecell{Les élèves brillants \\  ]34,50]} \\\hline
Tronc Commun Scientifique (Tcs 1) & \makecell{18 élèves \\ 64\%} & \makecell{10 élèves \\ 36\%}& \makecell{0 élève \\ 0\%} \\\hline
\end{tabular}
\end{center}

Après que les apprenants aient passé ce test, on peut dire que les résultats obtenus sont en général positifs, mais certaines observations doivent être mentionnées, telles que :

\begin{itemize}
	\item La plupart des apprenants confondent entre les grandeurs physiques et leurs unités dans le système international
	\item Tous les apprenants ne pouvaient pas équilibrer les équations des réactions chimiques

	\item La plupart des apprenants ont oublié les concepts liés à la matière (constitution de l’atome, les ions, …)
	\item Certains apprenants ont oublié : comment relier le voltmètre et l’ampèremètre pour mesurer la tension et l’intensité de courant,  la loi d’ohm, l’expression de la puissance électrique
\end{itemize}

\section*{Conclusion : }
\hspace{2cm} Suite à la séance d’évaluation diagnostique, nous avons pu observer que les élèves ont montré une motivation notable à s’engager dans le processus d’apprentissage. Ce test a permis de dresser un premier bilan sur les acquis et les lacunes de la classe en physique-chimie. Bien que certains élèves aient eu des difficultés à répondre aux questions, ils se sont prêts à surmonter leurs faiblesses. Le défi pour les prochaines étapes sera de soutenir efficacement ces élèves pour qu'ils puissent suivre le rythme de la classe.

Pour combler les lacunes constatées et améliorer la compréhension globale, nous proposons les actions suivantes :

\begin{itemize}
  \item \textbf{Mise en place de séances de soutien ciblées} : Ces séances permettront de retravailler les notions non assimilées en physique et en chimie, en adoptant une approche progressive adaptée aux besoins de chaque élève.

  \item \textbf{Évaluations formatives au début de chaque chapitre} : Elles serviront à vérifier la compréhension des prérequis avant de poursuivre avec de nouvelles notions, afin d’éviter l’accumulation des difficultés.

\end{itemize}
Cette évaluation diagnostique a donc été une étape essentielle pour identifier les points à améliorer et orienter notre approche pédagogique. Grâce à ces ajustements, nous espérons accompagner efficacement chaque élève vers une meilleure maîtrise des notions du programme et garantir une progression harmonieuse pour l’ensemble de la classe.

 \vspace{5cm}
 SIGNATURE DU PROFESSEUR \hspace{3cm} DIRECTEUR  \hspace{3cm} INSPECTEUR

 \vspace{6cm}
\emph{Pièces jointes : Evaluation Diagnostique + Fiches pédagogiques.}
\end{document}
