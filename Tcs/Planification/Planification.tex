\documentclass[12pt]{article}
\usepackage[a4paper, margin=.20in]{geometry}
%\usepackage{array}
\usepackage{graphicx, subfig, wrapfig, makecell,fancyhdr,xcolor }
\newcommand\headerMe[2]{\noindent{}#1\hfill#2}
\renewcommand \thesection{\Roman{section}}

\newcolumntype{M}[1]{>{\raggedright}m{#1}}

\chead{\includegraphics[width = 0.1\textwidth]{./img/logoMin.png}}
\cfoot{helllo}

\begin{document}

\begin{center}
\includegraphics[width = 0.18\textwidth]{./img/logoMin.png}
\vspace{-3cm}
\end{center}
\headerMe{Matière : Physique-Chimie}{Établissement : \emph{Lycée SKHOR qualifiant}}\\
\headerMe{ Niveau : Tronc Commun scientifique }{  Professeur :\emph{Zakaria Haouzan}}\\
%\headerMe{Niveau : 2BAC-SM-X}{Heure : 5H}\\

\begin{center}
	\vspace{0.5cm}
\underline{Tronc Commun scientifique }\\
    %\vspace{.2cm}
\hrulefill
\Large{Planification annuelle
du programme de la matière chimie physique}
\hrulefill\\
\end{center}
%end Headerss------------------------


%__________________Chimie ______________________-
%%%%%%%+_+_+_+_+_+_+_+_+_Partie1
\begin{center}
	 \begin{tabular}{||p{0.15\textwidth}||p{0.5\textwidth}||p{0.1\textwidth}||p{0.1\textwidth}|}
\hline
 %\multicolumn{4}{|c|}{Déroulement}\\\hline

\makecell{La période de\\réalisation} & \makecell{Le contenu
de programme } & \multicolumn{2}{|c|}{L’enveloppe horaire }  \\\hline


\makecell{
\color{red}{Semaine 1}\\De 09-09-2024\\à 14-09-2024
\\\color{red}{Semaine 2}\\De 16-09-2024\\à 21-09-2024
\\\color{red}{Semaine 3}\\De 23-09-2024\\à 28-09-2024
}	  &
\makecell{Pendant cette période, nous réalisons\\-le contrat didactique
\\- Révision générale
\\- Examens diagnostiques
\\- Soutien pédagogique
} & 12H & \\\hline\hline

% sem 1% ====================================== 

\makecell{
\color{red}{Semaine 1}\\De 30-09-2024\\à 05-10-2024}&
\makecell{\bf{Les} espèces chimiques } &2H&\\\cline{2-3}
									   &
\makecell{\bf{Extraction}, séparation et identification\\d'espèces chimiques. }&3H&\\\hline\hline

% sem 2: =================================== 

\makecell{
\color{red}{Semaine 2}\\De 07-10-2024\\à 12-10-2024}&
\makecell{\bf{Synthèse} des espèces chimiques }&2H&\\\cline{2-3}
																							&\makecell{Exercices: La chimie
autours de nous } &2H&\\\hline\hline

%sem 3 ======================================

\makecell{
\color{red}{Semaine 3}\\De 14-10-2024\\à 19-10-2024
}&

\makecell{\bf{La} Gravitation universelle}
&4H&\\\hline
\hline
% sem 4: ======================================

\makecell{
\color{red}{Semaine 4}\\De 20-10-2024\\à 27-10-2024}&
\makecell{Vacances d'automne}& 8 jours&\\\hline\hline


%sem 5: ====================================== 

\makecell{
\color{red}{Semaine 5}\\De 28-10-2024\\à 02-11-2024
}&
	\makecell{\bf{Exemples} d'actions mécaniques  } &2H& \\\cline{2-3}
&
	\makecell{Exercices: Interactions mécaniques }&1H&\\\hline\hline

% sem 6: ======================================
\makecell{
\color{red}{Semaine 6}\\De 04-11-2024\\à 09-11-2024
}	  &
\makecell{\bf{ Mouvement} }&2H&\\\cline{2-3}
						   &
\makecell{\bf{Devoir} $N^{\circ}1$ \emph{Semestre $N^{\circ}1$}} &2H&\\\hline\hline
	
% sem 7 ========================================
\makecell{
\color{red}{Semaine 7}\\De 11-11-2024\\à 16-11-2024
}	&
	\makecell{\bf{Mouvement}}&4H& \\\hline
	% sem 08 ======================================
\makecell{
\color{red}{Semaine 8}\\De 18-11-2024\\à 23-11-2024
}	&
	\makecell{Exercices: Mouvement} &1H&\\\cline{2-3}
& corriger le Devoir N 1 & 1H&\\\cline{2-3}
&\makecell{\bf{Principe d'inertie.} } &2H& \\\hline

\end{tabular}\end{center}

%%%%%%%+_+_+_+_+_+_+_+_+_Partie1
\begin{center}
	 \begin{tabular}{||p{0.15\textwidth}||p{0.5\textwidth}||p{0.1\textwidth}||p{0.1\textwidth}|}
\hline

\makecell{La période de\\réalisation} & \makecell{Le contenu
de programme } & \multicolumn{2}{|c|}{L’enveloppe horaire }  \\\hline

%sem09 =======================================
\makecell{
\color{red}{Semaine 9}\\De 25-11-2024\\à 30-11-2024} 
& \makecell{\bf{Principe d’inertie.} \\ Exercices: Principe d’inertie.} & 4H&\\\hline\hline 

% sem 10 ========================================

\makecell{
\color{red}{Semaine 10}\\De 02-12-2024\\à 07-12-2024} 
&\makecell{\bf{Modèle de l'atome}}&4H&\\\hline\hline

% sem 11: ======================================

\makecell{
\color{red}{Semaine 11}\\De 09-12-2024\\à 14-12-2024}&
\makecell{Vacances d'automne}& 8 jours&\\\hline\hline


% sem 12 =========================================
\makecell{
\color{red}{Semaine 12}\\De 15-12-2024\\à 21-12-2024} 
&\makecell{Révision} &2H&\\\cline{2-3}
&\makecell{\bf{Devoir} $N^{\circ}2$ \emph{Semestre $N^{\circ}1$}} &2H&\\\hline\hline

% sem 13 ============================================

\makecell{
\color{red}{Semaine 13}\\De 23-12-2024\\à 28-12-2024} 
&\makecell{\bf{Géométrie de quelques molécules} }&4H&\\\hline\hline

% sem 14 ===========================================

\makecell{
\color{red}{Semaine 14}\\De 30-12-2024\\à 04-01-2025}
&\makecell{\bf{Classification périodique des éléments chimiques.}}&2H&\\\cline{2-3}
&\makecell{Exercices: Géométrie de quelques molécules }&1H&\\\cline{2-3}
&\makecell{ corriger le Devoir N 2  }&1H&\\\hline
\hline

% sem 15 ===============================================

\makecell{
\color{red}{Semaine 15}\\De 06-01-2025\\à 11-01-2025}
 &
	\makecell{Exercices: Classification périodique des éléments\\chimiques.}&1H&\\\cline{2-3}
																			&\makecell{\bf{Quelques} applications de l’équilibre d’un solide\\soumis à deux forces}&3H&\\\hline\hline

% sem 16 ===============================================

\makecell{
\color{red}{Semaine 16}\\De 13-01-2025\\à 18-01-2025}
&\makecell{\bf{Quelques} applications de l’équilibre d’un solide\\soumis à deux forces   }&2H&\\\cline{2-3}
&\makecell{Exercices:Quelques applications de l’équilibre\\d’un solide
soumis à deux forces }&1H&\\\cline{2-3}
&\makecell{Révision}&1H&\\\hline\hline


% sem 17 ================================================

\makecell{
\color{red}{Semaine 17}\\De 20-01-2025\\à 25-01-2025}
&\makecell{\bf{Devoir} $N^{\circ}3$ \emph{Semestre $N^{\circ}1$}} &2H&\\\cline{2-3}
&\makecell{\bf{Equilibre} d'un corps solide soumis à l'action\\de trois
forces} &1H&\\\cline{2-3}
&\makecell{ corriger le Devoir N 3  }&1H&\\\hline

%\end{tabular}\end{center}

%%__________________Chimie ______________________-
%%%%%%%%+_+_+_+_+_+_+_+_+_Partie1
%\begin{center}
	 %\begin{tabular}{||p{0.15\textwidth}||p{0.5\textwidth}||p{0.1\textwidth}||p{0.1\textwidth}|}
%\hline
 %%\multicolumn{4}{|c|}{Déroulement}\\\hline


%% sem 19

%\makecell{
%\color{red}{Semaine 20}\\De 16-01-2023\\à 22-01-2023}
%&\makecell{\bf{Le circuit} RLC série en régime sinusoïdal forcé} &6H&\\\cline{2-3}


\end{tabular}
\end{center}


\end{document}
