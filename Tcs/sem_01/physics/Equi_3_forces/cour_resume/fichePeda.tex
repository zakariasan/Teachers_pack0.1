\documentclass[12pt]{article}
\usepackage[a4paper, margin=.20in]{geometry}
%\usepackage{array}
\usepackage{graphicx, subfig, wrapfig, makecell,fancyhdr,xcolor }
\newcommand\headerMe[2]{\noindent{}#1\hfill#2}
\renewcommand \thesection{\Roman{section}}

\newcolumntype{M}[1]{>{\raggedright}m{#1}}

\chead{\includegraphics[width = 0.1\textwidth]{./img/logoMin.png}}
\cfoot{helllo}

\begin{document}

\begin{center}
\includegraphics[width = 0.18\textwidth]{./img/logoMin.png}
\vspace{-3cm}
\end{center}
\headerMe{Matière : Physique-Chimie}{Établissement : \emph{Lycée SKHOR qualifiant}}\\
\headerMe{ Unité : Mécanique }{  Professeur :\emph{Zakaria Haouzan}}\\
\headerMe{Niveau : TCS}{Heure : 6H}\\

\begin{center}
	%\vspace{1cm}
\underline{Leçon N°4.2: Equilibre d'un corps solide soumis à l'action de trois forces.}\\
Durée 6h00
\\
    \vspace{.2cm}
\hrulefill
\Large{Fiche Pédagogique}
\hrulefill\\
\end{center}
%end Headerss------------------------


%__________________Chimie ______________________-
%%%%%%%+_+_+_+_+_+_+_+_+_Partie1
 \begin{center}
	 \begin{tabular}{|p{0.2\textwidth}||p{0.3\textwidth}||p{0.3\textwidth}||p{0.1\textwidth}|}
\hline
\textbf{Prérequis} & \textbf{Compétences visées } & \textbf{Savoir et savoir-faire}&\textbf{Outils didactiques }\\
    \hline
	 -Equilibre d’un corps solide soumis à deux forces
 
	 -Poids, masse, actions
mécaniques,
				   &
- Formuler une hypothèse au sujet d'un événement potentiel ou d'un paramètre susceptible de jouer un rôle dans un phénomène.

-Proposer une expérimentation susceptible de valider ou d'invalider une hypothèse ou d'atteindre un objectif spécifique.

- Elaborer une démarche

- Faire le schéma d’une expérience

- Formuler un résultat, conclure
				   &
-Connaitre les deux conditions d’équilibre
d’un corps solide soumis à 3 forces et les
appliquer

-Utiliser la ligne polygone (méthode
géométrique), et la méthode analytique pour
déterminer les intensités de quelques forces

-Connaitre les forces de frottement et le
coefficient de frottement k 


 & Ordinateur  simulation data-show 
Corde   Dynamomètres

Morceau de carton

Supports
\\
    \hline
\end{tabular} 
\end{center}
\section*{Situation-problème :}
un alpiniste qui est en équilibre sous
l’action de 3 forces : son poids , la réaction , et la tension de la corde.

\begin{enumerate}
\item Quelles sont les conditions d’équilibre d’un corps solide
soumis à 3 forces ?
\item Comment utiliser ses conditions pour déterminer les intensités
de quelques forces, et aussi les valeurs d’autres grandeurs  ?
\end{enumerate}

\begin{center}
	 \begin{tabular}{|p{0.2\textwidth}||p{0.3\textwidth}||p{0.3\textwidth}||p{0.1\textwidth}|}
\hline
\multicolumn{4}{|c|}{Déroulement}\\\hline
Eléments du & \multicolumn{2}{c||}{Activités didactiques} &  \\\cline{2-3}
cours & Enseignant & Apprenant & Evaluation\\\hline

\color{red}{I-Introduction}	  &
-Le professeur pose la situation-problème.

-Demande aux apprenants de répondre aux questions de la situation-problème.

-Ecrire les hypothèses proposées par les apprenants.

-Garde les hypothèses convenues pour vérifier pendant
du cours.
				  &
				  -L’apprenant analyse la situation déclenchante
et formule des hypothèses.

				  &
				  Evaluation
diagnostique\\\hline


%II- partie 2 
\color{red}{II Conditions d’équilibre d’un corps solide sous
l’action de trois forces non parallèles : }

\vspace{0.5cm}
\color{blue}1 Etude de l’équilibre d’un solide soumis à trois forces
non parallèles

\vspace{0.5cm}

\color{blue}2Activité expérimentale N°1
\vspace{0.5cm}


				  &
-Le professeur pose les questions de l'Activité 1 : 


				  &
				  --- L’apprenant Travail en groupes , Répondre aux
questions , Atteindre les deux
 , conditions d’équilibre

--- L’apprenant conclut que  Lorsqu’un corps solide est en équilibre sous l’action de trois forces non parallèles, alors :

\vspace{0.5cm}

--- La ligne polygone de ces trois forces est fermée, c-à-d sa somme vectorielle est nulle

---Les lignes d’action de ces trois forces sont coplanaires et concourantes

\vspace{0.2cm}
				  & 
	Évaluation formative \\\hline			  

\color{red}{III Application : méthode géométrique, méthode analytique }

\vspace{0.5cm}
\color{blue}1 Equilibre d’un solide sur un plan incliné: cas d’un contact sans frottement

\vspace{0.5cm}

\color{blue}2Activité expérimentale N°2
\vspace{0.5cm}


				  &
-Le professeur pose les questions de l'Activité 2 : 


				  &
				  --- L’apprenant Travail en groupes , Répondre aux
questions , Atteindre les deux
 , conditions d’équilibre

--- L’apprenant conclut que  Lorsqu’un corps solide est en équilibre sous l’action de trois forces non parallèles, alors :

\vspace{0.5cm}

--- La ligne polygone de ces trois forces est fermée, c-à-d sa somme vectorielle est nulle

---Les lignes d’action de ces trois forces sont coplanaires et concourantes

\vspace{0.2cm}
				  & 
	Évaluation formative \\\hline			  



\end{tabular}
\end{center}

%Page 3 in parte 2

\begin{center}
	 \begin{tabular}{|p{0.2\textwidth}||p{0.3\textwidth}||p{0.3\textwidth}||p{0.1\textwidth}|}
\hline
\multicolumn{4}{|c|}{Déroulement}\\\hline
Eléments du & \multicolumn{2}{c||}{Activités didactiques} &  \\\cline{2-3}
cours & Enseignant & Apprenant & Evaluation\\\hline

%II- partie 2 
\color{red}{II Ondes longitudinales, transversales, et leurs caractéristiques.}

\vspace{0.5cm}
\color{blue}4
Les Ondes sonores :

\vspace{0.5cm}
\color{blue}5 Vitesse de propagation d’une onde :
				  &
Activité : 

on allume le téléphone, puis on vide la cloche de l’air par une pompe.

Exp2 : on frappe le diapason 
\vspace{0.2cm}

Exploitation: 

a-Dire ce qui arrive au son émis par le
téléphone lorsqu’on vide de l’air ? Que
concluez-vous ?

b-Dire ce qui arrive à la balle après avoir frappé
le diapason ? Conclure la nature de l’onde
sonore ?
\vspace{0.5cm}

-Le professeur pose la question suivante : 

\emph{Qu’est-ce qu’une vitesse de propagation d’une onde}
\vspace{0.2cm}

-Le professeur pose la simulation et la  question suivante : 

\emph{Qu’est-ce qu’une vitesse de propagation d’une onde}
\vspace{0.2cm}

-Le professeur pose la simulation et la  question suivante : 

\emph{Qu'est-ce qu'un retard temporaire}





				  &

---\textbf{Interprétation : }

a-On observe l’absence de son après le vidage de
l’air, on conclut que le son ne se
propage pas dans le vide mais il nécessite un milieu
matériel pour se propager.
\vspace{0.2cm}

b- Lorsqu’on frappe le diapason, la balle se déplace
horizontalement, ce qui indique
que la direction de perturbation et celle de
propagation sont alignées, donc le son
est une onde longitudinale.
\vspace{0.5cm}

-Les élèves écrivent une conclusion dans le cahier.
\vspace{0.5cm}


--- L’apprenant répond la question en donnant la
définition d’une onde mécanique :

-\textbf{La vitesse de propagation} d’une onde (nommée célérité) est égale à la distance parcourue au temps mis
à la parcourir.

--- L'apprenant répond à la question en donnant la définition du retard temporaire à partir de la définition de la vitesse :
				  & 
	Évaluation formative\\\hline 
\end{tabular}
\end{center}






\end{document}
