\documentclass[13pt]{article}
\usepackage[a4paper, margin=.20in]{geometry}
%\usepackage{array}
\usepackage{graphicx, subfig, wrapfig, makecell,fancyhdr, xcolor }
\newcommand\headerMe[2]{\noindent{}#1\hfill#2}
\renewcommand \thesection{\Roman{section}}

\newcolumntype{M}[1]{>{\raggedright}m{#1}}

\chead{\includegraphics[width = 0.1\textwidth]{./img/logoMin.png}}
\cfoot{helllo}

\begin{document}

\begin{center}
	\includegraphics[width = 0.18\textwidth]{./img/logoMin.png}
	\vspace{-3cm}
\end{center}
\headerMe{Matière : Physique-Chimie}{Établissement : \emph{Lycée SKHOR qualifiant}}\\
\headerMe{ Unité :  Transformations nucléaires}{  Professeur :\emph{Zakaria Haouzan}}\\
\headerMe{Niveau : TCS}{Heure : 6H}\\

\begin{center}
	\vspace{1cm}
	\underline{Leçon N°3: Principe d’inertie}\\
	Durée 6h00
	\\
	\vspace{.2cm}
	\hrulefill
	\Large{Fiche Pédagogique}
	\hrulefill\\
\end{center}
%end Headerss------------------------


%__________________Chimie ______________________-
%%%%%%%+_+_+_+_+_+_+_+_+_Partie1
\begin{center}
	\begin{tabular}{|p{0.2\textwidth}||p{0.3\textwidth}||p{0.3\textwidth}||p{0.1\textwidth}|}
		\hline
		\textbf{Prérequis} & \textbf{Compétences visées } & \textbf{Savoir et savoir-faire} & \textbf{Outils didactiques } \\
		\hline
		- Notions de forces

		- Vecteurs et opérations vectorielles

		- Notion de mouvement

		- Concepts de vitesse et de trajectoire

		- Géométrie de base
		                   &
		- Comprendre le concept de centre d'inertie

		- Identifier un système isolé ou pseudo-isolé

		- Appliquer le principe d'inertie

		- Analyser les mouvements dans différents référentiels

		- Utiliser la relation barycentrique
		                   &


		- Définir et localiser le centre d'inertie d'un solide

		- Identifier les forces appliquées sur un système

		- Caractériser un référentiel galiléen

		- Utiliser la relation barycentrique

		- Analyser les mouvements d'un solide
		                   &
		- Table à coussin d'air

		- Autoporteur

		- Dispositif d'enregistrement


		\\
		\hline
	\end{tabular}
\end{center}
\section*{Situation-problème :}

Un joueur lance un palet de curling sur le terrain. On observe que son centre d'inertie garde un mouvement rectiligne uniforme tant qu'il ne heurte aucun obstacle.

\begin{enumerate}
	\item Qu'est-ce qu'un centre d'inertie? Comment trouver sa position?
	\item Comment caractériser le mouvement du palet?
	\item Quel principe physique explique ce phénomène?
	\item Un mouvement nécessite-t-il toujours des forces?
\end{enumerate}

\begin{center}
	\begin{tabular}{|p{0.2\textwidth}||p{0.3\textwidth}||p{0.3\textwidth}||p{0.1\textwidth}|}
		\hline
		\multicolumn{4}{|c|}{Déroulement}                                                  \\\hline
		Eléments du & \multicolumn{2}{c||}{Activités didactiques} &                        \\\cline{2-3}
		cours       & Enseignant                                  & Apprenant & Evaluation \\\hline

		\color{red}{ I- Centre d’inertie d’un corps solide :

			\vspace{0.5cm}

			\color{magenta}I-1. Activité 1

			\vspace{0.5cm}

			\color{magenta}I-2. Définition du centre d’inertie G :


		}           &

		- Présenter la situation-problème

		- Inviter les apprenants à formuler des hypothèses

		- Guider la réflexion sur le mouvement du palet

		- Démonstration avec l'autoporteur

		- Analyse des trajectoires

		- Définition du centre d'inertie
		            &
		- Analyser la situation

		- Proposer des hypothèses

		- Participer à la discussion collective

		- Le centre d'inertie est un point particulier du solide

		- Le mouvement peut continuer sans force

		- Les forces se compensent

		- Observer les trajectoires

		- Comparer les mouvements

		- Noter les observations

		- Déduire les caractéristiques du centre d'inertie

		            &
		Evaluation
		diagnostique                                                                       \\\hline


		%II- partie 2 
		\color{red}{II  Principe d'inertie:}

		\vspace{0.5cm}
		\color{blue}1  Activité 2 :

		\color{blue}2 Système isolé et pseudo-isolé :
		\vspace{0.5cm}

		\color{blue}3 Enoncé du principe d’inertie :

		            &
		- Expérience avec l'autoporteur

		- Inventaire des forces

		- Définition des systèmes isolés/pseudo-isolés
		            &
		- Identifier les forces

		- Calculer la somme vectorielle

		- Caractériser le mouvement
		            &
		Évaluation formative
		\\\hline

		\color{red}{III  Relation barycentrique}

		\vspace{0.5cm}
		\color{blue}1  Définition de centre de masse d’un système matériel :

		\color{blue}2 Relation barycentrique :
		\vspace{0.5cm}

		\color{blue}3 Exercice d’application :


		            &
		-Présentation des systèmes matériels

		- Démonstration de la relation

		- Exercices d'application


		            &
		-Comprendre la relation

		- Appliquer aux cas simples

		- Résoudre des exercices
		            & Formative et sommative
		\\\hline
	\end{tabular}
\end{center}

%Page 3 in parte 2



\end{document}
