\documentclass[14pt]{article}
\usepackage[a4paper, margin=.20in]{geometry}
%\usepackage{array}
\usepackage{graphicx, subfig, wrapfig, makecell,fancyhdr, xcolor }
\newcommand\headerMe[2]{\noindent{}#1\hfill#2}
\renewcommand \thesection{\Roman{section}}

\newcolumntype{M}[1]{>{\raggedright}m{#1}}

\chead{\includegraphics[width = 0.1\textwidth]{./img/logoMin.png}}
\cfoot{helllo}

\begin{document}

\begin{center}
\includegraphics[width = 0.18\textwidth]{./img/logoMin.png}
\vspace{-3cm}
\end{center}
\headerMe{Matière : Physique-Chimie}{Établissement : \emph{Lycée SKHOR qualifiant}}\\
\headerMe{ Unité :  Mécanique}{  Professeur :\emph{Zakaria Haouzan}}\\
\headerMe{Niveau : TCS}{Heure : 6H}\\

\begin{center}
	\vspace{1cm}
\underline{Leçon N°3:  Le Mouvement}\\
Durée 6h00
\\
    \vspace{.2cm}
\hrulefill
\Large{Fiche Pédagogique}
\hrulefill\\
\end{center}
%end Headerss------------------------


%__________________Chimie ______________________-
%%%%%%%+_+_+_+_+_+_+_+_+_Partie1
 \begin{center}
	 \begin{tabular}{|p{0.2\textwidth}||p{0.3\textwidth}||p{0.3\textwidth}||p{0.1\textwidth}|}
\hline
\textbf{Prérequis} & \textbf{Compétences visées } & \textbf{Savoir et savoir-faire}&\textbf{Outils didactiques }\\
    \hline
    	
      - Connaissance des bases de la cinématique : mouvement, trajectoire, vitesse.			   

      - Notions d’espace et de temps.

           &
         -Comprendre la relativité du mouvement en fonction du référentiel choisi.
- Identifier et caractériser les types de trajectoires et de mouvements.

     - Appliquer les formules de vitesse moyenne et instantanée.

- Étudier les mouvements rectilignes et circulaires uniformes.
 & 
- Le mouvement est relatif et dépend du référentiel choisi.

 - Les différents types de trajectoires : rectiligne, curviligne, circulaire.

- La vitesse moyenne et instantanée comme grandeurs caractéristiques du mouvement.

- Les particularités des mouvements rectilignes et circulaires uniformes.

- Décrire un mouvement dans un référentiel donné.

- Déterminer la vitesse moyenne et la vitesse instantanée d’un mobile.

- Représenter graphiquement une trajectoire et un vecteur vitesse.

- Résoudre des exercices de calcul relatifs au mouvement rectiligne ou circulaire. 
 & 
- Tableau 

- Caméra pour enregistrer des mouvements.

- Data-show pour les simulations et démonstrations.

- Coussin d’air pour expérimenter les mouvements sans frottement.


     \\
    \hline
\end{tabular} 
\end{center}
\section*{Situation-problème :}
Dans notre vie quotidienne, les mouvements sont relatifs : une voiture, un train, une personne peuvent se déplacer différemment selon le référentiel choisi. Par exemple, un passager dans un train peut sembler immobile par rapport à un autre passager, mais en mouvement pour un observateur sur le quai.

\begin{enumerate}
  \item Comment décrire le mouvement d’un corps par rapport à un référentiel ?
  \item Quelles sont les caractéristiques d’un mouvement (vitesse, trajectoire) ?
  \item Comment calculer la vitesse d’un objet en mouvement rectiligne ou circulaire ?
\end{enumerate}

\begin{center}
	 \begin{tabular}{|p{0.2\textwidth}||p{0.3\textwidth}||p{0.3\textwidth}||p{0.1\textwidth}|}
\hline
\multicolumn{4}{|c|}{Déroulement}\\\hline
Eléments du & \multicolumn{2}{c||}{Activités didactiques} &  \\\cline{2-3}
cours & Enseignant & Apprenant & Evaluation\\\hline

\color{red}{I-Introduction : Relativité du mouvement: 

     }	  &

- Présenter la situation-problème

- Inviter les apprenants à formuler des hypothèses

 - Le professeur guide le débat en amenant les élèves à comprendre que le mouvement dépend du référentiel choisi.

- Le professeur fait un lien avec la situation problème pour illustrer ces référentiels.

- Le professeur décrit différents types de trajectoires :

Rectiligne : Une voiture roulant sur une autoroute droite.

Curviligne : Une balle lancée en cloche.

Circulaire : Un satellite autour de la Terre.


				  &
           - Analyser la situation

           Proposer des hypothèses

Réfléchir sur la situation problème et proposer des hypothèses.

Prendre des notes sur les définitions clés.

Participer activement au débat et poser des questions.

Les élèves prennent des notes et posent des questions sur la relation entre l’espace et le temps.

				  &
				  Evaluation
diagnostique\\\hline


%II- partie 2 
\color{red}{II  La vitesse}


				  &
-Le professeur 
Présenter la définition de la vitesse moyenne

Expliquer les variables 
d (distance parcourue) et 
t (temps écoulé).

Fournir des exemples simples

Poser des questions :
"Pourquoi la vitesse moyenne ne reflète-t-elle pas toujours la vitesse réelle d’un trajet ?"

Introduire la vitesse instantanée comme la vitesse à un moment précis.

Expérience avec la Table à Coussin d’Air :

Installer et expliquer le dispositif.

Lancer un mobile sur la table et enregistrer ses positions successives avec un chronomètre.

Montrer comment mesurer la vitesse instantanée pour différents points du trajet.

"Pourquoi la vitesse instantanée diffère-t-elle de la vitesse moyenne ?"

"Comment peut-on mesurer la vitesse instantanée dans un véhicule ?"

Lien avec le Mouvement Circulaire :

				  &
				  ---Résoudre des exercices simples pour appliquer les concepts théoriques. 

          Écouter attentivement les explications et prendre des notes.

Répondre aux questions posées par le professeur.

Calculer la vitesse moyenne pour les exemples donnés et discuter des résultats.

Observer l’expérience pratique et noter les données collectées.

Calculer la vitesse instantanée à partir des mesures fournies.

Poser des questions sur les différences entre vitesse moyenne et instantanée.

Appliquer la formule pour des exemples concrets, comme le mouvement circulaire.


				  & 
	Évaluation formative			  
  \\\hline

\color{red}{III  Le mouvement rectiligne uniforme : }

&
 Définition de la Trajectoire Rectiligne

 Définition du Mouvement Rectiligne Uniforme

 Expliquer que cela signifie :

Le mobile parcourt des distances égales pendant des intervalles de temps égaux.

Le vecteur vitesse ne change pas.

Introduire l’équation horaire du mouvement rectiligne uniforme :


& 

Noter la définition et les exemples dans leurs cahiers.

Répondre aux questions posées par le professeur.

Proposer d'autres exemples de trajectoires rectilignes.

Noter la définition et les exemples.


&Formative et sommative
\\\hline





\end{tabular}
\end{center}




\end{document}
